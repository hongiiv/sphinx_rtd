%% Generated by Sphinx.
\def\sphinxdocclass{report}
\documentclass[letterpaper,10pt,english]{sphinxmanual}
\ifdefined\pdfpxdimen
   \let\sphinxpxdimen\pdfpxdimen\else\newdimen\sphinxpxdimen
\fi \sphinxpxdimen=.75bp\relax
%% turn off hyperref patch of \index as sphinx.xdy xindy module takes care of
%% suitable \hyperpage mark-up, working around hyperref-xindy incompatibility
\PassOptionsToPackage{hyperindex=false}{hyperref}

\PassOptionsToPackage{warn}{textcomp}

\catcode`^^^^00a0\active\protected\def^^^^00a0{\leavevmode\nobreak\ }
\usepackage{cmap}
\usepackage{fontspec}
\defaultfontfeatures[\rmfamily,\sffamily,\ttfamily]{}
\usepackage{amsmath,amssymb,amstext}
\usepackage{polyglossia}
\setmainlanguage{english}



\usepackage{kotex}
\setmainfont[Mapping=tex-text]{NanumBarunGothic}
\setsansfont[Mapping=tex-text]{Noto Sans CJK KR}
\setmonofont{Monaco}
\setmainhangulfont[Mapping=tex-text]{NanumBarunGothic}
\setsanshangulfont[Mapping=tex-text]{Noto Sans CJK KR}
\setmonohangulfont{Monaco}


\usepackage[Bjornstrup]{fncychap}
\usepackage{sphinx}

\fvset{fontsize=\small}
\usepackage{geometry}


% Include hyperref last.
\usepackage{hyperref}
% Fix anchor placement for figures with captions.
\usepackage{hypcap}% it must be loaded after hyperref.
% Set up styles of URL: it should be placed after hyperref.
\urlstyle{same}

\addto\captionsenglish{\renewcommand{\contentsname}{목차}}

\usepackage{sphinxmessages}
\setcounter{tocdepth}{1}


\usepackage[titles]{tocloft}
\cftsetpnumwidth {1.25cm}\cftsetrmarg{1.5cm}
\setlength{\cftchapnumwidth}{0.75cm}
\setlength{\cftsecindent}{\cftchapnumwidth}
\setlength{\cftsecnumwidth}{1.25cm}


\title{사용자 매뉴얼}
\date{Dec 28, 2020}
\release{0.0.32+6.g3c3d153.dirty}
\author{Changbum Hong}
\newcommand{\sphinxlogo}{\vbox{}}
\renewcommand{\releasename}{Release}
\makeindex
\begin{document}

\pagestyle{empty}
\sphinxmaketitle
\pagestyle{plain}
\sphinxtableofcontents
\pagestyle{normal}
\phantomsection\label{\detokenize{index::doc}}



\chapter{Customization}
\label{\detokenize{customization:customization}}\label{\detokenize{customization:id1}}\label{\detokenize{customization::doc}}
\noindent\sphinxincludegraphics{{screenshot}.png}

안녕하세요. 반갑습니다. 고객 레코드는 파트너 센터에서 가장 중요 한 정보 자산 중 하나입니다. 고객 계정의 데이터베이스를 검색 하거나, 전체 고객 데이터베이스를 내보내거나, 하위 집합을 Excel과 호환 되는 쉼표로 구분 된 값 파일 형식 (.csv)으로 내보낼 수 있습니다. 고객의 구독 정보를 .csv 파일로 내보낼 수도 있습니다.
활동 로그는 또한 고객에 대 한 트랜잭션 및 관리 작업에 대해 내보낼 수 있는 데이터를 제공 합니다. 자세한 내용은 고객 활동 로그 보기를 참조 하세요.

This theme provides a responsive Material Design theme for Sphinx
documentation. It derives heavily from
\sphinxhref{https://squidfunk.github.io/mkdocs-material/}{Material for MkDocs}%
\begin{footnote}[1]\sphinxAtStartFootnote
\sphinxnolinkurl{https://squidfunk.github.io/mkdocs-material/}
%
\end{footnote},
and also uses code from
\sphinxhref{https://github.com/guzzle/guzzle\_sphinx\_theme}{Guzzle Sphinx Theme}%
\begin{footnote}[2]\sphinxAtStartFootnote
\sphinxnolinkurl{https://github.com/guzzle/guzzle\_sphinx\_theme}
%
\end{footnote}.

There are two methods to alter the theme.  The first, and simplest, uses the
options exposed through \sphinxcode{\sphinxupquote{html\_theme\_options}} in \sphinxcode{\sphinxupquote{conf.py}}. This site’s
options are:

\begin{sphinxVerbatim}[commandchars=\\\{\}]
\PYG{n}{html\PYGZus{}theme\PYGZus{}options} \PYG{o}{=} \PYG{p}{\PYGZob{}}
    \PYG{l+s+s1}{\PYGZsq{}}\PYG{l+s+s1}{base\PYGZus{}url}\PYG{l+s+s1}{\PYGZsq{}}\PYG{p}{:} \PYG{l+s+s1}{\PYGZsq{}}\PYG{l+s+s1}{http://bashtage.github.io/sphinx\PYGZhy{}material/}\PYG{l+s+s1}{\PYGZsq{}}\PYG{p}{,}
    \PYG{l+s+s1}{\PYGZsq{}}\PYG{l+s+s1}{repo\PYGZus{}url}\PYG{l+s+s1}{\PYGZsq{}}\PYG{p}{:} \PYG{l+s+s1}{\PYGZsq{}}\PYG{l+s+s1}{https://github.com/bashtage/sphinx\PYGZhy{}material/}\PYG{l+s+s1}{\PYGZsq{}}\PYG{p}{,}
    \PYG{l+s+s1}{\PYGZsq{}}\PYG{l+s+s1}{repo\PYGZus{}name}\PYG{l+s+s1}{\PYGZsq{}}\PYG{p}{:} \PYG{l+s+s1}{\PYGZsq{}}\PYG{l+s+s1}{Material for Sphinx}\PYG{l+s+s1}{\PYGZsq{}}\PYG{p}{,}
    \PYG{l+s+s1}{\PYGZsq{}}\PYG{l+s+s1}{google\PYGZus{}analytics\PYGZus{}account}\PYG{l+s+s1}{\PYGZsq{}}\PYG{p}{:} \PYG{l+s+s1}{\PYGZsq{}}\PYG{l+s+s1}{UA\PYGZhy{}XXXXX}\PYG{l+s+s1}{\PYGZsq{}}\PYG{p}{,}
    \PYG{l+s+s1}{\PYGZsq{}}\PYG{l+s+s1}{html\PYGZus{}minify}\PYG{l+s+s1}{\PYGZsq{}}\PYG{p}{:} \PYG{n+nb+bp}{True}\PYG{p}{,}
    \PYG{l+s+s1}{\PYGZsq{}}\PYG{l+s+s1}{css\PYGZus{}minify}\PYG{l+s+s1}{\PYGZsq{}}\PYG{p}{:} \PYG{n+nb+bp}{True}\PYG{p}{,}
    \PYG{l+s+s1}{\PYGZsq{}}\PYG{l+s+s1}{nav\PYGZus{}title}\PYG{l+s+s1}{\PYGZsq{}}\PYG{p}{:} \PYG{l+s+s1}{\PYGZsq{}}\PYG{l+s+s1}{Material Sphinx Demo}\PYG{l+s+s1}{\PYGZsq{}}\PYG{p}{,}
    \PYG{l+s+s1}{\PYGZsq{}}\PYG{l+s+s1}{logo\PYGZus{}icon}\PYG{l+s+s1}{\PYGZsq{}}\PYG{p}{:} \PYG{l+s+s1}{\PYGZsq{}}\PYG{l+s+s1}{\PYGZam{}\PYGZsh{}xe869}\PYG{l+s+s1}{\PYGZsq{}}\PYG{p}{,}
    \PYG{l+s+s1}{\PYGZsq{}}\PYG{l+s+s1}{globaltoc\PYGZus{}depth}\PYG{l+s+s1}{\PYGZsq{}}\PYG{p}{:} \PYG{l+m+mi}{2}
\PYG{p}{\PYGZcb{}}
\end{sphinxVerbatim}

The complete list of options with detailed explanations appears in
\sphinxcode{\sphinxupquote{theme.conf}}.


\section{Configuration Options}
\label{\detokenize{customization:configuration-options}}\begin{description}
\item[{\sphinxcode{\sphinxupquote{nav\_title}}}] \leavevmode
Set the name to appear in the left sidebar/header. If not provided, uses
html\_short\_title if defined, or html\_title.

\item[{\sphinxcode{\sphinxupquote{touch\_icon}}}] \leavevmode
Path to a touch icon, should be 152x152 or larger.

\item[{\sphinxcode{\sphinxupquote{google\_analytics\_account}}}] \leavevmode
Set to enable google analytics.

\item[{\sphinxcode{\sphinxupquote{repo\_url}}}] \leavevmode
Set the repo url for the link to appear.

\item[{\sphinxcode{\sphinxupquote{repo\_name}}}] \leavevmode
The name of the repo. If must be set if repo\_url is set.

\item[{\sphinxcode{\sphinxupquote{repo\_type}}}] \leavevmode
Must be one of github, gitlab or bitbucket.

\item[{\sphinxcode{\sphinxupquote{base\_url}}}] \leavevmode
Specify a base\_url used to generate sitemap.xml links. If not specified, then
no sitemap will be built.

\item[{\sphinxcode{\sphinxupquote{globaltoc\_depth}}}] \leavevmode
The maximum depth of the global TOC; set it to \sphinxhyphen{}1 to allow unlimited depth.

\item[{\sphinxcode{\sphinxupquote{globaltoc\_collapse}}}] \leavevmode
If true, TOC entries that are not ancestors of the current page are collapsed.

\item[{\sphinxcode{\sphinxupquote{globaltoc\_includehidden}}}] \leavevmode
If true, the global TOC tree will also contain hidden entries.

\item[{\sphinxcode{\sphinxupquote{theme\_color}}}] \leavevmode
The theme color for mobile browsers. Hex Color without the leading \#.

\item[{\sphinxcode{\sphinxupquote{color\_primary}}}] \leavevmode
Primary color. Options are
red, pink, purple, deep\sphinxhyphen{}purple, indigo, blue, light\sphinxhyphen{}blue, cyan,
teal, green, light\sphinxhyphen{}green, lime, yellow, amber, orange, deep\sphinxhyphen{}orange,
brown, grey, blue\sphinxhyphen{}grey, and white.

\item[{\sphinxcode{\sphinxupquote{color\_accent}}}] \leavevmode
Accent color. Options are
red, pink, purple, deep\sphinxhyphen{}purple, indigo, blue, light\sphinxhyphen{}blue, cyan,
teal, green, light\sphinxhyphen{}green, lime, yellow, amber, orange, and deep\sphinxhyphen{}orange.

\item[{\sphinxcode{\sphinxupquote{html\_minify}}}] \leavevmode
Minify pages after creation using htmlmin.

\item[{\sphinxcode{\sphinxupquote{html\_prettify}}}] \leavevmode
Prettify pages, usually only for debugging.

\item[{\sphinxcode{\sphinxupquote{css\_minify}}}] \leavevmode
Minify css files found in the output directory.

\item[{\sphinxcode{\sphinxupquote{logo\_icon}}}] \leavevmode
Set the logo icon. Should be a pre\sphinxhyphen{}escaped html string that indicates a
unicode point, e.g., \sphinxcode{\sphinxupquote{'\&\#xe869'}} which is used on this site.

\item[{\sphinxcode{\sphinxupquote{master\_doc}}}] \leavevmode
Include the master document at the top of the page in the breadcrumb bar.
You must also set this to true if you want to override the rootrellink block, in which
case the content of the overridden block will appear

\item[{\sphinxcode{\sphinxupquote{nav\_links}}}] \leavevmode
A list of dictionaries where each has three keys:
\begin{itemize}
\item {} 
\sphinxcode{\sphinxupquote{href}}: The URL or pagename (str)

\item {} 
\sphinxcode{\sphinxupquote{title}}: The title to appear (str)

\item {} 
\sphinxcode{\sphinxupquote{internal}}: Flag indicating to use pathto to find the page.  Set to False for
external content. (bool)

\end{itemize}

\item[{\sphinxcode{\sphinxupquote{heroes}}}] \leavevmode
A \sphinxcode{\sphinxupquote{dict{[}str,str{]}}} where the key is a pagename and the value is the text to display in the
page’s hero location.

\item[{\sphinxcode{\sphinxupquote{version\_dropdown}}}] \leavevmode
A flag indicating whether the version drop down should be included. You must supply a JSON file
to use this feature.

\item[{\sphinxcode{\sphinxupquote{version\_dropdown\_text}}}] \leavevmode
The text in the version dropdown button

\item[{\sphinxcode{\sphinxupquote{version\_json}}}] \leavevmode
The location of the JSON file that contains the version information. The default assumes there
is a file versions.json located in the root of the site.

\item[{\sphinxcode{\sphinxupquote{version\_info}}}] \leavevmode
A dictionary used to populate the version dropdown.  If this variable is provided, the static
dropdown is used and any JavaScript information is ignored.

\item[{\sphinxcode{\sphinxupquote{table\_classes}}}] \leavevmode
A list of classes to \sphinxstylestrong{not strip} from tables. All other classes are stripped, and the default
table has no class attribute. Custom table classes need to provide the full style for the table.

\end{description}


\section{Sidebars}
\label{\detokenize{customization:sidebars}}
You must set \sphinxcode{\sphinxupquote{html\_sidebars}} in order for the side bar to appear. There are
four in the complete set.

\begin{sphinxVerbatim}[commandchars=\\\{\}]
\PYG{n}{html\PYGZus{}sidebars} \PYG{o}{=} \PYG{p}{\PYGZob{}}
    \PYG{l+s+s2}{\PYGZdq{}}\PYG{l+s+s2}{**}\PYG{l+s+s2}{\PYGZdq{}}\PYG{p}{:} \PYG{p}{[}\PYG{l+s+s2}{\PYGZdq{}}\PYG{l+s+s2}{logo\PYGZhy{}text.html}\PYG{l+s+s2}{\PYGZdq{}}\PYG{p}{,} \PYG{l+s+s2}{\PYGZdq{}}\PYG{l+s+s2}{globaltoc.html}\PYG{l+s+s2}{\PYGZdq{}}\PYG{p}{,} \PYG{l+s+s2}{\PYGZdq{}}\PYG{l+s+s2}{localtoc.html}\PYG{l+s+s2}{\PYGZdq{}}\PYG{p}{,} \PYG{l+s+s2}{\PYGZdq{}}\PYG{l+s+s2}{searchbox.html}\PYG{l+s+s2}{\PYGZdq{}}\PYG{p}{]}
\PYG{p}{\PYGZcb{}}
\end{sphinxVerbatim}

You can exclude any to hide a specific sidebar. For example, if this is changed to

\begin{sphinxVerbatim}[commandchars=\\\{\}]
\PYG{n}{html\PYGZus{}sidebars} \PYG{o}{=} \PYG{p}{\PYGZob{}}
    \PYG{l+s+s2}{\PYGZdq{}}\PYG{l+s+s2}{**}\PYG{l+s+s2}{\PYGZdq{}}\PYG{p}{:} \PYG{p}{[}\PYG{l+s+s2}{\PYGZdq{}}\PYG{l+s+s2}{globaltoc.html}\PYG{l+s+s2}{\PYGZdq{}}\PYG{p}{]}
\PYG{p}{\PYGZcb{}}
\end{sphinxVerbatim}

then only the global ToC would appear on all pages (\sphinxcode{\sphinxupquote{**}} is a glob pattern).


\section{Customizing the layout}
\label{\detokenize{customization:customizing-the-layout}}
You can customize the theme by overriding Jinja template blocks. For example,
‘layout.html’ contains several blocks that can be overridden or extended.

Place a ‘layout.html’ file in your project’s ‘/\_templates’ directory.

\begin{sphinxVerbatim}[commandchars=\\\{\}]
mkdir source/\PYGZus{}templates
touch source/\PYGZus{}templates/layout.html
\end{sphinxVerbatim}

Then, configure your ‘conf.py’:

\begin{sphinxVerbatim}[commandchars=\\\{\}]
\PYG{n}{templates\PYGZus{}path} \PYG{o}{=} \PYG{p}{[}\PYG{l+s+s1}{\PYGZsq{}}\PYG{l+s+s1}{\PYGZus{}templates}\PYG{l+s+s1}{\PYGZsq{}}\PYG{p}{]}
\end{sphinxVerbatim}

Finally, edit your override file \sphinxcode{\sphinxupquote{source/\_templates/layout.html}}:

\begin{sphinxVerbatim}[commandchars=\\\{\}]
\PYG{c}{\PYGZob{}\PYGZsh{} Import the theme\PYGZsq{}s layout. \PYGZsh{}\PYGZcb{}}
\PYG{c+cp}{\PYGZob{}\PYGZpc{}} \PYG{k}{extends} \PYG{l+s+s1}{\PYGZsq{}!layout.html\PYGZsq{}} \PYG{c+cp}{\PYGZpc{}\PYGZcb{}}

\PYG{c+cp}{\PYGZob{}\PYGZpc{}}\PYGZhy{} \PYG{k}{block} \PYG{n+nv}{extrahead} \PYG{c+cp}{\PYGZpc{}\PYGZcb{}}
\PYG{c}{\PYGZob{}\PYGZsh{} Add custom things to the head HTML tag \PYGZsh{}\PYGZcb{}}
\PYG{c}{\PYGZob{}\PYGZsh{} Call the parent block \PYGZsh{}\PYGZcb{}}
\PYG{c+cp}{\PYGZob{}\PYGZob{}} \PYG{n+nb}{super}\PYG{o}{(}\PYG{o}{)} \PYG{c+cp}{\PYGZcb{}\PYGZcb{}}
\PYG{c+cp}{\PYGZob{}\PYGZpc{}}\PYGZhy{} \PYG{k}{endblock} \PYG{c+cp}{\PYGZpc{}\PYGZcb{}}
\end{sphinxVerbatim}


\section{New Blocks}
\label{\detokenize{customization:new-blocks}}
The theme has a small number of new blocks to simplify some types of
customization:
\begin{description}
\item[{\sphinxcode{\sphinxupquote{footerrel}}}] \leavevmode
Previous and next in the footer.

\item[{\sphinxcode{\sphinxupquote{font}}}] \leavevmode
The default font inline CSS and the class to the google API. Use this
block when changing the font.

\item[{\sphinxcode{\sphinxupquote{fonticon}}}] \leavevmode
Block that contains the icon font. Use this to add additional icon fonts
(e.g., \sphinxhref{https://fontawesome.com/}{FontAwesome}%
\begin{footnote}[3]\sphinxAtStartFootnote
\sphinxnolinkurl{https://fontawesome.com/}
%
\end{footnote}). You should probably call \sphinxcode{\sphinxupquote{\{\{ super() \}\}}} at
the end of the block to include the default icon font as well.

\end{description}


\section{Version Dropdown}
\label{\detokenize{customization:version-dropdown}}
A version dropdown is available that lets you store multiple versions in a single site.
The standard structure of the site, relative to the base is usually:

\begin{sphinxVerbatim}[commandchars=\\\{\}]
\PYG{o}{/}
\PYG{o}{/}\PYG{n}{devel}
\PYG{o}{/}\PYG{n}{v1}\PYG{o}{.}\PYG{l+m+mf}{0.0}
\PYG{o}{/}\PYG{n}{v1}\PYG{o}{.}\PYG{l+m+mf}{1.0}
\PYG{o}{/}\PYG{n}{v1}\PYG{o}{.}\PYG{l+m+mf}{1.1}
\PYG{o}{/}\PYG{n}{v1}\PYG{o}{.}\PYG{l+m+mf}{2.0}
\end{sphinxVerbatim}

To use the version dropdown, you must set \sphinxcode{\sphinxupquote{version\_dropdown}} to \sphinxcode{\sphinxupquote{True}} in
the sites configuration.

There are two approaches, one which stores the version information in a JavaScript file
and one which uses a dictionary in the configuration.


\subsection{Using a Javascript File}
\label{\detokenize{customization:using-a-javascript-file}}
The data used is read via javascript from a file. The basic structure of the file is a dictionary of the form {[}label, path{]}.

This dictionary tells the dropdown that the release version is in the root of the site, the
other versions are archived under their version number, and the development version is
located in /devel.

\begin{sphinxadmonition}{note}{Note:}
The advantage of this approach is that you can separate version information
from the rendered documentation.  This makes is easy to change the version
dropdown in \_older\_ versions of the documentation to reflect additional versions
that are released later. Changing the Javascript file changes the version dropdown
content in all versions.  This approach is used in
\sphinxhref{https://www.statsmodels.org/}{statsmodels}%
\begin{footnote}[4]\sphinxAtStartFootnote
\sphinxnolinkurl{https://www.statsmodels.org/}
%
\end{footnote}.
\end{sphinxadmonition}


\subsection{Using \sphinxstyleliteralintitle{\sphinxupquote{conf.py}}}
\label{\detokenize{customization:using-conf-py}}
\begin{sphinxadmonition}{warning}{Warning:}
This method has precedence over the JavaScript approach. If \sphinxcode{\sphinxupquote{version\_info}} is
not empty in a site’s \sphinxcode{\sphinxupquote{html\_theme\_options}}, then the static approach is used.
\end{sphinxadmonition}

The alternative uses a dictionary where the key is the title and the value is the target.
The dictionary is part of the size configuration’s \sphinxcode{\sphinxupquote{html\_theme\_options}}.

The dictionary structure is nearly identical.  Here you can use relative paths
like in the JavaScript version. You can also use absolute paths.

\begin{sphinxadmonition}{note}{Note:}
This approach is easier if you only want to have a fixed set of documentation,
e.g., stable and devel.
\end{sphinxadmonition}


\chapter{Specimen}
\label{\detokenize{specimen:specimen}}\label{\detokenize{specimen::doc}}

\section{Body copy}
\label{\detokenize{specimen:body-copy}}
Lorem ipsum dolor sit amet, consectetur adipiscing elit. Cras arcu
libero, mollis sed massa vel, \sphinxstyleemphasis{ornare viverra ex}. Mauris a ullamcorper
lacus. Nullam urna elit, malesuada eget finibus ut, ullamcorper ac
tortor. Vestibulum sodales pulvinar nisl, pharetra aliquet est. Quisque
volutpat erat ac nisi accumsan tempor.

\sphinxstylestrong{Sed suscipit}, orci non pretium pretium, quam mi gravida metus, vel
venenatis justo est condimentum diam. Maecenas non ornare justo. Nam a
ipsum eros. {\hyperref[\detokenize{specimen:}]{\emph{Nulla aliquam}}} orci sit amet nisl posuere malesuada.
Proin aliquet nulla velit, quis ultricies orci feugiat et.
\sphinxcode{\sphinxupquote{Ut tincidunt sollicitudin}} tincidunt. Aenean ullamcorper sit amet
nulla at interdum.


\section{Headings}
\label{\detokenize{specimen:headings}}

\subsection{The 3rd level}
\label{\detokenize{specimen:the-3rd-level}}

\subsubsection{The 4th level}
\label{\detokenize{specimen:the-4th-level}}

\paragraph{The 5th level}
\label{\detokenize{specimen:the-5th-level}}
The 6th level


\section{Headings with secondary text}
\label{\detokenize{specimen:headings-with-secondary-text}}

\subsection{The 3rd level with secondary text}
\label{\detokenize{specimen:the-3rd-level-with-secondary-text}}

\subsubsection{The 4th level with secondary text}
\label{\detokenize{specimen:the-4th-level-with-secondary-text}}

\paragraph{The 5th level with secondary text}
\label{\detokenize{specimen:the-5th-level-with-secondary-text}}
The 6th level with secondary text


\section{Blockquotes}
\label{\detokenize{specimen:blockquotes}}\begin{quote}

Morbi eget dapibus felis. Vivamus venenatis porttitor tortor sit amet
rutrum. Pellentesque aliquet quam enim, eu volutpat urna rutrum a.
Nam vehicula nunc mauris, a ultricies libero efficitur sed. \sphinxstyleemphasis{Class
aptent} taciti sociosqu ad litora torquent per conubia nostra, per
inceptos himenaeos. Sed molestie imperdiet consectetur.
\end{quote}


\subsection{Blockquote nesting}
\label{\detokenize{specimen:blockquote-nesting}}\begin{quote}

\sphinxstylestrong{Sed aliquet}, neque at rutrum mollis, neque nisi tincidunt nibh,
vitae faucibus lacus nunc at lacus. Nunc scelerisque, quam id cursus
sodales, lorem {\hyperref[\detokenize{specimen:}]{\emph{libero fermentum}}} urna, ut efficitur elit
ligula et nunc.
\end{quote}
\begin{quote}

Mauris dictum mi lacus, sit amet pellentesque urna vehicula
fringilla. Ut sit amet placerat ante. Proin sed elementum nulla.
Nunc vitae sem odio. Suspendisse ac eros arcu. Vivamus orci erat,
volutpat a tempor et, rutrum. eu odio.
\begin{quote}

\sphinxcode{\sphinxupquote{Suspendisse rutrum facilisis risus}}, eu posuere neque
commodo a. Interdum et malesuada fames ac ante ipsum primis in
faucibus. Sed nec leo bibendum, sodales mauris ut, tincidunt
massa.
\end{quote}
\end{quote}


\subsection{Other content blocks}
\label{\detokenize{specimen:other-content-blocks}}\begin{quote}

Vestibulum vitae orci quis ante viverra ultricies ut eget turpis. Sed
eu lectus dapibus, eleifend nulla varius, lobortis turpis. In ac
hendrerit nisl, sit amet laoreet nibh.
\begin{quote}

Praesent at \sphinxcode{\sphinxupquote{\DUrole{keyword}{return} \DUrole{name,other}{target}}}, sodales nibh vel, tempor
felis. Fusce vel lacinia lacus. Suspendisse rhoncus nunc non nisi
iaculis ultrices. Donec consectetur mauris non neque imperdiet,
eget volutpat libero.
\end{quote}

\fvset{hllines={, 8,}}%
\begin{sphinxVerbatim}[commandchars=\\\{\}]
\PYG{k+kd}{var} \PYG{n+nx}{\PYGZus{}extends} \PYG{o}{=} \PYG{k+kd}{function}\PYG{p}{(}\PYG{n+nx}{target}\PYG{p}{)} \PYG{p}{\PYGZob{}}
  \PYG{k}{for} \PYG{p}{(}\PYG{k+kd}{var} \PYG{n+nx}{i} \PYG{o}{=} \PYG{l+m+mi}{1}\PYG{p}{;} \PYG{n+nx}{i} \PYG{o}{\PYGZlt{}} \PYG{n+nx}{arguments}\PYG{p}{.}\PYG{n+nx}{length}\PYG{p}{;} \PYG{n+nx}{i}\PYG{o}{++}\PYG{p}{)} \PYG{p}{\PYGZob{}}
    \PYG{k+kd}{var} \PYG{n+nx}{source} \PYG{o}{=} \PYG{n+nx}{arguments}\PYG{p}{[}\PYG{n+nx}{i}\PYG{p}{]}\PYG{p}{;}
    \PYG{k}{for} \PYG{p}{(}\PYG{k+kd}{var} \PYG{n+nx}{key} \PYG{k}{in} \PYG{n+nx}{source}\PYG{p}{)} \PYG{p}{\PYGZob{}}
      \PYG{n+nx}{target}\PYG{p}{[}\PYG{n+nx}{key}\PYG{p}{]} \PYG{o}{=} \PYG{n+nx}{source}\PYG{p}{[}\PYG{n+nx}{key}\PYG{p}{]}\PYG{p}{;}
    \PYG{p}{\PYGZcb{}}
  \PYG{p}{\PYGZcb{}}
  \PYG{k}{return} \PYG{n+nx}{target}\PYG{p}{;}
\PYG{p}{\PYGZcb{}}\PYG{p}{;}
\end{sphinxVerbatim}
\sphinxresetverbatimhllines
\end{quote}


\section{Lists}
\label{\detokenize{specimen:lists}}

\subsection{Unordered lists}
\label{\detokenize{specimen:unordered-lists}}\begin{itemize}
\item {} 
Sed sagittis eleifend rutrum. Donec vitae suscipit est. Nullam tempus
tellus non sem sollicitudin, quis rutrum leo facilisis. Nulla tempor
lobortis orci, at elementum urna sodales vitae. In in vehicula nulla,
quis ornare libero.
\begin{itemize}
\item {} 
Duis mollis est eget nibh volutpat, fermentum aliquet dui mollis.

\item {} 
Nam vulputate tincidunt fringilla.

\item {} 
Nullam dignissim ultrices urna non auctor.

\end{itemize}

\item {} 
Aliquam metus eros, pretium sed nulla venenatis, faucibus auctor ex.
Proin ut eros sed sapien ullamcorper consequat. Nunc ligula ante,
fringilla at aliquam ac, aliquet sed mauris.

\item {} 
Nulla et rhoncus turpis. Mauris ultricies elementum leo. Duis
efficitur accumsan nibh eu mattis. Vivamus tempus velit eros,
porttitor placerat nibh lacinia sed. Aenean in finibus diam.

\end{itemize}


\subsection{Ordered lists}
\label{\detokenize{specimen:ordered-lists}}\begin{enumerate}
\sphinxsetlistlabels{\arabic}{enumi}{enumii}{}{.}%
\item {} 
Integer vehicula feugiat magna, a mollis tellus. Nam mollis ex ante,
quis elementum eros tempor rutrum. Aenean efficitur lobortis lacinia.
Nulla consectetur feugiat sodales.

\item {} 
Cum sociis natoque penatibus et magnis dis parturient montes,
nascetur ridiculus mus. Aliquam ornare feugiat quam et egestas. Nunc
id erat et quam pellentesque lacinia eu vel odio.
\begin{enumerate}
\sphinxsetlistlabels{\arabic}{enumii}{enumiii}{}{.}%
\item {} 
Vivamus venenatis porttitor tortor sit amet rutrum. Pellentesque
aliquet quam enim, eu volutpat urna rutrum a. Nam vehicula nunc
mauris, a ultricies libero efficitur sed.
\begin{enumerate}
\sphinxsetlistlabels{\arabic}{enumiii}{enumiv}{}{.}%
\item {} 
Mauris dictum mi lacus

\item {} 
Ut sit amet placerat ante

\item {} 
Suspendisse ac eros arcu

\end{enumerate}

\item {} 
Morbi eget dapibus felis. Vivamus venenatis porttitor tortor sit
amet rutrum. Pellentesque aliquet quam enim, eu volutpat urna
rutrum a. Sed aliquet, neque at rutrum mollis, neque nisi
tincidunt nibh.

\item {} 
Pellentesque eget \sphinxcode{\sphinxupquote{\DUrole{keyword,declaration}{var} \DUrole{name,other}{\_extends}}} ornare tellus, ut gravida
mi.

\fvset{hllines={, 1,}}%
\begin{sphinxVerbatim}[commandchars=\\\{\}]
\PYG{k+kd}{var} \PYG{n+nx}{\PYGZus{}extends} \PYG{o}{=} \PYG{k+kd}{function}\PYG{p}{(}\PYG{n+nx}{target}\PYG{p}{)} \PYG{p}{\PYGZob{}}
  \PYG{k}{for} \PYG{p}{(}\PYG{k+kd}{var} \PYG{n+nx}{i} \PYG{o}{=} \PYG{l+m+mi}{1}\PYG{p}{;} \PYG{n+nx}{i} \PYG{o}{\PYGZlt{}} \PYG{n+nx}{arguments}\PYG{p}{.}\PYG{n+nx}{length}\PYG{p}{;} \PYG{n+nx}{i}\PYG{o}{++}\PYG{p}{)} \PYG{p}{\PYGZob{}}
    \PYG{k+kd}{var} \PYG{n+nx}{source} \PYG{o}{=} \PYG{n+nx}{arguments}\PYG{p}{[}\PYG{n+nx}{i}\PYG{p}{]}\PYG{p}{;}
    \PYG{k}{for} \PYG{p}{(}\PYG{k+kd}{var} \PYG{n+nx}{key} \PYG{k}{in} \PYG{n+nx}{source}\PYG{p}{)} \PYG{p}{\PYGZob{}}
      \PYG{n+nx}{target}\PYG{p}{[}\PYG{n+nx}{key}\PYG{p}{]} \PYG{o}{=} \PYG{n+nx}{source}\PYG{p}{[}\PYG{n+nx}{key}\PYG{p}{]}\PYG{p}{;}
    \PYG{p}{\PYGZcb{}}
  \PYG{p}{\PYGZcb{}}
  \PYG{k}{return} \PYG{n+nx}{target}\PYG{p}{;}
\PYG{p}{\PYGZcb{}}\PYG{p}{;}
\end{sphinxVerbatim}
\sphinxresetverbatimhllines

\end{enumerate}

\item {} 
Vivamus id mi enim. Integer id turpis sapien. Ut condimentum lobortis
sagittis. Aliquam purus tellus, faucibus eget urna at, iaculis
venenatis nulla. Vivamus a pharetra leo.

\end{enumerate}


\subsection{Definition lists}
\label{\detokenize{specimen:definition-lists}}\begin{description}
\item[{Lorem ipsum dolor sit amet}] \leavevmode
Sed sagittis eleifend rutrum. Donec vitae suscipit est. Nullam tempus
tellus non sem sollicitudin, quis rutrum leo facilisis. Nulla tempor
lobortis orci, at elementum urna sodales vitae. In in vehicula nulla.

Duis mollis est eget nibh volutpat, fermentum aliquet dui mollis. Nam
vulputate tincidunt fringilla. Nullam dignissim ultrices urna non
auctor.

\item[{Cras arcu libero}] \leavevmode
Aliquam metus eros, pretium sed nulla venenatis, faucibus auctor ex.
Proin ut eros sed sapien ullamcorper consequat. Nunc ligula ante,
fringilla at aliquam ac, aliquet sed mauris.

\end{description}


\section{Code blocks}
\label{\detokenize{specimen:code-blocks}}

\subsection{Inline}
\label{\detokenize{specimen:inline}}
Morbi eget \sphinxcode{\sphinxupquote{dapibus felis}}. Vivamus \sphinxstyleemphasis{``venenatis porttitor``} tortor
sit amet rutrum. Class aptent taciti sociosqu ad litora torquent per
conubia nostra, per inceptos himenaeos. {\hyperref[\detokenize{specimen:}]{\emph{\sphinxcode{\sphinxupquote{Pellentesque aliquet quam enim}}}}},
eu volutpat urna rutrum a.

Nam vehicula nunc \sphinxcode{\sphinxupquote{:::js return target}} mauris, a ultricies libero
efficitur sed. Sed molestie imperdiet consectetur. Vivamus a pharetra
leo. Pellentesque eget ornare tellus, ut gravida mi. Fusce vel lacinia
lacus.


\subsection{Listing}
\label{\detokenize{specimen:listing}}
\fvset{hllines={, 1, 5, 8,}}%
\begin{sphinxVerbatim}[commandchars=\\\{\}]
\PYG{k+kd}{var} \PYG{n+nx}{\PYGZus{}extends} \PYG{o}{=} \PYG{k+kd}{function}\PYG{p}{(}\PYG{n+nx}{target}\PYG{p}{)} \PYG{p}{\PYGZob{}}
  \PYG{k}{for} \PYG{p}{(}\PYG{k+kd}{var} \PYG{n+nx}{i} \PYG{o}{=} \PYG{l+m+mi}{1}\PYG{p}{;} \PYG{n+nx}{i} \PYG{o}{\PYGZlt{}} \PYG{n+nx}{arguments}\PYG{p}{.}\PYG{n+nx}{length}\PYG{p}{;} \PYG{n+nx}{i}\PYG{o}{++}\PYG{p}{)} \PYG{p}{\PYGZob{}}
    \PYG{k+kd}{var} \PYG{n+nx}{source} \PYG{o}{=} \PYG{n+nx}{arguments}\PYG{p}{[}\PYG{n+nx}{i}\PYG{p}{]}\PYG{p}{;}
    \PYG{k}{for} \PYG{p}{(}\PYG{k+kd}{var} \PYG{n+nx}{key} \PYG{k}{in} \PYG{n+nx}{source}\PYG{p}{)} \PYG{p}{\PYGZob{}}
      \PYG{n+nx}{target}\PYG{p}{[}\PYG{n+nx}{key}\PYG{p}{]} \PYG{o}{=} \PYG{n+nx}{source}\PYG{p}{[}\PYG{n+nx}{key}\PYG{p}{]}\PYG{p}{;}
    \PYG{p}{\PYGZcb{}}
  \PYG{p}{\PYGZcb{}}
  \PYG{k}{return} \PYG{n+nx}{target}\PYG{p}{;}
\PYG{p}{\PYGZcb{}}\PYG{p}{;}
\end{sphinxVerbatim}
\sphinxresetverbatimhllines


\section{Horizontal rules}
\label{\detokenize{specimen:horizontal-rules}}
Aenean in finibus diam. Duis mollis est eget nibh volutpat, fermentum
aliquet dui mollis. Nam vulputate tincidunt fringilla. Nullam dignissim
ultrices urna non auctor.


\bigskip\hrule\bigskip


Integer vehicula feugiat magna, a mollis tellus. Nam mollis ex ante,
quis elementum eros tempor rutrum. Aenean efficitur lobortis lacinia.
Nulla consectetur feugiat sodales.


\section{Data tables}
\label{\detokenize{specimen:data-tables}}

\begin{savenotes}\sphinxattablestart
\centering
\begin{tabular}[t]{|*{6}{\X{1}{6}|}}
\hline
\sphinxstyletheadfamily 
Sollicitudo / Pellentesi
&\sphinxstyletheadfamily 
consectetur
&\sphinxstyletheadfamily 
adipiscing
&\sphinxstyletheadfamily 
elit
&\sphinxstyletheadfamily 
arcu
&\sphinxstyletheadfamily 
sed
\\
\hline
Vivamus a pharetra
&
yes
&
yes
&
yes
&
yes
&
yes
\\
\hline
Ornare viverra ex
&
yes
&
yes
&
yes
&
yes
&
yes
\\
\hline
Mauris a ullamcorper
&
yes
&
yes
&
partial
&
yes
&
yes
\\
\hline
Nullam urna elit
&
yes
&
yes
&
yes
&
yes
&
yes
\\
\hline
Malesuada eget finibus
&
yes
&
yes
&
yes
&
yes
&
yes
\\
\hline
Ullamcorper
&
yes
&
yes
&
yes
&
yes
&
yes
\\
\hline
Vestibulum sodales
&
yes
&\begin{itemize}
\item {} 
\end{itemize}
&
yes
&\begin{itemize}
\item {} 
\end{itemize}
&
yes
\\
\hline
Pulvinar nisl
&
yes
&
yes
&
yes
&\begin{itemize}
\item {} 
\end{itemize}
&\begin{itemize}
\item {} 
\end{itemize}
\\
\hline
Pharetra aliquet est
&
yes
&
yes
&
yes
&
yes
&
yes
\\
\hline
Sed suscipit
&
yes
&
yes
&
yes
&
yes
&
yes
\\
\hline
Orci non pretium
&
yes
&
partial
&\begin{itemize}
\item {} 
\end{itemize}
&\begin{itemize}
\item {} 
\end{itemize}
&\begin{itemize}
\item {} 
\end{itemize}
\\
\hline
\end{tabular}
\par
\sphinxattableend\end{savenotes}

Sed sagittis eleifend rutrum. Donec vitae suscipit est. Nullam tempus
tellus non sem sollicitudin, quis rutrum leo facilisis. Nulla tempor
lobortis orci, at elementum urna sodales vitae. In in vehicula nulla,
quis ornare libero.


\begin{savenotes}\sphinxattablestart
\centering
\begin{tabulary}{\linewidth}[t]{|T|T|T|}
\hline
\sphinxstyletheadfamily 
Left
&\sphinxstyletheadfamily 
Center
&\sphinxstyletheadfamily 
Right
\\
\hline
Lorem
&
\sphinxstyleemphasis{dolor}
&
\sphinxcode{\sphinxupquote{amet}}
\\
\hline
{\hyperref[\detokenize{specimen:}]{\emph{ipsum}}}
&
\sphinxstylestrong{sit}
&\\
\hline
\end{tabulary}
\par
\sphinxattableend\end{savenotes}

Vestibulum vitae orci quis ante viverra ultricies ut eget turpis. Sed eu
lectus dapibus, eleifend nulla varius, lobortis turpis. In ac hendrerit
nisl, sit amet laoreet nibh.


\begin{savenotes}\sphinxattablestart
\centering
\begin{tabular}[t]{|\X{30}{100}|\X{70}{100}|}
\hline
\sphinxstyletheadfamily 
Table
&\sphinxstyletheadfamily 
with colgroups (Pandoc)
\\
\hline
Lorem
&
ipsum dolor sit amet.
\\
\hline
Sed sagittis
&
eleifend rutrum. Donec vitae suscipit est.
\\
\hline
\end{tabular}
\par
\sphinxattableend\end{savenotes}

Lorem ipsum dolor sit amet, consectetur adipiscing elit. Vivamus nec ipsum a
eros convallis facilisis eget at leo. Cras eu pulvinar eros, at accumsan dolor.
Ut gravida massa sed eros imperdiet fermentum. Donec ac diam ut lorem consequat
laoreet. Maecenas at ex diam. Phasellus tincidunt orci felis, nec commodo nisl
aliquet ac. Aenean eget ornare tellus. Nullam vel nunc quis nisi sodales
finibus in ut metus. Praesent ultrices mollis leo, auctor volutpat eros
consectetur in. Sed ac odio nisi. Cras aliquet ultrices nisl ac mattis. Nulla a
dui velit. Proin et ipsum quis metus auctor viverra. Proin suscipit massa quis
magna mattis, vel tincidunt quam tincidunt. Vestibulum nec feugiat metus, nec
scelerisque eros. Ut ultricies ornare aliquam.


\section{Section II}
\label{\detokenize{specimen:section-ii}}
Proin ac mi tempor, ullamcorper ante at, sodales augue. Duis enim turpis,
volutpat eget consectetur id, facilisis vel nisl. Sed non leo aliquam, tempus
nisl eu, vestibulum enim. Suspendisse et leo imperdiet, pulvinar lacus sed,
commodo felis.

\begin{sphinxadmonition}{note}{Note:}
Praesent elit mi, pretium nec pellentesque eget, ultricies
euismod turpis.
\end{sphinxadmonition}


\subsection{Sub section}
\label{\detokenize{specimen:sub-section}}
In lobortis elementum tempus. Nam facilisis orci neque, eget vestibulum lectus
imperdiet sed. Aenean ac eros sollicitudin, accumsan turpis ac, faucibus arcu.


\section{Section III}
\label{\detokenize{specimen:section-iii}}
Donec sodales, velit ac sagittis fermentum, metus ante pharetra ex, ac eleifend
erat ligula in lacus. Donec tincidunt urna est, non mollis turpis lacinia sit
amet. Duis ac facilisis libero, ut interdum nibh. Sed rutrum dapibus pharetra.
Ut ac luctus nisi, vitae volutpat arcu. Vivamus venenatis eu nibh ut
consectetur. Cras tincidunt dui nisi, et facilisis eros feugiat nec.

Fusce ante:
\begin{itemize}
\item {} 
libero

\item {} 
consequat quis facilisis id

\item {} 
sollicitudin et nisl.

\end{itemize}

Aliquam diam mi, vulputate nec posuere id, consequat id elit. Ut feugiat lectus
quam, sed aliquet augue placerat nec. Sed volutpat leo ac condimentum
ullamcorper. Vivamus eleifend magna tellus, sit amet porta nunc ultrices eget.
Nullam id laoreet ex. Nam ultricies, ante et molestie mollis, magna sem porta
libero, sed molestie neque risus ut purus. Ut tellus sapien, auctor a lacus eu,
iaculis congue ex.

Duis et nisi a odio \sphinxstylestrong{scelerisque} sodales ac ut sapien. Ut eleifend blandit
velit luctus euismod. Curabitur at pulvinar mi. Cras molestie lorem non accumsan
gravida. Sed vulputate, ligula ut tincidunt congue, metus risus luctus lacus,
sed rhoncus ligula turpis non erat. Phasellus est est, \sphinxstyleemphasis{sollicitudin ut}
elementum vel, placerat in orci. Proin molestie posuere dolor sit amet
convallis. Donec id urna vel lacus ultrices pulvinar sit amet id metus. Donec
in venenatis ante. Nam eu rhoncus leo. Quisque posuere, leo vel porttitor
malesuada, nisi urna dignissim justo, vel consectetur purus elit in mauris.
Vestibulum lectus arcu, varius ut ligula quis, viverra gravida sem.

\begin{sphinxadmonition}{warning}{Warning:}
Pellentesque in enim leo.
\end{sphinxadmonition}


\section{Images}
\label{\detokenize{specimen:images}}

\subsection{Default Alignment}
\label{\detokenize{specimen:default-alignment}}
\noindent\sphinxincludegraphics{{desert-flower}.jpg}


\subsection{Center Alignment}
\label{\detokenize{specimen:center-alignment}}
\noindent{\hspace*{\fill}\sphinxincludegraphics[scale=0.8]{{desert-flower}.jpg}\hspace*{\fill}}


\subsection{Right Alignment}
\label{\detokenize{specimen:right-alignment}}
\noindent{\hspace*{\fill}\sphinxincludegraphics[scale=0.6]{{desert-flower}.jpg}}


\chapter{NumPy Docstrings}
\label{\detokenize{numpydoc:numpy-docstrings}}\label{\detokenize{numpydoc::doc}}
This page shows how \sphinxcode{\sphinxupquote{autosummary}} works with \sphinxcode{\sphinxupquote{numpydoc}} and a
NumPy\sphinxhyphen{}style docstring.

\#    notebook.ipynb
\#    markdown.md
\#    rst\sphinxhyphen{}cheatsheet/rst\sphinxhyphen{}cheatsheet

\#.. toctree::
\#    :caption: Changes and License
\#    :maxdepth: 2

\#    change\sphinxhyphen{}log
\#    license

\#Index
\#\textasciitilde{}\textasciitilde{}\textasciitilde{}\textasciitilde{}\textasciitilde{}
\#:ref:\sphinxtitleref{genindex}



\renewcommand{\indexname}{Index}
\footnotesize\raggedright\printindex
\end{document}
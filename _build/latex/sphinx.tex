%% Generated by Sphinx.
\def\sphinxdocclass{report}
\documentclass[letterpaper,10pt,english]{sphinxmanual}
\ifdefined\pdfpxdimen
   \let\sphinxpxdimen\pdfpxdimen\else\newdimen\sphinxpxdimen
\fi \sphinxpxdimen=.75bp\relax
%% turn off hyperref patch of \index as sphinx.xdy xindy module takes care of
%% suitable \hyperpage mark-up, working around hyperref-xindy incompatibility
\PassOptionsToPackage{hyperindex=false}{hyperref}

\PassOptionsToPackage{warn}{textcomp}

\catcode`^^^^00a0\active\protected\def^^^^00a0{\leavevmode\nobreak\ }
\usepackage{cmap}
\usepackage{fontspec}
\defaultfontfeatures[\rmfamily,\sffamily,\ttfamily]{}
\usepackage{amsmath,amssymb,amstext}
\usepackage{polyglossia}
\setmainlanguage{english}



\usepackage{kotex}
\setmainfont[Mapping=tex-text]{NanumBarunGothic}
\setsansfont[Mapping=tex-text]{Noto Sans CJK KR}
\setmonofont{Monaco}
\setmainhangulfont[Mapping=tex-text]{NanumBarunGothic}
\setsanshangulfont[Mapping=tex-text]{Noto Sans CJK KR}
\setmonohangulfont{Monaco}


\usepackage[Bjornstrup]{fncychap}
\usepackage{sphinx}

\fvset{fontsize=\small}
\usepackage{geometry}


% Include hyperref last.
\usepackage{hyperref}
% Fix anchor placement for figures with captions.
\usepackage{hypcap}% it must be loaded after hyperref.
% Set up styles of URL: it should be placed after hyperref.
\urlstyle{same}

\addto\captionsenglish{\renewcommand{\contentsname}{목차}}

\usepackage{sphinxmessages}
\setcounter{tocdepth}{1}


% Jupyter Notebook code cell colors
\definecolor{nbsphinxin}{HTML}{307FC1}
\definecolor{nbsphinxout}{HTML}{BF5B3D}
\definecolor{nbsphinx-code-bg}{HTML}{F5F5F5}
\definecolor{nbsphinx-code-border}{HTML}{E0E0E0}
\definecolor{nbsphinx-stderr}{HTML}{FFDDDD}
% ANSI colors for output streams and traceback highlighting
\definecolor{ansi-black}{HTML}{3E424D}
\definecolor{ansi-black-intense}{HTML}{282C36}
\definecolor{ansi-red}{HTML}{E75C58}
\definecolor{ansi-red-intense}{HTML}{B22B31}
\definecolor{ansi-green}{HTML}{00A250}
\definecolor{ansi-green-intense}{HTML}{007427}
\definecolor{ansi-yellow}{HTML}{DDB62B}
\definecolor{ansi-yellow-intense}{HTML}{B27D12}
\definecolor{ansi-blue}{HTML}{208FFB}
\definecolor{ansi-blue-intense}{HTML}{0065CA}
\definecolor{ansi-magenta}{HTML}{D160C4}
\definecolor{ansi-magenta-intense}{HTML}{A03196}
\definecolor{ansi-cyan}{HTML}{60C6C8}
\definecolor{ansi-cyan-intense}{HTML}{258F8F}
\definecolor{ansi-white}{HTML}{C5C1B4}
\definecolor{ansi-white-intense}{HTML}{A1A6B2}
\definecolor{ansi-default-inverse-fg}{HTML}{FFFFFF}
\definecolor{ansi-default-inverse-bg}{HTML}{000000}

% Define an environment for non-plain-text code cell outputs (e.g. images)
\makeatletter
\newenvironment{nbsphinxfancyoutput}{%
    % Avoid fatal error with framed.sty if graphics too long to fit on one page
    \let\sphinxincludegraphics\nbsphinxincludegraphics
    \nbsphinx@image@maxheight\textheight
    \advance\nbsphinx@image@maxheight -2\fboxsep   % default \fboxsep 3pt
    \advance\nbsphinx@image@maxheight -2\fboxrule  % default \fboxrule 0.4pt
    \advance\nbsphinx@image@maxheight -\baselineskip
\def\nbsphinxfcolorbox{\spx@fcolorbox{nbsphinx-code-border}{white}}%
\def\FrameCommand{\nbsphinxfcolorbox\nbsphinxfancyaddprompt\@empty}%
\def\FirstFrameCommand{\nbsphinxfcolorbox\nbsphinxfancyaddprompt\sphinxVerbatim@Continues}%
\def\MidFrameCommand{\nbsphinxfcolorbox\sphinxVerbatim@Continued\sphinxVerbatim@Continues}%
\def\LastFrameCommand{\nbsphinxfcolorbox\sphinxVerbatim@Continued\@empty}%
\MakeFramed{\advance\hsize-\width\@totalleftmargin\z@\linewidth\hsize\@setminipage}%
\lineskip=1ex\lineskiplimit=1ex\raggedright%
}{\par\unskip\@minipagefalse\endMakeFramed}
\makeatother
\newbox\nbsphinxpromptbox
\def\nbsphinxfancyaddprompt{\ifvoid\nbsphinxpromptbox\else
    \kern\fboxrule\kern\fboxsep
    \copy\nbsphinxpromptbox
    \kern-\ht\nbsphinxpromptbox\kern-\dp\nbsphinxpromptbox
    \kern-\fboxsep\kern-\fboxrule\nointerlineskip
    \fi}
\newlength\nbsphinxcodecellspacing
\setlength{\nbsphinxcodecellspacing}{0pt}

% Define support macros for attaching opening and closing lines to notebooks
\newsavebox\nbsphinxbox
\makeatletter
\newcommand{\nbsphinxstartnotebook}[1]{%
    \par
    % measure needed space
    \setbox\nbsphinxbox\vtop{{#1\par}}
    % reserve some space at bottom of page, else start new page
    \needspace{\dimexpr2.5\baselineskip+\ht\nbsphinxbox+\dp\nbsphinxbox}
    % mimick vertical spacing from \section command
      \addpenalty\@secpenalty
      \@tempskipa 3.5ex \@plus 1ex \@minus .2ex\relax
      \addvspace\@tempskipa
      {\Large\@tempskipa\baselineskip
             \advance\@tempskipa-\prevdepth
             \advance\@tempskipa-\ht\nbsphinxbox
             \ifdim\@tempskipa>\z@
               \vskip \@tempskipa
             \fi}
    \unvbox\nbsphinxbox
    % if notebook starts with a \section, prevent it from adding extra space
    \@nobreaktrue\everypar{\@nobreakfalse\everypar{}}%
    % compensate the parskip which will get inserted by next paragraph
    \nobreak\vskip-\parskip
    % do not break here
    \nobreak
}% end of \nbsphinxstartnotebook

\newcommand{\nbsphinxstopnotebook}[1]{%
    \par
    % measure needed space
    \setbox\nbsphinxbox\vbox{{#1\par}}
    \nobreak % it updates page totals
    \dimen@\pagegoal
    \advance\dimen@-\pagetotal \advance\dimen@-\pagedepth
    \advance\dimen@-\ht\nbsphinxbox \advance\dimen@-\dp\nbsphinxbox
    \ifdim\dimen@<\z@
      % little space left
      \unvbox\nbsphinxbox
      \kern-.8\baselineskip
      \nobreak\vskip\z@\@plus1fil
      \penalty100
      \vskip\z@\@plus-1fil
      \kern.8\baselineskip
    \else
      \unvbox\nbsphinxbox
    \fi
}% end of \nbsphinxstopnotebook

% Ensure height of an included graphics fits in nbsphinxfancyoutput frame
\newdimen\nbsphinx@image@maxheight % set in nbsphinxfancyoutput environment
\newcommand*{\nbsphinxincludegraphics}[2][]{%
    \gdef\spx@includegraphics@options{#1}%
    \setbox\spx@image@box\hbox{\includegraphics[#1,draft]{#2}}%
    \in@false
    \ifdim \wd\spx@image@box>\linewidth
      \g@addto@macro\spx@includegraphics@options{,width=\linewidth}%
      \in@true
    \fi
    % no rotation, no need to worry about depth
    \ifdim \ht\spx@image@box>\nbsphinx@image@maxheight
      \g@addto@macro\spx@includegraphics@options{,height=\nbsphinx@image@maxheight}%
      \in@true
    \fi
    \ifin@
      \g@addto@macro\spx@includegraphics@options{,keepaspectratio}%
    \fi
    \setbox\spx@image@box\box\voidb@x % clear memory
    \expandafter\includegraphics\expandafter[\spx@includegraphics@options]{#2}%
}% end of "\MakeFrame"-safe variant of \sphinxincludegraphics
\makeatother

\makeatletter
\renewcommand*\sphinx@verbatim@nolig@list{\do\'\do\`}
\begingroup
\catcode`'=\active
\let\nbsphinx@noligs\@noligs
\g@addto@macro\nbsphinx@noligs{\let'\PYGZsq}
\endgroup
\makeatother
\renewcommand*\sphinxbreaksbeforeactivelist{\do\<\do\"\do\'}
\renewcommand*\sphinxbreaksafteractivelist{\do\.\do\,\do\:\do\;\do\?\do\!\do\/\do\>\do\-}
\makeatletter
\fvset{codes*=\sphinxbreaksattexescapedchars\do\^\^\let\@noligs\nbsphinx@noligs}
\makeatother


\usepackage[titles]{tocloft}
\cftsetpnumwidth {1.25cm}\cftsetrmarg{1.5cm}
\setlength{\cftchapnumwidth}{0.75cm}
\setlength{\cftsecindent}{\cftchapnumwidth}
\setlength{\cftsecnumwidth}{1.25cm}


\title{사용자 매뉴얼}
\date{Dec 30, 2020}
\release{0.0.32+6.g3c3d153.dirty}
\author{Changbum Hong}
\newcommand{\sphinxlogo}{\vbox{}}
\renewcommand{\releasename}{Release}
\makeindex
\begin{document}

\pagestyle{empty}
\sphinxmaketitle
\pagestyle{plain}
\sphinxtableofcontents
\pagestyle{normal}
\phantomsection\label{\detokenize{index::doc}}



\chapter{Overview}
\label{\detokenize{customization:overview}}\label{\detokenize{customization:customization}}\label{\detokenize{customization::doc}}
최근 comprehensive genomic profiling을 사용한 대규모 코호트 연구들에 따르면 90\%가 유용한 alteration을 가지고 있다고 보고되고 있다cite\{Priestley:2019bp\}. 엔젠바이오는 연구자들을 돕기 위해 323개의 암관련 유전자 (225 coding exon, 98 hotspot cover)를 분석하는 ONCOaccuPanel을 제공한다. small nucleotide variants(SNVs), insertions/deletions(indels), copy\sphinxhyphen{}number variations(CNVs), splice variants, fusions과 함께 최근 다수의 genomic loci 분석을 기반으로 하는 tumor mutational burden(TMB) and microsatellite instability(MSI) 분석을 제공한다.


\begin{savenotes}\sphinxattablestart
\centering
\begin{tabular}[t]{|\X{30}{100}|\X{70}{100}|}
\hline
\sphinxstyletheadfamily 
Variant type
&\sphinxstyletheadfamily 
Relevant exmaples
\\
\hline
SNVs and indels
&
KRAS G12D, EGFR exon 19 deletions, BRAF V600E
\\
\hline
Fusions
&
EGFR, ROS1, RET, ALK, NTRK1
\\
\hline
Splice variants
&
MET exon 14
\\
\hline
CNVs
&
HER2
\\
\hline
MSI %
\begin{footnote}[1]\sphinxAtStartFootnote
A numerical footnote.
%
\end{footnote}
&
MSI\sphinxhyphen{}HIGH
\\
\hline
TMB %
\begin{footnote}[2]\sphinxAtStartFootnote
모든 암종에 대해서 기준값 (>23 Muts/Mb)을 제공하지만, TMB의 high/low는 암 종마다 기준값이 달라질 수 있음
%
\end{footnote}
&
TMB\sphinxhyphen{}HIGH
\\
\hline
\end{tabular}
\par
\sphinxattableend\end{savenotes}

\begin{sphinxadmonition}{note}{Note:}
로그인 창이 나타나지 않거나 로그인 버튼을 눌러도 연결되지 않는 경우에는 인터넷 또는 기관 내부망(인트라넷) 연결이 활성화 되었는지 확인하거나, 아이디 및 비밀번호를 올바르게 입력했는지 확인한다.
\end{sphinxadmonition}

\begin{sphinxadmonition}{warning}{Warning:}\begin{enumerate}
\sphinxsetlistlabels{\arabic}{enumi}{enumii}{}{.}%
\item {} 
선택한 암종에 따라 해석(interpreation)의 tier 정보가 달라진다.

\item {} 
선택한 암종에 따라 white list가 달라진다.

\end{enumerate}
\end{sphinxadmonition}
\begin{enumerate}
\sphinxsetlistlabels{\arabic}{enumi}{enumii}{}{.}%
\item {} 
NGS를 수행한 RUN 이름을 입력한다.

\item {} 
“Local Fastq Files”을 선택하여 로컬 컴퓨터의 FASTQ 파일 또는
“Server Fastq Files”를 선택1하여 원격지의 FASTQ 파일을 선택한다.

\item {} 
분석할 FASTQ 파일을 선택한 후, 열기 버튼을 클릭한다.

\end{enumerate}

This theme provides a responsive Material Design theme for Sphinx
documentation. It derives heavily from
\sphinxhref{https://squidfunk.github.io/mkdocs-material/}{Material for MkDocs}%
\begin{footnote}[3]\sphinxAtStartFootnote
\sphinxnolinkurl{https://squidfunk.github.io/mkdocs-material/}
%
\end{footnote},
and also uses code from
\sphinxhref{https://github.com/guzzle/guzzle\_sphinx\_theme}{Guzzle Sphinx Theme}%
\begin{footnote}[4]\sphinxAtStartFootnote
\sphinxnolinkurl{https://github.com/guzzle/guzzle\_sphinx\_theme}
%
\end{footnote}.

There are two methods to alter the theme.  The first, and simplest, uses the
options exposed through \sphinxcode{\sphinxupquote{html\_theme\_options}} in \sphinxcode{\sphinxupquote{conf.py}}. This site’s
options are:

\begin{sphinxVerbatim}[commandchars=\\\{\}]
\PYG{n}{html\PYGZus{}theme\PYGZus{}options} \PYG{o}{=} \PYG{p}{\PYGZob{}}
    \PYG{l+s+s1}{\PYGZsq{}}\PYG{l+s+s1}{base\PYGZus{}url}\PYG{l+s+s1}{\PYGZsq{}}\PYG{p}{:} \PYG{l+s+s1}{\PYGZsq{}}\PYG{l+s+s1}{http://bashtage.github.io/sphinx\PYGZhy{}material/}\PYG{l+s+s1}{\PYGZsq{}}\PYG{p}{,}
    \PYG{l+s+s1}{\PYGZsq{}}\PYG{l+s+s1}{repo\PYGZus{}url}\PYG{l+s+s1}{\PYGZsq{}}\PYG{p}{:} \PYG{l+s+s1}{\PYGZsq{}}\PYG{l+s+s1}{https://github.com/bashtage/sphinx\PYGZhy{}material/}\PYG{l+s+s1}{\PYGZsq{}}\PYG{p}{,}
    \PYG{l+s+s1}{\PYGZsq{}}\PYG{l+s+s1}{repo\PYGZus{}name}\PYG{l+s+s1}{\PYGZsq{}}\PYG{p}{:} \PYG{l+s+s1}{\PYGZsq{}}\PYG{l+s+s1}{Material for Sphinx}\PYG{l+s+s1}{\PYGZsq{}}\PYG{p}{,}
    \PYG{l+s+s1}{\PYGZsq{}}\PYG{l+s+s1}{google\PYGZus{}analytics\PYGZus{}account}\PYG{l+s+s1}{\PYGZsq{}}\PYG{p}{:} \PYG{l+s+s1}{\PYGZsq{}}\PYG{l+s+s1}{UA\PYGZhy{}XXXXX}\PYG{l+s+s1}{\PYGZsq{}}\PYG{p}{,}
    \PYG{l+s+s1}{\PYGZsq{}}\PYG{l+s+s1}{html\PYGZus{}minify}\PYG{l+s+s1}{\PYGZsq{}}\PYG{p}{:} \PYG{k+kc}{True}\PYG{p}{,}
    \PYG{l+s+s1}{\PYGZsq{}}\PYG{l+s+s1}{css\PYGZus{}minify}\PYG{l+s+s1}{\PYGZsq{}}\PYG{p}{:} \PYG{k+kc}{True}\PYG{p}{,}
    \PYG{l+s+s1}{\PYGZsq{}}\PYG{l+s+s1}{nav\PYGZus{}title}\PYG{l+s+s1}{\PYGZsq{}}\PYG{p}{:} \PYG{l+s+s1}{\PYGZsq{}}\PYG{l+s+s1}{Material Sphinx Demo}\PYG{l+s+s1}{\PYGZsq{}}\PYG{p}{,}
    \PYG{l+s+s1}{\PYGZsq{}}\PYG{l+s+s1}{logo\PYGZus{}icon}\PYG{l+s+s1}{\PYGZsq{}}\PYG{p}{:} \PYG{l+s+s1}{\PYGZsq{}}\PYG{l+s+s1}{\PYGZam{}\PYGZsh{}xe869}\PYG{l+s+s1}{\PYGZsq{}}\PYG{p}{,}
    \PYG{l+s+s1}{\PYGZsq{}}\PYG{l+s+s1}{globaltoc\PYGZus{}depth}\PYG{l+s+s1}{\PYGZsq{}}\PYG{p}{:} \PYG{l+m+mi}{2}
\PYG{p}{\PYGZcb{}}
\end{sphinxVerbatim}

The complete list of options with detailed explanations appears in
\sphinxcode{\sphinxupquote{theme.conf}}.


\section{Configuration Options}
\label{\detokenize{customization:configuration-options}}\begin{description}
\item[{\sphinxcode{\sphinxupquote{nav\_title}}}] \leavevmode
Set the name to appear in the left sidebar/header. If not provided, uses
html\_short\_title if defined, or html\_title.

\item[{\sphinxcode{\sphinxupquote{touch\_icon}}}] \leavevmode
Path to a touch icon, should be 152x152 or larger.

\item[{\sphinxcode{\sphinxupquote{google\_analytics\_account}}}] \leavevmode
Set to enable google analytics.

\item[{\sphinxcode{\sphinxupquote{repo\_url}}}] \leavevmode
Set the repo url for the link to appear.

\item[{\sphinxcode{\sphinxupquote{repo\_name}}}] \leavevmode
The name of the repo. If must be set if repo\_url is set.

\item[{\sphinxcode{\sphinxupquote{repo\_type}}}] \leavevmode
Must be one of github, gitlab or bitbucket.

\item[{\sphinxcode{\sphinxupquote{base\_url}}}] \leavevmode
Specify a base\_url used to generate sitemap.xml links. If not specified, then
no sitemap will be built.

\item[{\sphinxcode{\sphinxupquote{globaltoc\_depth}}}] \leavevmode
The maximum depth of the global TOC; set it to \sphinxhyphen{}1 to allow unlimited depth.

\item[{\sphinxcode{\sphinxupquote{globaltoc\_collapse}}}] \leavevmode
If true, TOC entries that are not ancestors of the current page are collapsed.

\item[{\sphinxcode{\sphinxupquote{globaltoc\_includehidden}}}] \leavevmode
If true, the global TOC tree will also contain hidden entries.

\item[{\sphinxcode{\sphinxupquote{theme\_color}}}] \leavevmode
The theme color for mobile browsers. Hex Color without the leading \#.

\item[{\sphinxcode{\sphinxupquote{color\_primary}}}] \leavevmode
Primary color. Options are
red, pink, purple, deep\sphinxhyphen{}purple, indigo, blue, light\sphinxhyphen{}blue, cyan,
teal, green, light\sphinxhyphen{}green, lime, yellow, amber, orange, deep\sphinxhyphen{}orange,
brown, grey, blue\sphinxhyphen{}grey, and white.

\item[{\sphinxcode{\sphinxupquote{color\_accent}}}] \leavevmode
Accent color. Options are
red, pink, purple, deep\sphinxhyphen{}purple, indigo, blue, light\sphinxhyphen{}blue, cyan,
teal, green, light\sphinxhyphen{}green, lime, yellow, amber, orange, and deep\sphinxhyphen{}orange.

\item[{\sphinxcode{\sphinxupquote{html\_minify}}}] \leavevmode
Minify pages after creation using htmlmin.

\item[{\sphinxcode{\sphinxupquote{html\_prettify}}}] \leavevmode
Prettify pages, usually only for debugging.

\item[{\sphinxcode{\sphinxupquote{css\_minify}}}] \leavevmode
Minify css files found in the output directory.

\item[{\sphinxcode{\sphinxupquote{logo\_icon}}}] \leavevmode
Set the logo icon. Should be a pre\sphinxhyphen{}escaped html string that indicates a
unicode point, e.g., \sphinxcode{\sphinxupquote{'\&\#xe869'}} which is used on this site.

\item[{\sphinxcode{\sphinxupquote{master\_doc}}}] \leavevmode
Include the master document at the top of the page in the breadcrumb bar.
You must also set this to true if you want to override the rootrellink block, in which
case the content of the overridden block will appear

\item[{\sphinxcode{\sphinxupquote{nav\_links}}}] \leavevmode
A list of dictionaries where each has three keys:
\begin{itemize}
\item {} 
\sphinxcode{\sphinxupquote{href}}: The URL or pagename (str)

\item {} 
\sphinxcode{\sphinxupquote{title}}: The title to appear (str)

\item {} 
\sphinxcode{\sphinxupquote{internal}}: Flag indicating to use pathto to find the page.  Set to False for
external content. (bool)

\end{itemize}

\item[{\sphinxcode{\sphinxupquote{heroes}}}] \leavevmode
A \sphinxcode{\sphinxupquote{dict{[}str,str{]}}} where the key is a pagename and the value is the text to display in the
page’s hero location.

\item[{\sphinxcode{\sphinxupquote{version\_dropdown}}}] \leavevmode
A flag indicating whether the version drop down should be included. You must supply a JSON file
to use this feature.

\item[{\sphinxcode{\sphinxupquote{version\_dropdown\_text}}}] \leavevmode
The text in the version dropdown button

\item[{\sphinxcode{\sphinxupquote{version\_json}}}] \leavevmode
The location of the JSON file that contains the version information. The default assumes there
is a file versions.json located in the root of the site.

\item[{\sphinxcode{\sphinxupquote{version\_info}}}] \leavevmode
A dictionary used to populate the version dropdown.  If this variable is provided, the static
dropdown is used and any JavaScript information is ignored.

\item[{\sphinxcode{\sphinxupquote{table\_classes}}}] \leavevmode
A list of classes to \sphinxstylestrong{not strip} from tables. All other classes are stripped, and the default
table has no class attribute. Custom table classes need to provide the full style for the table.

\end{description}


\section{Sidebars}
\label{\detokenize{customization:sidebars}}
You must set \sphinxcode{\sphinxupquote{html\_sidebars}} in order for the side bar to appear. There are
four in the complete set.

\begin{sphinxVerbatim}[commandchars=\\\{\}]
\PYG{n}{html\PYGZus{}sidebars} \PYG{o}{=} \PYG{p}{\PYGZob{}}
    \PYG{l+s+s2}{\PYGZdq{}}\PYG{l+s+s2}{**}\PYG{l+s+s2}{\PYGZdq{}}\PYG{p}{:} \PYG{p}{[}\PYG{l+s+s2}{\PYGZdq{}}\PYG{l+s+s2}{logo\PYGZhy{}text.html}\PYG{l+s+s2}{\PYGZdq{}}\PYG{p}{,} \PYG{l+s+s2}{\PYGZdq{}}\PYG{l+s+s2}{globaltoc.html}\PYG{l+s+s2}{\PYGZdq{}}\PYG{p}{,} \PYG{l+s+s2}{\PYGZdq{}}\PYG{l+s+s2}{localtoc.html}\PYG{l+s+s2}{\PYGZdq{}}\PYG{p}{,} \PYG{l+s+s2}{\PYGZdq{}}\PYG{l+s+s2}{searchbox.html}\PYG{l+s+s2}{\PYGZdq{}}\PYG{p}{]}
\PYG{p}{\PYGZcb{}}
\end{sphinxVerbatim}

You can exclude any to hide a specific sidebar. For example, if this is changed to

\begin{sphinxVerbatim}[commandchars=\\\{\}]
\PYG{n}{html\PYGZus{}sidebars} \PYG{o}{=} \PYG{p}{\PYGZob{}}
    \PYG{l+s+s2}{\PYGZdq{}}\PYG{l+s+s2}{**}\PYG{l+s+s2}{\PYGZdq{}}\PYG{p}{:} \PYG{p}{[}\PYG{l+s+s2}{\PYGZdq{}}\PYG{l+s+s2}{globaltoc.html}\PYG{l+s+s2}{\PYGZdq{}}\PYG{p}{]}
\PYG{p}{\PYGZcb{}}
\end{sphinxVerbatim}

then only the global ToC would appear on all pages (\sphinxcode{\sphinxupquote{**}} is a glob pattern).


\section{Customizing the layout}
\label{\detokenize{customization:customizing-the-layout}}
You can customize the theme by overriding Jinja template blocks. For example,
‘layout.html’ contains several blocks that can be overridden or extended.

Place a ‘layout.html’ file in your project’s ‘/\_templates’ directory.

\begin{sphinxVerbatim}[commandchars=\\\{\}]
mkdir source/\PYGZus{}templates
touch source/\PYGZus{}templates/layout.html
\end{sphinxVerbatim}

Then, configure your ‘conf.py’:

\begin{sphinxVerbatim}[commandchars=\\\{\}]
\PYG{n}{templates\PYGZus{}path} \PYG{o}{=} \PYG{p}{[}\PYG{l+s+s1}{\PYGZsq{}}\PYG{l+s+s1}{\PYGZus{}templates}\PYG{l+s+s1}{\PYGZsq{}}\PYG{p}{]}
\end{sphinxVerbatim}

Finally, edit your override file \sphinxcode{\sphinxupquote{source/\_templates/layout.html}}:

\begin{sphinxVerbatim}[commandchars=\\\{\}]
\PYG{c}{\PYGZob{}\PYGZsh{} Import the theme\PYGZsq{}s layout. \PYGZsh{}\PYGZcb{}}
\PYG{c+cp}{\PYGZob{}\PYGZpc{}} \PYG{k}{extends} \PYG{l+s+s1}{\PYGZsq{}!layout.html\PYGZsq{}} \PYG{c+cp}{\PYGZpc{}\PYGZcb{}}

\PYG{c+cp}{\PYGZob{}\PYGZpc{}}\PYGZhy{} \PYG{k}{block} \PYG{n+nv}{extrahead} \PYG{c+cp}{\PYGZpc{}\PYGZcb{}}
\PYG{c}{\PYGZob{}\PYGZsh{} Add custom things to the head HTML tag \PYGZsh{}\PYGZcb{}}
\PYG{c}{\PYGZob{}\PYGZsh{} Call the parent block \PYGZsh{}\PYGZcb{}}
\PYG{c+cp}{\PYGZob{}\PYGZob{}} \PYG{n+nb}{super}\PYG{o}{(}\PYG{o}{)} \PYG{c+cp}{\PYGZcb{}\PYGZcb{}}
\PYG{c+cp}{\PYGZob{}\PYGZpc{}}\PYGZhy{} \PYG{k}{endblock} \PYG{c+cp}{\PYGZpc{}\PYGZcb{}}
\end{sphinxVerbatim}


\section{New Blocks}
\label{\detokenize{customization:new-blocks}}
The theme has a small number of new blocks to simplify some types of
customization:
\begin{description}
\item[{\sphinxcode{\sphinxupquote{footerrel}}}] \leavevmode
Previous and next in the footer.

\item[{\sphinxcode{\sphinxupquote{font}}}] \leavevmode
The default font inline CSS and the class to the google API. Use this
block when changing the font.

\item[{\sphinxcode{\sphinxupquote{fonticon}}}] \leavevmode
Block that contains the icon font. Use this to add additional icon fonts
(e.g., \sphinxhref{https://fontawesome.com/}{FontAwesome}%
\begin{footnote}[6]\sphinxAtStartFootnote
\sphinxnolinkurl{https://fontawesome.com/}
%
\end{footnote}). You should probably call \sphinxcode{\sphinxupquote{\{\{ super() \}\}}} at
the end of the block to include the default icon font as well.

\end{description}


\section{Version Dropdown}
\label{\detokenize{customization:version-dropdown}}
A version dropdown is available that lets you store multiple versions in a single site.
The standard structure of the site, relative to the base is usually:

\begin{sphinxVerbatim}[commandchars=\\\{\}]
\PYG{o}{/}
\PYG{o}{/}\PYG{n}{devel}
\PYG{o}{/}\PYG{n}{v1}\PYG{o}{.}\PYG{l+m+mf}{0.0}
\PYG{o}{/}\PYG{n}{v1}\PYG{o}{.}\PYG{l+m+mf}{1.0}
\PYG{o}{/}\PYG{n}{v1}\PYG{o}{.}\PYG{l+m+mf}{1.1}
\PYG{o}{/}\PYG{n}{v1}\PYG{o}{.}\PYG{l+m+mf}{2.0}
\end{sphinxVerbatim}

To use the version dropdown, you must set \sphinxcode{\sphinxupquote{version\_dropdown}} to \sphinxcode{\sphinxupquote{True}} in
the sites configuration.

There are two approaches, one which stores the version information in a JavaScript file
and one which uses a dictionary in the configuration.


\subsection{Using a Javascript File}
\label{\detokenize{customization:using-a-javascript-file}}
The data used is read via javascript from a file. The basic structure of the file is a dictionary of the form {[}label, path{]}.

This dictionary tells the dropdown that the release version is in the root of the site, the
other versions are archived under their version number, and the development version is
located in /devel.

\begin{sphinxadmonition}{note}{Note:}
The advantage of this approach is that you can separate version information
from the rendered documentation.  This makes is easy to change the version
dropdown in \_older\_ versions of the documentation to reflect additional versions
that are released later. Changing the Javascript file changes the version dropdown
content in all versions.  This approach is used in
\sphinxhref{https://www.statsmodels.org/}{statsmodels}%
\begin{footnote}[7]\sphinxAtStartFootnote
\sphinxnolinkurl{https://www.statsmodels.org/}
%
\end{footnote}.
\end{sphinxadmonition}


\subsection{Using \sphinxstyleliteralintitle{\sphinxupquote{conf.py}}}
\label{\detokenize{customization:using-conf-py}}
\begin{sphinxadmonition}{warning}{Warning:}
This method has precedence over the JavaScript approach. If \sphinxcode{\sphinxupquote{version\_info}} is
not empty in a site’s \sphinxcode{\sphinxupquote{html\_theme\_options}}, then the static approach is used.
\end{sphinxadmonition}

The alternative uses a dictionary where the key is the title and the value is the target.
The dictionary is part of the size configuration’s \sphinxcode{\sphinxupquote{html\_theme\_options}}.

The dictionary structure is nearly identical.  Here you can use relative paths
like in the JavaScript version. You can also use absolute paths.

\begin{sphinxadmonition}{note}{Note:}
This approach is easier if you only want to have a fixed set of documentation,
e.g., stable and devel.
\end{sphinxadmonition}


\chapter{Specimen}
\label{\detokenize{specimen:specimen}}\label{\detokenize{specimen::doc}}

\section{Body copy}
\label{\detokenize{specimen:body-copy}}
Lorem ipsum dolor sit amet, consectetur adipiscing elit. Cras arcu
libero, mollis sed massa vel, \sphinxstyleemphasis{ornare viverra ex}. Mauris a ullamcorper
lacus. Nullam urna elit, malesuada eget finibus ut, ullamcorper ac
tortor. Vestibulum sodales pulvinar nisl, pharetra aliquet est. Quisque
volutpat erat ac nisi accumsan tempor.

\sphinxstylestrong{Sed suscipit}, orci non pretium pretium, quam mi gravida metus, vel
venenatis justo est condimentum diam. Maecenas non ornare justo. Nam a
ipsum eros. {\hyperref[\detokenize{specimen:}]{\emph{Nulla aliquam}}} orci sit amet nisl posuere malesuada.
Proin aliquet nulla velit, quis ultricies orci feugiat et.
\sphinxcode{\sphinxupquote{Ut tincidunt sollicitudin}} tincidunt. Aenean ullamcorper sit amet
nulla at interdum.


\section{Headings}
\label{\detokenize{specimen:headings}}

\subsection{The 3rd level}
\label{\detokenize{specimen:the-3rd-level}}

\subsubsection{The 4th level}
\label{\detokenize{specimen:the-4th-level}}

\paragraph{The 5th level}
\label{\detokenize{specimen:the-5th-level}}
The 6th level


\section{Headings with secondary text}
\label{\detokenize{specimen:headings-with-secondary-text}}

\subsection{The 3rd level with secondary text}
\label{\detokenize{specimen:the-3rd-level-with-secondary-text}}

\subsubsection{The 4th level with secondary text}
\label{\detokenize{specimen:the-4th-level-with-secondary-text}}

\paragraph{The 5th level with secondary text}
\label{\detokenize{specimen:the-5th-level-with-secondary-text}}
The 6th level with secondary text


\section{Blockquotes}
\label{\detokenize{specimen:blockquotes}}\begin{quote}

Morbi eget dapibus felis. Vivamus venenatis porttitor tortor sit amet
rutrum. Pellentesque aliquet quam enim, eu volutpat urna rutrum a.
Nam vehicula nunc mauris, a ultricies libero efficitur sed. \sphinxstyleemphasis{Class
aptent} taciti sociosqu ad litora torquent per conubia nostra, per
inceptos himenaeos. Sed molestie imperdiet consectetur.
\end{quote}


\subsection{Blockquote nesting}
\label{\detokenize{specimen:blockquote-nesting}}\begin{quote}

\sphinxstylestrong{Sed aliquet}, neque at rutrum mollis, neque nisi tincidunt nibh,
vitae faucibus lacus nunc at lacus. Nunc scelerisque, quam id cursus
sodales, lorem {\hyperref[\detokenize{specimen:}]{\emph{libero fermentum}}} urna, ut efficitur elit
ligula et nunc.
\end{quote}
\begin{quote}

Mauris dictum mi lacus, sit amet pellentesque urna vehicula
fringilla. Ut sit amet placerat ante. Proin sed elementum nulla.
Nunc vitae sem odio. Suspendisse ac eros arcu. Vivamus orci erat,
volutpat a tempor et, rutrum. eu odio.
\begin{quote}

\sphinxcode{\sphinxupquote{Suspendisse rutrum facilisis risus}}, eu posuere neque
commodo a. Interdum et malesuada fames ac ante ipsum primis in
faucibus. Sed nec leo bibendum, sodales mauris ut, tincidunt
massa.
\end{quote}
\end{quote}


\subsection{Other content blocks}
\label{\detokenize{specimen:other-content-blocks}}\begin{quote}

Vestibulum vitae orci quis ante viverra ultricies ut eget turpis. Sed
eu lectus dapibus, eleifend nulla varius, lobortis turpis. In ac
hendrerit nisl, sit amet laoreet nibh.
\begin{quote}

Praesent at \sphinxcode{\sphinxupquote{\DUrole{keyword}{return} \DUrole{name,other}{target}}}, sodales nibh vel, tempor
felis. Fusce vel lacinia lacus. Suspendisse rhoncus nunc non nisi
iaculis ultrices. Donec consectetur mauris non neque imperdiet,
eget volutpat libero.
\end{quote}

\fvset{hllines={, 8,}}%
\begin{sphinxVerbatim}[commandchars=\\\{\}]
\PYG{k+kd}{var} \PYG{n+nx}{\PYGZus{}extends} \PYG{o}{=} \PYG{k+kd}{function}\PYG{p}{(}\PYG{n+nx}{target}\PYG{p}{)} \PYG{p}{\PYGZob{}}
  \PYG{k}{for} \PYG{p}{(}\PYG{k+kd}{var} \PYG{n+nx}{i} \PYG{o}{=} \PYG{l+m+mf}{1}\PYG{p}{;} \PYG{n+nx}{i} \PYG{o}{\PYGZlt{}} \PYG{n+nx}{arguments}\PYG{p}{.}\PYG{n+nx}{length}\PYG{p}{;} \PYG{n+nx}{i}\PYG{o}{++}\PYG{p}{)} \PYG{p}{\PYGZob{}}
    \PYG{k+kd}{var} \PYG{n+nx}{source} \PYG{o}{=} \PYG{n+nx}{arguments}\PYG{p}{[}\PYG{n+nx}{i}\PYG{p}{]}\PYG{p}{;}
    \PYG{k}{for} \PYG{p}{(}\PYG{k+kd}{var} \PYG{n+nx}{key} \PYG{k}{in} \PYG{n+nx}{source}\PYG{p}{)} \PYG{p}{\PYGZob{}}
      \PYG{n+nx}{target}\PYG{p}{[}\PYG{n+nx}{key}\PYG{p}{]} \PYG{o}{=} \PYG{n+nx}{source}\PYG{p}{[}\PYG{n+nx}{key}\PYG{p}{]}\PYG{p}{;}
    \PYG{p}{\PYGZcb{}}
  \PYG{p}{\PYGZcb{}}
  \PYG{k}{return} \PYG{n+nx}{target}\PYG{p}{;}
\PYG{p}{\PYGZcb{}}\PYG{p}{;}
\end{sphinxVerbatim}
\sphinxresetverbatimhllines
\end{quote}


\section{Lists}
\label{\detokenize{specimen:lists}}

\subsection{Unordered lists}
\label{\detokenize{specimen:unordered-lists}}\begin{itemize}
\item {} 
Sed sagittis eleifend rutrum. Donec vitae suscipit est. Nullam tempus
tellus non sem sollicitudin, quis rutrum leo facilisis. Nulla tempor
lobortis orci, at elementum urna sodales vitae. In in vehicula nulla,
quis ornare libero.
\begin{itemize}
\item {} 
Duis mollis est eget nibh volutpat, fermentum aliquet dui mollis.

\item {} 
Nam vulputate tincidunt fringilla.

\item {} 
Nullam dignissim ultrices urna non auctor.

\end{itemize}

\item {} 
Aliquam metus eros, pretium sed nulla venenatis, faucibus auctor ex.
Proin ut eros sed sapien ullamcorper consequat. Nunc ligula ante,
fringilla at aliquam ac, aliquet sed mauris.

\item {} 
Nulla et rhoncus turpis. Mauris ultricies elementum leo. Duis
efficitur accumsan nibh eu mattis. Vivamus tempus velit eros,
porttitor placerat nibh lacinia sed. Aenean in finibus diam.

\end{itemize}


\subsection{Ordered lists}
\label{\detokenize{specimen:ordered-lists}}\begin{enumerate}
\sphinxsetlistlabels{\arabic}{enumi}{enumii}{}{.}%
\item {} 
Integer vehicula feugiat magna, a mollis tellus. Nam mollis ex ante,
quis elementum eros tempor rutrum. Aenean efficitur lobortis lacinia.
Nulla consectetur feugiat sodales.

\item {} 
Cum sociis natoque penatibus et magnis dis parturient montes,
nascetur ridiculus mus. Aliquam ornare feugiat quam et egestas. Nunc
id erat et quam pellentesque lacinia eu vel odio.
\begin{enumerate}
\sphinxsetlistlabels{\arabic}{enumii}{enumiii}{}{.}%
\item {} 
Vivamus venenatis porttitor tortor sit amet rutrum. Pellentesque
aliquet quam enim, eu volutpat urna rutrum a. Nam vehicula nunc
mauris, a ultricies libero efficitur sed.
\begin{enumerate}
\sphinxsetlistlabels{\arabic}{enumiii}{enumiv}{}{.}%
\item {} 
Mauris dictum mi lacus

\item {} 
Ut sit amet placerat ante

\item {} 
Suspendisse ac eros arcu

\end{enumerate}

\item {} 
Morbi eget dapibus felis. Vivamus venenatis porttitor tortor sit
amet rutrum. Pellentesque aliquet quam enim, eu volutpat urna
rutrum a. Sed aliquet, neque at rutrum mollis, neque nisi
tincidunt nibh.

\item {} 
Pellentesque eget \sphinxcode{\sphinxupquote{\DUrole{keyword,declaration}{var} \DUrole{name,other}{\_extends}}} ornare tellus, ut gravida
mi.

\fvset{hllines={, 1,}}%
\begin{sphinxVerbatim}[commandchars=\\\{\}]
\PYG{k+kd}{var} \PYG{n+nx}{\PYGZus{}extends} \PYG{o}{=} \PYG{k+kd}{function}\PYG{p}{(}\PYG{n+nx}{target}\PYG{p}{)} \PYG{p}{\PYGZob{}}
  \PYG{k}{for} \PYG{p}{(}\PYG{k+kd}{var} \PYG{n+nx}{i} \PYG{o}{=} \PYG{l+m+mf}{1}\PYG{p}{;} \PYG{n+nx}{i} \PYG{o}{\PYGZlt{}} \PYG{n+nx}{arguments}\PYG{p}{.}\PYG{n+nx}{length}\PYG{p}{;} \PYG{n+nx}{i}\PYG{o}{++}\PYG{p}{)} \PYG{p}{\PYGZob{}}
    \PYG{k+kd}{var} \PYG{n+nx}{source} \PYG{o}{=} \PYG{n+nx}{arguments}\PYG{p}{[}\PYG{n+nx}{i}\PYG{p}{]}\PYG{p}{;}
    \PYG{k}{for} \PYG{p}{(}\PYG{k+kd}{var} \PYG{n+nx}{key} \PYG{k}{in} \PYG{n+nx}{source}\PYG{p}{)} \PYG{p}{\PYGZob{}}
      \PYG{n+nx}{target}\PYG{p}{[}\PYG{n+nx}{key}\PYG{p}{]} \PYG{o}{=} \PYG{n+nx}{source}\PYG{p}{[}\PYG{n+nx}{key}\PYG{p}{]}\PYG{p}{;}
    \PYG{p}{\PYGZcb{}}
  \PYG{p}{\PYGZcb{}}
  \PYG{k}{return} \PYG{n+nx}{target}\PYG{p}{;}
\PYG{p}{\PYGZcb{}}\PYG{p}{;}
\end{sphinxVerbatim}
\sphinxresetverbatimhllines

\end{enumerate}

\item {} 
Vivamus id mi enim. Integer id turpis sapien. Ut condimentum lobortis
sagittis. Aliquam purus tellus, faucibus eget urna at, iaculis
venenatis nulla. Vivamus a pharetra leo.

\end{enumerate}


\subsection{Definition lists}
\label{\detokenize{specimen:definition-lists}}\begin{description}
\item[{Lorem ipsum dolor sit amet}] \leavevmode
Sed sagittis eleifend rutrum. Donec vitae suscipit est. Nullam tempus
tellus non sem sollicitudin, quis rutrum leo facilisis. Nulla tempor
lobortis orci, at elementum urna sodales vitae. In in vehicula nulla.

Duis mollis est eget nibh volutpat, fermentum aliquet dui mollis. Nam
vulputate tincidunt fringilla. Nullam dignissim ultrices urna non
auctor.

\item[{Cras arcu libero}] \leavevmode
Aliquam metus eros, pretium sed nulla venenatis, faucibus auctor ex.
Proin ut eros sed sapien ullamcorper consequat. Nunc ligula ante,
fringilla at aliquam ac, aliquet sed mauris.

\end{description}


\section{Code blocks}
\label{\detokenize{specimen:code-blocks}}

\subsection{Inline}
\label{\detokenize{specimen:inline}}
Morbi eget \sphinxcode{\sphinxupquote{dapibus felis}}. Vivamus \sphinxstyleemphasis{``venenatis porttitor``} tortor
sit amet rutrum. Class aptent taciti sociosqu ad litora torquent per
conubia nostra, per inceptos himenaeos. {\hyperref[\detokenize{specimen:}]{\emph{\sphinxcode{\sphinxupquote{Pellentesque aliquet quam enim}}}}},
eu volutpat urna rutrum a.

Nam vehicula nunc \sphinxcode{\sphinxupquote{:::js return target}} mauris, a ultricies libero
efficitur sed. Sed molestie imperdiet consectetur. Vivamus a pharetra
leo. Pellentesque eget ornare tellus, ut gravida mi. Fusce vel lacinia
lacus.


\subsection{Listing}
\label{\detokenize{specimen:listing}}
\fvset{hllines={, 1, 5, 8,}}%
\begin{sphinxVerbatim}[commandchars=\\\{\}]
\PYG{k+kd}{var} \PYG{n+nx}{\PYGZus{}extends} \PYG{o}{=} \PYG{k+kd}{function}\PYG{p}{(}\PYG{n+nx}{target}\PYG{p}{)} \PYG{p}{\PYGZob{}}
  \PYG{k}{for} \PYG{p}{(}\PYG{k+kd}{var} \PYG{n+nx}{i} \PYG{o}{=} \PYG{l+m+mf}{1}\PYG{p}{;} \PYG{n+nx}{i} \PYG{o}{\PYGZlt{}} \PYG{n+nx}{arguments}\PYG{p}{.}\PYG{n+nx}{length}\PYG{p}{;} \PYG{n+nx}{i}\PYG{o}{++}\PYG{p}{)} \PYG{p}{\PYGZob{}}
    \PYG{k+kd}{var} \PYG{n+nx}{source} \PYG{o}{=} \PYG{n+nx}{arguments}\PYG{p}{[}\PYG{n+nx}{i}\PYG{p}{]}\PYG{p}{;}
    \PYG{k}{for} \PYG{p}{(}\PYG{k+kd}{var} \PYG{n+nx}{key} \PYG{k}{in} \PYG{n+nx}{source}\PYG{p}{)} \PYG{p}{\PYGZob{}}
      \PYG{n+nx}{target}\PYG{p}{[}\PYG{n+nx}{key}\PYG{p}{]} \PYG{o}{=} \PYG{n+nx}{source}\PYG{p}{[}\PYG{n+nx}{key}\PYG{p}{]}\PYG{p}{;}
    \PYG{p}{\PYGZcb{}}
  \PYG{p}{\PYGZcb{}}
  \PYG{k}{return} \PYG{n+nx}{target}\PYG{p}{;}
\PYG{p}{\PYGZcb{}}\PYG{p}{;}
\end{sphinxVerbatim}
\sphinxresetverbatimhllines


\section{Horizontal rules}
\label{\detokenize{specimen:horizontal-rules}}
Aenean in finibus diam. Duis mollis est eget nibh volutpat, fermentum
aliquet dui mollis. Nam vulputate tincidunt fringilla. Nullam dignissim
ultrices urna non auctor.


\bigskip\hrule\bigskip


Integer vehicula feugiat magna, a mollis tellus. Nam mollis ex ante,
quis elementum eros tempor rutrum. Aenean efficitur lobortis lacinia.
Nulla consectetur feugiat sodales.


\section{Data tables}
\label{\detokenize{specimen:data-tables}}

\begin{savenotes}\sphinxattablestart
\centering
\begin{tabular}[t]{|*{6}{\X{1}{6}|}}
\hline
\sphinxstyletheadfamily 
Sollicitudo / Pellentesi
&\sphinxstyletheadfamily 
consectetur
&\sphinxstyletheadfamily 
adipiscing
&\sphinxstyletheadfamily 
elit
&\sphinxstyletheadfamily 
arcu
&\sphinxstyletheadfamily 
sed
\\
\hline
Vivamus a pharetra
&
yes
&
yes
&
yes
&
yes
&
yes
\\
\hline
Ornare viverra ex
&
yes
&
yes
&
yes
&
yes
&
yes
\\
\hline
Mauris a ullamcorper
&
yes
&
yes
&
partial
&
yes
&
yes
\\
\hline
Nullam urna elit
&
yes
&
yes
&
yes
&
yes
&
yes
\\
\hline
Malesuada eget finibus
&
yes
&
yes
&
yes
&
yes
&
yes
\\
\hline
Ullamcorper
&
yes
&
yes
&
yes
&
yes
&
yes
\\
\hline
Vestibulum sodales
&
yes
&\begin{itemize}
\item {} 
\end{itemize}
&
yes
&\begin{itemize}
\item {} 
\end{itemize}
&
yes
\\
\hline
Pulvinar nisl
&
yes
&
yes
&
yes
&\begin{itemize}
\item {} 
\end{itemize}
&\begin{itemize}
\item {} 
\end{itemize}
\\
\hline
Pharetra aliquet est
&
yes
&
yes
&
yes
&
yes
&
yes
\\
\hline
Sed suscipit
&
yes
&
yes
&
yes
&
yes
&
yes
\\
\hline
Orci non pretium
&
yes
&
partial
&\begin{itemize}
\item {} 
\end{itemize}
&\begin{itemize}
\item {} 
\end{itemize}
&\begin{itemize}
\item {} 
\end{itemize}
\\
\hline
\end{tabular}
\par
\sphinxattableend\end{savenotes}

Sed sagittis eleifend rutrum. Donec vitae suscipit est. Nullam tempus
tellus non sem sollicitudin, quis rutrum leo facilisis. Nulla tempor
lobortis orci, at elementum urna sodales vitae. In in vehicula nulla,
quis ornare libero.


\begin{savenotes}\sphinxattablestart
\centering
\begin{tabulary}{\linewidth}[t]{|T|T|T|}
\hline
\sphinxstyletheadfamily 
Left
&\sphinxstyletheadfamily 
Center
&\sphinxstyletheadfamily 
Right
\\
\hline
Lorem
&
\sphinxstyleemphasis{dolor}
&
\sphinxcode{\sphinxupquote{amet}}
\\
\hline
{\hyperref[\detokenize{specimen:}]{\emph{ipsum}}}
&
\sphinxstylestrong{sit}
&\\
\hline
\end{tabulary}
\par
\sphinxattableend\end{savenotes}

Vestibulum vitae orci quis ante viverra ultricies ut eget turpis. Sed eu
lectus dapibus, eleifend nulla varius, lobortis turpis. In ac hendrerit
nisl, sit amet laoreet nibh.


\begin{savenotes}\sphinxattablestart
\centering
\begin{tabular}[t]{|\X{30}{100}|\X{70}{100}|}
\hline
\sphinxstyletheadfamily 
Table
&\sphinxstyletheadfamily 
with colgroups (Pandoc)
\\
\hline
Lorem
&
ipsum dolor sit amet.
\\
\hline
Sed sagittis
&
eleifend rutrum. Donec vitae suscipit est.
\\
\hline
\end{tabular}
\par
\sphinxattableend\end{savenotes}

Lorem ipsum dolor sit amet, consectetur adipiscing elit. Vivamus nec ipsum a
eros convallis facilisis eget at leo. Cras eu pulvinar eros, at accumsan dolor.
Ut gravida massa sed eros imperdiet fermentum. Donec ac diam ut lorem consequat
laoreet. Maecenas at ex diam. Phasellus tincidunt orci felis, nec commodo nisl
aliquet ac. Aenean eget ornare tellus. Nullam vel nunc quis nisi sodales
finibus in ut metus. Praesent ultrices mollis leo, auctor volutpat eros
consectetur in. Sed ac odio nisi. Cras aliquet ultrices nisl ac mattis. Nulla a
dui velit. Proin et ipsum quis metus auctor viverra. Proin suscipit massa quis
magna mattis, vel tincidunt quam tincidunt. Vestibulum nec feugiat metus, nec
scelerisque eros. Ut ultricies ornare aliquam.


\section{Section II}
\label{\detokenize{specimen:section-ii}}
Proin ac mi tempor, ullamcorper ante at, sodales augue. Duis enim turpis,
volutpat eget consectetur id, facilisis vel nisl. Sed non leo aliquam, tempus
nisl eu, vestibulum enim. Suspendisse et leo imperdiet, pulvinar lacus sed,
commodo felis.

\begin{sphinxadmonition}{note}{Note:}
Praesent elit mi, pretium nec pellentesque eget, ultricies
euismod turpis.
\end{sphinxadmonition}


\subsection{Sub section}
\label{\detokenize{specimen:sub-section}}
In lobortis elementum tempus. Nam facilisis orci neque, eget vestibulum lectus
imperdiet sed. Aenean ac eros sollicitudin, accumsan turpis ac, faucibus arcu.


\section{Section III}
\label{\detokenize{specimen:section-iii}}
Donec sodales, velit ac sagittis fermentum, metus ante pharetra ex, ac eleifend
erat ligula in lacus. Donec tincidunt urna est, non mollis turpis lacinia sit
amet. Duis ac facilisis libero, ut interdum nibh. Sed rutrum dapibus pharetra.
Ut ac luctus nisi, vitae volutpat arcu. Vivamus venenatis eu nibh ut
consectetur. Cras tincidunt dui nisi, et facilisis eros feugiat nec.

Fusce ante:
\begin{itemize}
\item {} 
libero

\item {} 
consequat quis facilisis id

\item {} 
sollicitudin et nisl.

\end{itemize}

Aliquam diam mi, vulputate nec posuere id, consequat id elit. Ut feugiat lectus
quam, sed aliquet augue placerat nec. Sed volutpat leo ac condimentum
ullamcorper. Vivamus eleifend magna tellus, sit amet porta nunc ultrices eget.
Nullam id laoreet ex. Nam ultricies, ante et molestie mollis, magna sem porta
libero, sed molestie neque risus ut purus. Ut tellus sapien, auctor a lacus eu,
iaculis congue ex.

Duis et nisi a odio \sphinxstylestrong{scelerisque} sodales ac ut sapien. Ut eleifend blandit
velit luctus euismod. Curabitur at pulvinar mi. Cras molestie lorem non accumsan
gravida. Sed vulputate, ligula ut tincidunt congue, metus risus luctus lacus,
sed rhoncus ligula turpis non erat. Phasellus est est, \sphinxstyleemphasis{sollicitudin ut}
elementum vel, placerat in orci. Proin molestie posuere dolor sit amet
convallis. Donec id urna vel lacus ultrices pulvinar sit amet id metus. Donec
in venenatis ante. Nam eu rhoncus leo. Quisque posuere, leo vel porttitor
malesuada, nisi urna dignissim justo, vel consectetur purus elit in mauris.
Vestibulum lectus arcu, varius ut ligula quis, viverra gravida sem.

\begin{sphinxadmonition}{warning}{Warning:}
Pellentesque in enim leo.
\end{sphinxadmonition}


\section{Images}
\label{\detokenize{specimen:images}}

\subsection{Default Alignment}
\label{\detokenize{specimen:default-alignment}}
\noindent\sphinxincludegraphics[width=512\sphinxpxdimen,height=366\sphinxpxdimen]{{desert-flower}.jpg}


\subsection{Center Alignment}
\label{\detokenize{specimen:center-alignment}}
\noindent{\hspace*{\fill}\sphinxincludegraphics[width=409.60000\sphinxpxdimen,height=292.80000\sphinxpxdimen]{{desert-flower}.jpg}\hspace*{\fill}}


\subsection{Right Alignment}
\label{\detokenize{specimen:right-alignment}}
\noindent{\hspace*{\fill}\sphinxincludegraphics[width=307.20000\sphinxpxdimen,height=219.60000\sphinxpxdimen]{{desert-flower}.jpg}}


\chapter{Jupyter Notebooks}
\label{\detokenize{notebook:Jupyter-Notebooks}}\label{\detokenize{notebook::doc}}
The \sphinxhref{https://github.com/spatialaudio/nbsphinx}{nbsphinx extension}%
\begin{footnote}[8]\sphinxAtStartFootnote
\sphinxnolinkurl{https://github.com/spatialaudio/nbsphinx}
%
\end{footnote} allow notebooks to be seemlessly integrated into a Sphinx website. This page demonstrates how notebooks are rendered.


\section{facebook}
\label{\detokenize{notebook:facebook}}

\begin{savenotes}\sphinxattablestart
\centering
\begin{tabulary}{\linewidth}[t]{|T|T|T|T|T|}
\hline
\sphinxstyletheadfamily 
asfasdfasdfasdf
&\sphinxstyletheadfamily 
asdfasdf
&\sphinxstyletheadfamily 
asdfasdf
&\sphinxstyletheadfamily 
asdfasdfasdf
&\sphinxstyletheadfamily 
asdfasdfasdfasdf
\\
\hline
1324
&
2
&
2
&
1
&
1
\\
\hline
1234
&
2
&
2
&
1
&
1
\\
\hline
1234
&
2
&
2
&
2
&
1
\\
\hline
\end{tabulary}
\par
\sphinxattableend\end{savenotes}


\begin{savenotes}\sphinxattablestart
\centering
\begin{tabulary}{\linewidth}[t]{|T|T|T|}
\hline
\sphinxstyletheadfamily 
Stretch/Untouched
&\sphinxstyletheadfamily 
ProbDistribution
&\sphinxstyletheadfamily 
Accuracy
\\
\hline
Stretched
&
Gaussian
&
.843
\\
\hline
\end{tabulary}
\par
\sphinxattableend\end{savenotes}

{
\sphinxsetup{VerbatimColor={named}{nbsphinx-code-bg}}
\sphinxsetup{VerbatimBorderColor={named}{nbsphinx-code-border}}
\begin{sphinxVerbatim}[commandchars=\\\{\}]
\llap{\color{nbsphinxin}[1]:\,\hspace{\fboxrule}\hspace{\fboxsep}}\PYG{k+kn}{from} \PYG{n+nn}{IPython}\PYG{n+nn}{.}\PYG{n+nn}{display} \PYG{k+kn}{import} \PYG{n}{HTML}\PYG{p}{,} \PYG{n}{display}
\PYG{k+kn}{import} \PYG{n+nn}{tabulate}
\PYG{n}{table} \PYG{o}{=} \PYG{p}{[}\PYG{p}{[}\PYG{l+s+s2}{\PYGZdq{}}\PYG{l+s+s2}{Sun}\PYG{l+s+s2}{\PYGZdq{}}\PYG{p}{,}\PYG{l+m+mi}{696000}\PYG{p}{,}\PYG{l+m+mi}{1989100000}\PYG{p}{]}\PYG{p}{,}
         \PYG{p}{[}\PYG{l+s+s2}{\PYGZdq{}}\PYG{l+s+s2}{Earth}\PYG{l+s+s2}{\PYGZdq{}}\PYG{p}{,}\PYG{l+m+mi}{6371}\PYG{p}{,}\PYG{l+m+mf}{5973.6}\PYG{p}{]}\PYG{p}{,}
         \PYG{p}{[}\PYG{l+s+s2}{\PYGZdq{}}\PYG{l+s+s2}{Moon}\PYG{l+s+s2}{\PYGZdq{}}\PYG{p}{,}\PYG{l+m+mi}{1737}\PYG{p}{,}\PYG{l+m+mf}{73.5}\PYG{p}{]}\PYG{p}{,}
         \PYG{p}{[}\PYG{l+s+s2}{\PYGZdq{}}\PYG{l+s+s2}{Mars}\PYG{l+s+s2}{\PYGZdq{}}\PYG{p}{,}\PYG{l+m+mi}{3390}\PYG{p}{,}\PYG{l+m+mf}{641.85}\PYG{p}{]}\PYG{p}{]}
\PYG{n}{display}\PYG{p}{(}\PYG{n}{HTML}\PYG{p}{(}\PYG{n}{tabulate}\PYG{o}{.}\PYG{n}{tabulate}\PYG{p}{(}\PYG{n}{table}\PYG{p}{,} \PYG{n}{tablefmt}\PYG{o}{=}\PYG{l+s+s1}{\PYGZsq{}}\PYG{l+s+s1}{html}\PYG{l+s+s1}{\PYGZsq{}}\PYG{p}{)}\PYG{p}{)}\PYG{p}{)}
\end{sphinxVerbatim}
}

{

\kern-\sphinxverbatimsmallskipamount\kern-\baselineskip
\kern+\FrameHeightAdjust\kern-\fboxrule
\vspace{\nbsphinxcodecellspacing}

\sphinxsetup{VerbatimColor={named}{white}}
\sphinxsetup{VerbatimBorderColor={named}{nbsphinx-code-border}}
\begin{sphinxVerbatim}[commandchars=\\\{\}]
<IPython.core.display.HTML object>
\end{sphinxVerbatim}
}


\section{DataFrames}
\label{\detokenize{notebook:DataFrames}}
pandas DataFrames are rendered with useful markup.

{
\sphinxsetup{VerbatimColor={named}{nbsphinx-code-bg}}
\sphinxsetup{VerbatimBorderColor={named}{nbsphinx-code-border}}
\begin{sphinxVerbatim}[commandchars=\\\{\}]
\llap{\color{nbsphinxin}[2]:\,\hspace{\fboxrule}\hspace{\fboxsep}}\PYG{k+kn}{import} \PYG{n+nn}{numpy} \PYG{k}{as} \PYG{n+nn}{np}
\PYG{k+kn}{import} \PYG{n+nn}{pandas} \PYG{k}{as} \PYG{n+nn}{pd}

\PYG{n}{df} \PYG{o}{=} \PYG{n}{pd}\PYG{o}{.}\PYG{n}{DataFrame}\PYG{p}{(}\PYG{p}{\PYGZob{}}\PYG{l+s+s1}{\PYGZsq{}}\PYG{l+s+s1}{ints}\PYG{l+s+s1}{\PYGZsq{}}\PYG{p}{:} \PYG{p}{[}\PYG{l+m+mi}{1}\PYG{p}{,} \PYG{l+m+mi}{2}\PYG{p}{,} \PYG{l+m+mi}{3}\PYG{p}{]}\PYG{p}{,}
                   \PYG{l+s+s1}{\PYGZsq{}}\PYG{l+s+s1}{floats}\PYG{l+s+s1}{\PYGZsq{}}\PYG{p}{:} \PYG{p}{[}\PYG{n}{np}\PYG{o}{.}\PYG{n}{pi}\PYG{p}{,} \PYG{n}{np}\PYG{o}{.}\PYG{n}{exp}\PYG{p}{(}\PYG{l+m+mi}{1}\PYG{p}{)}\PYG{p}{,} \PYG{p}{(}\PYG{l+m+mi}{1}\PYG{o}{+}\PYG{n}{np}\PYG{o}{.}\PYG{n}{sqrt}\PYG{p}{(}\PYG{l+m+mi}{5}\PYG{p}{)}\PYG{p}{)}\PYG{o}{/}\PYG{l+m+mi}{2}\PYG{p}{]}\PYG{p}{,}
                   \PYG{l+s+s1}{\PYGZsq{}}\PYG{l+s+s1}{strings}\PYG{l+s+s1}{\PYGZsq{}}\PYG{p}{:} \PYG{p}{[}\PYG{l+s+s1}{\PYGZsq{}}\PYG{l+s+s1}{aardvark}\PYG{l+s+s1}{\PYGZsq{}}\PYG{p}{,} \PYG{l+s+s1}{\PYGZsq{}}\PYG{l+s+s1}{bananarama}\PYG{l+s+s1}{\PYGZsq{}}\PYG{p}{,} \PYG{l+s+s1}{\PYGZsq{}}\PYG{l+s+s1}{charcuterie}\PYG{l+s+s1}{\PYGZsq{}} \PYG{p}{]}\PYG{p}{\PYGZcb{}}\PYG{p}{)}

\PYG{n}{df}
\end{sphinxVerbatim}
}

{

\kern-\sphinxverbatimsmallskipamount\kern-\baselineskip
\kern+\FrameHeightAdjust\kern-\fboxrule
\vspace{\nbsphinxcodecellspacing}

\sphinxsetup{VerbatimColor={named}{white}}
\sphinxsetup{VerbatimBorderColor={named}{nbsphinx-code-border}}
\begin{sphinxVerbatim}[commandchars=\\\{\}]
\llap{\color{nbsphinxout}[2]:\,\hspace{\fboxrule}\hspace{\fboxsep}}   ints    floats      strings
0     1  3.141593     aardvark
1     2  2.718282   bananarama
2     3  1.618034  charcuterie
\end{sphinxVerbatim}
}


\section{Plots and Figures}
\label{\detokenize{notebook:Plots-and-Figures}}
matplotlib can be used to produce plots in notebooks

This example comes from the \sphinxhref{https://matplotlib.org/3.1.1/gallery/ticks\_and\_spines/colorbar\_tick\_labelling\_demo.html\#sphx-glr-gallery-ticks-and-spines-colorbar-tick-labelling-demo-py}{matplotlib gallery}%
\begin{footnote}[9]\sphinxAtStartFootnote
\sphinxnolinkurl{https://matplotlib.org/3.1.1/gallery/ticks\_and\_spines/colorbar\_tick\_labelling\_demo.html\#sphx-glr-gallery-ticks-and-spines-colorbar-tick-labelling-demo-py}
%
\end{footnote}.

{
\sphinxsetup{VerbatimColor={named}{nbsphinx-code-bg}}
\sphinxsetup{VerbatimBorderColor={named}{nbsphinx-code-border}}
\begin{sphinxVerbatim}[commandchars=\\\{\}]
\llap{\color{nbsphinxin}[3]:\,\hspace{\fboxrule}\hspace{\fboxsep}}\PYG{o}{\PYGZpc{}}\PYG{k}{matplotlib} inline

\PYG{k+kn}{import} \PYG{n+nn}{numpy} \PYG{k}{as} \PYG{n+nn}{np}
\PYG{k+kn}{import} \PYG{n+nn}{matplotlib}\PYG{n+nn}{.}\PYG{n+nn}{pyplot} \PYG{k}{as} \PYG{n+nn}{plt}
\PYG{k+kn}{from} \PYG{n+nn}{matplotlib} \PYG{k+kn}{import} \PYG{n}{cm}

\PYG{n}{fig}\PYG{p}{,} \PYG{n}{ax} \PYG{o}{=} \PYG{n}{plt}\PYG{o}{.}\PYG{n}{subplots}\PYG{p}{(}\PYG{n}{figsize}\PYG{o}{=}\PYG{p}{(}\PYG{l+m+mi}{12}\PYG{p}{,}\PYG{l+m+mi}{8}\PYG{p}{)}\PYG{p}{)}

\PYG{n}{data} \PYG{o}{=} \PYG{n}{np}\PYG{o}{.}\PYG{n}{clip}\PYG{p}{(}\PYG{n}{np}\PYG{o}{.}\PYG{n}{random}\PYG{o}{.}\PYG{n}{randn}\PYG{p}{(}\PYG{l+m+mi}{250}\PYG{p}{,} \PYG{l+m+mi}{250}\PYG{p}{)}\PYG{p}{,} \PYG{o}{\PYGZhy{}}\PYG{l+m+mi}{1}\PYG{p}{,} \PYG{l+m+mi}{1}\PYG{p}{)}

\PYG{n}{cax} \PYG{o}{=} \PYG{n}{ax}\PYG{o}{.}\PYG{n}{imshow}\PYG{p}{(}\PYG{n}{data}\PYG{p}{,} \PYG{n}{interpolation}\PYG{o}{=}\PYG{l+s+s1}{\PYGZsq{}}\PYG{l+s+s1}{nearest}\PYG{l+s+s1}{\PYGZsq{}}\PYG{p}{,} \PYG{n}{cmap}\PYG{o}{=}\PYG{n}{cm}\PYG{o}{.}\PYG{n}{coolwarm}\PYG{p}{)}
\PYG{n}{ax}\PYG{o}{.}\PYG{n}{set\PYGZus{}title}\PYG{p}{(}\PYG{l+s+s1}{\PYGZsq{}}\PYG{l+s+s1}{Gaussian noise with vertical colorbar}\PYG{l+s+s1}{\PYGZsq{}}\PYG{p}{,} \PYG{n}{fontsize}\PYG{o}{=}\PYG{l+m+mi}{16}\PYG{p}{)}
\PYG{n}{plt}\PYG{o}{.}\PYG{n}{tick\PYGZus{}params}\PYG{p}{(}\PYG{n}{labelsize}\PYG{o}{=}\PYG{l+m+mi}{16}\PYG{p}{)}

\PYG{c+c1}{\PYGZsh{} Add colorbar, make sure to specify tick locations to match desired ticklabels}
\PYG{n}{cbar} \PYG{o}{=} \PYG{n}{fig}\PYG{o}{.}\PYG{n}{colorbar}\PYG{p}{(}\PYG{n}{cax}\PYG{p}{,} \PYG{n}{ticks}\PYG{o}{=}\PYG{p}{[}\PYG{o}{\PYGZhy{}}\PYG{l+m+mi}{1}\PYG{p}{,} \PYG{l+m+mi}{0}\PYG{p}{,} \PYG{l+m+mi}{1}\PYG{p}{]}\PYG{p}{)}
\PYG{n}{cbar}\PYG{o}{.}\PYG{n}{ax}\PYG{o}{.}\PYG{n}{set\PYGZus{}yticklabels}\PYG{p}{(}\PYG{p}{[}\PYG{l+s+s1}{\PYGZsq{}}\PYG{l+s+s1}{\PYGZlt{} \PYGZhy{}1}\PYG{l+s+s1}{\PYGZsq{}}\PYG{p}{,} \PYG{l+s+s1}{\PYGZsq{}}\PYG{l+s+s1}{0}\PYG{l+s+s1}{\PYGZsq{}}\PYG{p}{,} \PYG{l+s+s1}{\PYGZsq{}}\PYG{l+s+s1}{\PYGZgt{} 1}\PYG{l+s+s1}{\PYGZsq{}}\PYG{p}{]}\PYG{p}{)}  \PYG{c+c1}{\PYGZsh{} vertically oriented colorbar}
\PYG{n}{cbar}\PYG{o}{.}\PYG{n}{ax}\PYG{o}{.}\PYG{n}{tick\PYGZus{}params}\PYG{p}{(}\PYG{n}{labelsize}\PYG{o}{=}\PYG{l+m+mi}{16}\PYG{p}{)}
\end{sphinxVerbatim}
}

\hrule height -\fboxrule\relax
\vspace{\nbsphinxcodecellspacing}

\makeatletter\setbox\nbsphinxpromptbox\box\voidb@x\makeatother

\begin{nbsphinxfancyoutput}

\noindent\sphinxincludegraphics[width=583\sphinxpxdimen,height=489\sphinxpxdimen]{{notebook_7_0}.png}

\end{nbsphinxfancyoutput}


\chapter{NumPy Docstrings}
\label{\detokenize{numpydoc:numpy-docstrings}}\label{\detokenize{numpydoc::doc}}
This page shows how \sphinxcode{\sphinxupquote{autosummary}} works with \sphinxcode{\sphinxupquote{numpydoc}} and a
NumPy\sphinxhyphen{}style docstring.


\begin{savenotes}\sphinxatlongtablestart\begin{longtable}[c]{\X{1}{2}\X{1}{2}}
\hline

\endfirsthead

\multicolumn{2}{c}%
{\makebox[0pt]{\sphinxtablecontinued{\tablename\ \thetable{} \textendash{} continued from previous page}}}\\
\hline

\endhead

\hline
\multicolumn{2}{r}{\makebox[0pt][r]{\sphinxtablecontinued{continues on next page}}}\\
\endfoot

\endlastfoot

{\hyperref[\detokenize{generated/numpy.polynomial.Polynomial:numpy.polynomial.Polynomial}]{\sphinxcrossref{\sphinxcode{\sphinxupquote{Polynomial}}}}}(coef{[}, domain, window{]})
&
A power series class.
\\
\hline
\end{longtable}\sphinxatlongtableend\end{savenotes}


\section{numpy.polynomial.Polynomial}
\label{\detokenize{generated/numpy.polynomial.Polynomial:numpy-polynomial-polynomial}}\label{\detokenize{generated/numpy.polynomial.Polynomial::doc}}\index{Polynomial (class in numpy.polynomial)@\spxentry{Polynomial}\spxextra{class in numpy.polynomial}}

\begin{fulllineitems}
\phantomsection\label{\detokenize{generated/numpy.polynomial.Polynomial:numpy.polynomial.Polynomial}}\pysiglinewithargsret{\sphinxbfcode{\sphinxupquote{class }}\sphinxcode{\sphinxupquote{numpy.polynomial.}}\sphinxbfcode{\sphinxupquote{Polynomial}}}{\emph{\DUrole{n}{coef}}, \emph{\DUrole{n}{domain}\DUrole{o}{=}\DUrole{default_value}{None}}, \emph{\DUrole{n}{window}\DUrole{o}{=}\DUrole{default_value}{None}}}{}
A power series class.

The Polynomial class provides the standard Python numerical methods
‘+’, ‘\sphinxhyphen{}‘, ‘*’, ‘//’, ‘\%’, ‘divmod’, ‘**’, and ‘()’ as well as the
attributes and methods listed in the \sphinxtitleref{ABCPolyBase} documentation.
\begin{quote}\begin{description}
\item[{Parameters}] \leavevmode\begin{description}
\item[{\sphinxstylestrong{coef}}] \leavevmode{[}array\_like{]}
Polynomial coefficients in order of increasing degree, i.e.,
\sphinxcode{\sphinxupquote{(1, 2, 3)}} give \sphinxcode{\sphinxupquote{1 + 2*x + 3*x**2}}.

\item[{\sphinxstylestrong{domain}}] \leavevmode{[}(2,) array\_like, optional{]}
Domain to use. The interval \sphinxcode{\sphinxupquote{{[}domain{[}0{]}, domain{[}1{]}{]}}} is mapped
to the interval \sphinxcode{\sphinxupquote{{[}window{[}0{]}, window{[}1{]}{]}}} by shifting and scaling.
The default value is {[}\sphinxhyphen{}1, 1{]}.

\item[{\sphinxstylestrong{window}}] \leavevmode{[}(2,) array\_like, optional{]}
Window, see \sphinxtitleref{domain} for its use. The default value is {[}\sphinxhyphen{}1, 1{]}.

\DUrole{versionmodified,added}{New in version 1.6.0.}

\end{description}

\end{description}\end{quote}
\subsubsection*{Methods}


\begin{savenotes}\sphinxatlongtablestart\begin{longtable}[c]{\X{1}{2}\X{1}{2}}
\hline

\endfirsthead

\multicolumn{2}{c}%
{\makebox[0pt]{\sphinxtablecontinued{\tablename\ \thetable{} \textendash{} continued from previous page}}}\\
\hline

\endhead

\hline
\multicolumn{2}{r}{\makebox[0pt][r]{\sphinxtablecontinued{continues on next page}}}\\
\endfoot

\endlastfoot

\sphinxcode{\sphinxupquote{\_\_call\_\_}}(arg)
&
Call self as a function.
\\
\hline
{\hyperref[\detokenize{generated/generated/numpy.polynomial.Polynomial.basis:numpy.polynomial.Polynomial.basis}]{\sphinxcrossref{\sphinxcode{\sphinxupquote{basis}}}}}(deg{[}, domain, window{]})
&
Series basis polynomial of degree \sphinxtitleref{deg}.
\\
\hline
{\hyperref[\detokenize{generated/generated/numpy.polynomial.Polynomial.cast:numpy.polynomial.Polynomial.cast}]{\sphinxcrossref{\sphinxcode{\sphinxupquote{cast}}}}}(series{[}, domain, window{]})
&
Convert series to series of this class.
\\
\hline
{\hyperref[\detokenize{generated/generated/numpy.polynomial.Polynomial.convert:numpy.polynomial.Polynomial.convert}]{\sphinxcrossref{\sphinxcode{\sphinxupquote{convert}}}}}({[}domain, kind, window{]})
&
Convert series to a different kind and/or domain and/or window.
\\
\hline
{\hyperref[\detokenize{generated/generated/numpy.polynomial.Polynomial.copy:numpy.polynomial.Polynomial.copy}]{\sphinxcrossref{\sphinxcode{\sphinxupquote{copy}}}}}()
&
Return a copy.
\\
\hline
{\hyperref[\detokenize{generated/generated/numpy.polynomial.Polynomial.cutdeg:numpy.polynomial.Polynomial.cutdeg}]{\sphinxcrossref{\sphinxcode{\sphinxupquote{cutdeg}}}}}(deg)
&
Truncate series to the given degree.
\\
\hline
{\hyperref[\detokenize{generated/generated/numpy.polynomial.Polynomial.degree:numpy.polynomial.Polynomial.degree}]{\sphinxcrossref{\sphinxcode{\sphinxupquote{degree}}}}}()
&
The degree of the series.
\\
\hline
{\hyperref[\detokenize{generated/generated/numpy.polynomial.Polynomial.deriv:numpy.polynomial.Polynomial.deriv}]{\sphinxcrossref{\sphinxcode{\sphinxupquote{deriv}}}}}({[}m{]})
&
Differentiate.
\\
\hline
{\hyperref[\detokenize{generated/generated/numpy.polynomial.Polynomial.fit:numpy.polynomial.Polynomial.fit}]{\sphinxcrossref{\sphinxcode{\sphinxupquote{fit}}}}}(x, y, deg{[}, domain, rcond, full, w, window{]})
&
Least squares fit to data.
\\
\hline
{\hyperref[\detokenize{generated/generated/numpy.polynomial.Polynomial.fromroots:numpy.polynomial.Polynomial.fromroots}]{\sphinxcrossref{\sphinxcode{\sphinxupquote{fromroots}}}}}(roots{[}, domain, window{]})
&
Return series instance that has the specified roots.
\\
\hline
{\hyperref[\detokenize{generated/generated/numpy.polynomial.Polynomial.has_samecoef:numpy.polynomial.Polynomial.has_samecoef}]{\sphinxcrossref{\sphinxcode{\sphinxupquote{has\_samecoef}}}}}(other)
&
Check if coefficients match.
\\
\hline
{\hyperref[\detokenize{generated/generated/numpy.polynomial.Polynomial.has_samedomain:numpy.polynomial.Polynomial.has_samedomain}]{\sphinxcrossref{\sphinxcode{\sphinxupquote{has\_samedomain}}}}}(other)
&
Check if domains match.
\\
\hline
{\hyperref[\detokenize{generated/generated/numpy.polynomial.Polynomial.has_sametype:numpy.polynomial.Polynomial.has_sametype}]{\sphinxcrossref{\sphinxcode{\sphinxupquote{has\_sametype}}}}}(other)
&
Check if types match.
\\
\hline
{\hyperref[\detokenize{generated/generated/numpy.polynomial.Polynomial.has_samewindow:numpy.polynomial.Polynomial.has_samewindow}]{\sphinxcrossref{\sphinxcode{\sphinxupquote{has\_samewindow}}}}}(other)
&
Check if windows match.
\\
\hline
{\hyperref[\detokenize{generated/generated/numpy.polynomial.Polynomial.identity:numpy.polynomial.Polynomial.identity}]{\sphinxcrossref{\sphinxcode{\sphinxupquote{identity}}}}}({[}domain, window{]})
&
Identity function.
\\
\hline
{\hyperref[\detokenize{generated/generated/numpy.polynomial.Polynomial.integ:numpy.polynomial.Polynomial.integ}]{\sphinxcrossref{\sphinxcode{\sphinxupquote{integ}}}}}({[}m, k, lbnd{]})
&
Integrate.
\\
\hline
{\hyperref[\detokenize{generated/generated/numpy.polynomial.Polynomial.linspace:numpy.polynomial.Polynomial.linspace}]{\sphinxcrossref{\sphinxcode{\sphinxupquote{linspace}}}}}({[}n, domain{]})
&
Return x, y values at equally spaced points in domain.
\\
\hline
{\hyperref[\detokenize{generated/generated/numpy.polynomial.Polynomial.mapparms:numpy.polynomial.Polynomial.mapparms}]{\sphinxcrossref{\sphinxcode{\sphinxupquote{mapparms}}}}}()
&
Return the mapping parameters.
\\
\hline
{\hyperref[\detokenize{generated/generated/numpy.polynomial.Polynomial.roots:numpy.polynomial.Polynomial.roots}]{\sphinxcrossref{\sphinxcode{\sphinxupquote{roots}}}}}()
&
Return the roots of the series polynomial.
\\
\hline
{\hyperref[\detokenize{generated/generated/numpy.polynomial.Polynomial.trim:numpy.polynomial.Polynomial.trim}]{\sphinxcrossref{\sphinxcode{\sphinxupquote{trim}}}}}({[}tol{]})
&
Remove trailing coefficients
\\
\hline
{\hyperref[\detokenize{generated/generated/numpy.polynomial.Polynomial.truncate:numpy.polynomial.Polynomial.truncate}]{\sphinxcrossref{\sphinxcode{\sphinxupquote{truncate}}}}}(size)
&
Truncate series to length \sphinxtitleref{size}.
\\
\hline
\end{longtable}\sphinxatlongtableend\end{savenotes}
\subsubsection*{Methods}


\begin{savenotes}\sphinxatlongtablestart\begin{longtable}[c]{\X{1}{2}\X{1}{2}}
\hline

\endfirsthead

\multicolumn{2}{c}%
{\makebox[0pt]{\sphinxtablecontinued{\tablename\ \thetable{} \textendash{} continued from previous page}}}\\
\hline

\endhead

\hline
\multicolumn{2}{r}{\makebox[0pt][r]{\sphinxtablecontinued{continues on next page}}}\\
\endfoot

\endlastfoot

{\hyperref[\detokenize{generated/generated/numpy.polynomial.Polynomial.basis:numpy.polynomial.Polynomial.basis}]{\sphinxcrossref{\sphinxcode{\sphinxupquote{basis}}}}}(deg{[}, domain, window{]})
&
Series basis polynomial of degree \sphinxtitleref{deg}.
\\
\hline
{\hyperref[\detokenize{generated/generated/numpy.polynomial.Polynomial.cast:numpy.polynomial.Polynomial.cast}]{\sphinxcrossref{\sphinxcode{\sphinxupquote{cast}}}}}(series{[}, domain, window{]})
&
Convert series to series of this class.
\\
\hline
{\hyperref[\detokenize{generated/generated/numpy.polynomial.Polynomial.convert:numpy.polynomial.Polynomial.convert}]{\sphinxcrossref{\sphinxcode{\sphinxupquote{convert}}}}}({[}domain, kind, window{]})
&
Convert series to a different kind and/or domain and/or window.
\\
\hline
{\hyperref[\detokenize{generated/generated/numpy.polynomial.Polynomial.copy:numpy.polynomial.Polynomial.copy}]{\sphinxcrossref{\sphinxcode{\sphinxupquote{copy}}}}}()
&
Return a copy.
\\
\hline
{\hyperref[\detokenize{generated/generated/numpy.polynomial.Polynomial.cutdeg:numpy.polynomial.Polynomial.cutdeg}]{\sphinxcrossref{\sphinxcode{\sphinxupquote{cutdeg}}}}}(deg)
&
Truncate series to the given degree.
\\
\hline
{\hyperref[\detokenize{generated/generated/numpy.polynomial.Polynomial.degree:numpy.polynomial.Polynomial.degree}]{\sphinxcrossref{\sphinxcode{\sphinxupquote{degree}}}}}()
&
The degree of the series.
\\
\hline
{\hyperref[\detokenize{generated/generated/numpy.polynomial.Polynomial.deriv:numpy.polynomial.Polynomial.deriv}]{\sphinxcrossref{\sphinxcode{\sphinxupquote{deriv}}}}}({[}m{]})
&
Differentiate.
\\
\hline
{\hyperref[\detokenize{generated/generated/numpy.polynomial.Polynomial.fit:numpy.polynomial.Polynomial.fit}]{\sphinxcrossref{\sphinxcode{\sphinxupquote{fit}}}}}(x, y, deg{[}, domain, rcond, full, w, window{]})
&
Least squares fit to data.
\\
\hline
{\hyperref[\detokenize{generated/generated/numpy.polynomial.Polynomial.fromroots:numpy.polynomial.Polynomial.fromroots}]{\sphinxcrossref{\sphinxcode{\sphinxupquote{fromroots}}}}}(roots{[}, domain, window{]})
&
Return series instance that has the specified roots.
\\
\hline
{\hyperref[\detokenize{generated/generated/numpy.polynomial.Polynomial.has_samecoef:numpy.polynomial.Polynomial.has_samecoef}]{\sphinxcrossref{\sphinxcode{\sphinxupquote{has\_samecoef}}}}}(other)
&
Check if coefficients match.
\\
\hline
{\hyperref[\detokenize{generated/generated/numpy.polynomial.Polynomial.has_samedomain:numpy.polynomial.Polynomial.has_samedomain}]{\sphinxcrossref{\sphinxcode{\sphinxupquote{has\_samedomain}}}}}(other)
&
Check if domains match.
\\
\hline
{\hyperref[\detokenize{generated/generated/numpy.polynomial.Polynomial.has_sametype:numpy.polynomial.Polynomial.has_sametype}]{\sphinxcrossref{\sphinxcode{\sphinxupquote{has\_sametype}}}}}(other)
&
Check if types match.
\\
\hline
{\hyperref[\detokenize{generated/generated/numpy.polynomial.Polynomial.has_samewindow:numpy.polynomial.Polynomial.has_samewindow}]{\sphinxcrossref{\sphinxcode{\sphinxupquote{has\_samewindow}}}}}(other)
&
Check if windows match.
\\
\hline
{\hyperref[\detokenize{generated/generated/numpy.polynomial.Polynomial.identity:numpy.polynomial.Polynomial.identity}]{\sphinxcrossref{\sphinxcode{\sphinxupquote{identity}}}}}({[}domain, window{]})
&
Identity function.
\\
\hline
{\hyperref[\detokenize{generated/generated/numpy.polynomial.Polynomial.integ:numpy.polynomial.Polynomial.integ}]{\sphinxcrossref{\sphinxcode{\sphinxupquote{integ}}}}}({[}m, k, lbnd{]})
&
Integrate.
\\
\hline
{\hyperref[\detokenize{generated/generated/numpy.polynomial.Polynomial.linspace:numpy.polynomial.Polynomial.linspace}]{\sphinxcrossref{\sphinxcode{\sphinxupquote{linspace}}}}}({[}n, domain{]})
&
Return x, y values at equally spaced points in domain.
\\
\hline
{\hyperref[\detokenize{generated/generated/numpy.polynomial.Polynomial.mapparms:numpy.polynomial.Polynomial.mapparms}]{\sphinxcrossref{\sphinxcode{\sphinxupquote{mapparms}}}}}()
&
Return the mapping parameters.
\\
\hline
{\hyperref[\detokenize{generated/generated/numpy.polynomial.Polynomial.roots:numpy.polynomial.Polynomial.roots}]{\sphinxcrossref{\sphinxcode{\sphinxupquote{roots}}}}}()
&
Return the roots of the series polynomial.
\\
\hline
{\hyperref[\detokenize{generated/generated/numpy.polynomial.Polynomial.trim:numpy.polynomial.Polynomial.trim}]{\sphinxcrossref{\sphinxcode{\sphinxupquote{trim}}}}}({[}tol{]})
&
Remove trailing coefficients
\\
\hline
{\hyperref[\detokenize{generated/generated/numpy.polynomial.Polynomial.truncate:numpy.polynomial.Polynomial.truncate}]{\sphinxcrossref{\sphinxcode{\sphinxupquote{truncate}}}}}(size)
&
Truncate series to length \sphinxtitleref{size}.
\\
\hline
\end{longtable}\sphinxatlongtableend\end{savenotes}


\subsection{numpy.polynomial.Polynomial.basis}
\label{\detokenize{generated/generated/numpy.polynomial.Polynomial.basis:numpy-polynomial-polynomial-basis}}\label{\detokenize{generated/generated/numpy.polynomial.Polynomial.basis::doc}}
method
\index{basis() (numpy.polynomial.Polynomial class method)@\spxentry{basis()}\spxextra{numpy.polynomial.Polynomial class method}}

\begin{fulllineitems}
\phantomsection\label{\detokenize{generated/generated/numpy.polynomial.Polynomial.basis:numpy.polynomial.Polynomial.basis}}\pysiglinewithargsret{\sphinxbfcode{\sphinxupquote{classmethod }}\sphinxcode{\sphinxupquote{Polynomial.}}\sphinxbfcode{\sphinxupquote{basis}}}{\emph{\DUrole{n}{deg}}, \emph{\DUrole{n}{domain}\DUrole{o}{=}\DUrole{default_value}{None}}, \emph{\DUrole{n}{window}\DUrole{o}{=}\DUrole{default_value}{None}}}{}
Series basis polynomial of degree \sphinxtitleref{deg}.

Returns the series representing the basis polynomial of degree \sphinxtitleref{deg}.

\DUrole{versionmodified,added}{New in version 1.7.0.}
\begin{quote}\begin{description}
\item[{Parameters}] \leavevmode\begin{description}
\item[{\sphinxstylestrong{deg}}] \leavevmode{[}int{]}
Degree of the basis polynomial for the series. Must be >= 0.

\item[{\sphinxstylestrong{domain}}] \leavevmode{[}\{None, array\_like\}, optional{]}
If given, the array must be of the form \sphinxcode{\sphinxupquote{{[}beg, end{]}}}, where
\sphinxcode{\sphinxupquote{beg}} and \sphinxcode{\sphinxupquote{end}} are the endpoints of the domain. If None is
given then the class domain is used. The default is None.

\item[{\sphinxstylestrong{window}}] \leavevmode{[}\{None, array\_like\}, optional{]}
If given, the resulting array must be if the form
\sphinxcode{\sphinxupquote{{[}beg, end{]}}}, where \sphinxcode{\sphinxupquote{beg}} and \sphinxcode{\sphinxupquote{end}} are the endpoints of
the window. If None is given then the class window is used. The
default is None.

\end{description}

\item[{Returns}] \leavevmode\begin{description}
\item[{\sphinxstylestrong{new\_series}}] \leavevmode{[}series{]}
A series with the coefficient of the \sphinxtitleref{deg} term set to one and
all others zero.

\end{description}

\end{description}\end{quote}

\end{fulllineitems}



\subsection{numpy.polynomial.Polynomial.cast}
\label{\detokenize{generated/generated/numpy.polynomial.Polynomial.cast:numpy-polynomial-polynomial-cast}}\label{\detokenize{generated/generated/numpy.polynomial.Polynomial.cast::doc}}
method
\index{cast() (numpy.polynomial.Polynomial class method)@\spxentry{cast()}\spxextra{numpy.polynomial.Polynomial class method}}

\begin{fulllineitems}
\phantomsection\label{\detokenize{generated/generated/numpy.polynomial.Polynomial.cast:numpy.polynomial.Polynomial.cast}}\pysiglinewithargsret{\sphinxbfcode{\sphinxupquote{classmethod }}\sphinxcode{\sphinxupquote{Polynomial.}}\sphinxbfcode{\sphinxupquote{cast}}}{\emph{\DUrole{n}{series}}, \emph{\DUrole{n}{domain}\DUrole{o}{=}\DUrole{default_value}{None}}, \emph{\DUrole{n}{window}\DUrole{o}{=}\DUrole{default_value}{None}}}{}
Convert series to series of this class.

The \sphinxtitleref{series} is expected to be an instance of some polynomial
series of one of the types supported by by the numpy.polynomial
module, but could be some other class that supports the convert
method.

\DUrole{versionmodified,added}{New in version 1.7.0.}
\begin{quote}\begin{description}
\item[{Parameters}] \leavevmode\begin{description}
\item[{\sphinxstylestrong{series}}] \leavevmode{[}series{]}
The series instance to be converted.

\item[{\sphinxstylestrong{domain}}] \leavevmode{[}\{None, array\_like\}, optional{]}
If given, the array must be of the form \sphinxcode{\sphinxupquote{{[}beg, end{]}}}, where
\sphinxcode{\sphinxupquote{beg}} and \sphinxcode{\sphinxupquote{end}} are the endpoints of the domain. If None is
given then the class domain is used. The default is None.

\item[{\sphinxstylestrong{window}}] \leavevmode{[}\{None, array\_like\}, optional{]}
If given, the resulting array must be if the form
\sphinxcode{\sphinxupquote{{[}beg, end{]}}}, where \sphinxcode{\sphinxupquote{beg}} and \sphinxcode{\sphinxupquote{end}} are the endpoints of
the window. If None is given then the class window is used. The
default is None.

\end{description}

\item[{Returns}] \leavevmode\begin{description}
\item[{\sphinxstylestrong{new\_series}}] \leavevmode{[}series{]}
A series of the same kind as the calling class and equal to
\sphinxtitleref{series} when evaluated.

\end{description}

\end{description}\end{quote}


\sphinxstrong{See also:}
\nopagebreak

\begin{description}
\item[{{\hyperref[\detokenize{generated/generated/numpy.polynomial.Polynomial.convert:numpy.polynomial.Polynomial.convert}]{\sphinxcrossref{\sphinxcode{\sphinxupquote{convert}}}}}}] \leavevmode
similar instance method

\end{description}



\end{fulllineitems}



\subsection{numpy.polynomial.Polynomial.convert}
\label{\detokenize{generated/generated/numpy.polynomial.Polynomial.convert:numpy-polynomial-polynomial-convert}}\label{\detokenize{generated/generated/numpy.polynomial.Polynomial.convert::doc}}
method
\index{convert() (numpy.polynomial.Polynomial method)@\spxentry{convert()}\spxextra{numpy.polynomial.Polynomial method}}

\begin{fulllineitems}
\phantomsection\label{\detokenize{generated/generated/numpy.polynomial.Polynomial.convert:numpy.polynomial.Polynomial.convert}}\pysiglinewithargsret{\sphinxcode{\sphinxupquote{Polynomial.}}\sphinxbfcode{\sphinxupquote{convert}}}{\emph{\DUrole{n}{domain}\DUrole{o}{=}\DUrole{default_value}{None}}, \emph{\DUrole{n}{kind}\DUrole{o}{=}\DUrole{default_value}{None}}, \emph{\DUrole{n}{window}\DUrole{o}{=}\DUrole{default_value}{None}}}{}
Convert series to a different kind and/or domain and/or window.
\begin{quote}\begin{description}
\item[{Parameters}] \leavevmode\begin{description}
\item[{\sphinxstylestrong{domain}}] \leavevmode{[}array\_like, optional{]}
The domain of the converted series. If the value is None,
the default domain of \sphinxtitleref{kind} is used.

\item[{\sphinxstylestrong{kind}}] \leavevmode{[}class, optional{]}
The polynomial series type class to which the current instance
should be converted. If kind is None, then the class of the
current instance is used.

\item[{\sphinxstylestrong{window}}] \leavevmode{[}array\_like, optional{]}
The window of the converted series. If the value is None,
the default window of \sphinxtitleref{kind} is used.

\end{description}

\item[{Returns}] \leavevmode\begin{description}
\item[{\sphinxstylestrong{new\_series}}] \leavevmode{[}series{]}
The returned class can be of different type than the current
instance and/or have a different domain and/or different
window.

\end{description}

\end{description}\end{quote}
\subsubsection*{Notes}

Conversion between domains and class types can result in
numerically ill defined series.

\end{fulllineitems}



\subsection{numpy.polynomial.Polynomial.copy}
\label{\detokenize{generated/generated/numpy.polynomial.Polynomial.copy:numpy-polynomial-polynomial-copy}}\label{\detokenize{generated/generated/numpy.polynomial.Polynomial.copy::doc}}
method
\index{copy() (numpy.polynomial.Polynomial method)@\spxentry{copy()}\spxextra{numpy.polynomial.Polynomial method}}

\begin{fulllineitems}
\phantomsection\label{\detokenize{generated/generated/numpy.polynomial.Polynomial.copy:numpy.polynomial.Polynomial.copy}}\pysiglinewithargsret{\sphinxcode{\sphinxupquote{Polynomial.}}\sphinxbfcode{\sphinxupquote{copy}}}{}{}
Return a copy.
\begin{quote}\begin{description}
\item[{Returns}] \leavevmode\begin{description}
\item[{\sphinxstylestrong{new\_series}}] \leavevmode{[}series{]}
Copy of self.

\end{description}

\end{description}\end{quote}

\end{fulllineitems}



\subsection{numpy.polynomial.Polynomial.cutdeg}
\label{\detokenize{generated/generated/numpy.polynomial.Polynomial.cutdeg:numpy-polynomial-polynomial-cutdeg}}\label{\detokenize{generated/generated/numpy.polynomial.Polynomial.cutdeg::doc}}
method
\index{cutdeg() (numpy.polynomial.Polynomial method)@\spxentry{cutdeg()}\spxextra{numpy.polynomial.Polynomial method}}

\begin{fulllineitems}
\phantomsection\label{\detokenize{generated/generated/numpy.polynomial.Polynomial.cutdeg:numpy.polynomial.Polynomial.cutdeg}}\pysiglinewithargsret{\sphinxcode{\sphinxupquote{Polynomial.}}\sphinxbfcode{\sphinxupquote{cutdeg}}}{\emph{\DUrole{n}{deg}}}{}
Truncate series to the given degree.

Reduce the degree of the series to \sphinxtitleref{deg} by discarding the
high order terms. If \sphinxtitleref{deg} is greater than the current degree a
copy of the current series is returned. This can be useful in least
squares where the coefficients of the high degree terms may be very
small.

\DUrole{versionmodified,added}{New in version 1.5.0.}
\begin{quote}\begin{description}
\item[{Parameters}] \leavevmode\begin{description}
\item[{\sphinxstylestrong{deg}}] \leavevmode{[}non\sphinxhyphen{}negative int{]}
The series is reduced to degree \sphinxtitleref{deg} by discarding the high
order terms. The value of \sphinxtitleref{deg} must be a non\sphinxhyphen{}negative integer.

\end{description}

\item[{Returns}] \leavevmode\begin{description}
\item[{\sphinxstylestrong{new\_series}}] \leavevmode{[}series{]}
New instance of series with reduced degree.

\end{description}

\end{description}\end{quote}

\end{fulllineitems}



\subsection{numpy.polynomial.Polynomial.degree}
\label{\detokenize{generated/generated/numpy.polynomial.Polynomial.degree:numpy-polynomial-polynomial-degree}}\label{\detokenize{generated/generated/numpy.polynomial.Polynomial.degree::doc}}
method
\index{degree() (numpy.polynomial.Polynomial method)@\spxentry{degree()}\spxextra{numpy.polynomial.Polynomial method}}

\begin{fulllineitems}
\phantomsection\label{\detokenize{generated/generated/numpy.polynomial.Polynomial.degree:numpy.polynomial.Polynomial.degree}}\pysiglinewithargsret{\sphinxcode{\sphinxupquote{Polynomial.}}\sphinxbfcode{\sphinxupquote{degree}}}{}{}
The degree of the series.

\DUrole{versionmodified,added}{New in version 1.5.0.}
\begin{quote}\begin{description}
\item[{Returns}] \leavevmode\begin{description}
\item[{\sphinxstylestrong{degree}}] \leavevmode{[}int{]}
Degree of the series, one less than the number of coefficients.

\end{description}

\end{description}\end{quote}

\end{fulllineitems}



\subsection{numpy.polynomial.Polynomial.deriv}
\label{\detokenize{generated/generated/numpy.polynomial.Polynomial.deriv:numpy-polynomial-polynomial-deriv}}\label{\detokenize{generated/generated/numpy.polynomial.Polynomial.deriv::doc}}
method
\index{deriv() (numpy.polynomial.Polynomial method)@\spxentry{deriv()}\spxextra{numpy.polynomial.Polynomial method}}

\begin{fulllineitems}
\phantomsection\label{\detokenize{generated/generated/numpy.polynomial.Polynomial.deriv:numpy.polynomial.Polynomial.deriv}}\pysiglinewithargsret{\sphinxcode{\sphinxupquote{Polynomial.}}\sphinxbfcode{\sphinxupquote{deriv}}}{\emph{\DUrole{n}{m}\DUrole{o}{=}\DUrole{default_value}{1}}}{}
Differentiate.

Return a series instance of that is the derivative of the current
series.
\begin{quote}\begin{description}
\item[{Parameters}] \leavevmode\begin{description}
\item[{\sphinxstylestrong{m}}] \leavevmode{[}non\sphinxhyphen{}negative int{]}
Find the derivative of order \sphinxtitleref{m}.

\end{description}

\item[{Returns}] \leavevmode\begin{description}
\item[{\sphinxstylestrong{new\_series}}] \leavevmode{[}series{]}
A new series representing the derivative. The domain is the same
as the domain of the differentiated series.

\end{description}

\end{description}\end{quote}

\end{fulllineitems}



\subsection{numpy.polynomial.Polynomial.fit}
\label{\detokenize{generated/generated/numpy.polynomial.Polynomial.fit:numpy-polynomial-polynomial-fit}}\label{\detokenize{generated/generated/numpy.polynomial.Polynomial.fit::doc}}
method
\index{fit() (numpy.polynomial.Polynomial class method)@\spxentry{fit()}\spxextra{numpy.polynomial.Polynomial class method}}

\begin{fulllineitems}
\phantomsection\label{\detokenize{generated/generated/numpy.polynomial.Polynomial.fit:numpy.polynomial.Polynomial.fit}}\pysiglinewithargsret{\sphinxbfcode{\sphinxupquote{classmethod }}\sphinxcode{\sphinxupquote{Polynomial.}}\sphinxbfcode{\sphinxupquote{fit}}}{\emph{\DUrole{n}{x}}, \emph{\DUrole{n}{y}}, \emph{\DUrole{n}{deg}}, \emph{\DUrole{n}{domain}\DUrole{o}{=}\DUrole{default_value}{None}}, \emph{\DUrole{n}{rcond}\DUrole{o}{=}\DUrole{default_value}{None}}, \emph{\DUrole{n}{full}\DUrole{o}{=}\DUrole{default_value}{False}}, \emph{\DUrole{n}{w}\DUrole{o}{=}\DUrole{default_value}{None}}, \emph{\DUrole{n}{window}\DUrole{o}{=}\DUrole{default_value}{None}}}{}
Least squares fit to data.

Return a series instance that is the least squares fit to the data
\sphinxtitleref{y} sampled at \sphinxtitleref{x}. The domain of the returned instance can be
specified and this will often result in a superior fit with less
chance of ill conditioning.
\begin{quote}\begin{description}
\item[{Parameters}] \leavevmode\begin{description}
\item[{\sphinxstylestrong{x}}] \leavevmode{[}array\_like, shape (M,){]}
x\sphinxhyphen{}coordinates of the M sample points \sphinxcode{\sphinxupquote{(x{[}i{]}, y{[}i{]})}}.

\item[{\sphinxstylestrong{y}}] \leavevmode{[}array\_like, shape (M,) or (M, K){]}
y\sphinxhyphen{}coordinates of the sample points. Several data sets of sample
points sharing the same x\sphinxhyphen{}coordinates can be fitted at once by
passing in a 2D\sphinxhyphen{}array that contains one dataset per column.

\item[{\sphinxstylestrong{deg}}] \leavevmode{[}int or 1\sphinxhyphen{}D array\_like{]}
Degree(s) of the fitting polynomials. If \sphinxtitleref{deg} is a single integer
all terms up to and including the \sphinxtitleref{deg}’th term are included in the
fit. For NumPy versions >= 1.11.0 a list of integers specifying the
degrees of the terms to include may be used instead.

\item[{\sphinxstylestrong{domain}}] \leavevmode{[}\{None, {[}beg, end{]}, {[}{]}\}, optional{]}
Domain to use for the returned series. If \sphinxcode{\sphinxupquote{None}},
then a minimal domain that covers the points \sphinxtitleref{x} is chosen.  If
\sphinxcode{\sphinxupquote{{[}{]}}} the class domain is used. The default value was the
class domain in NumPy 1.4 and \sphinxcode{\sphinxupquote{None}} in later versions.
The \sphinxcode{\sphinxupquote{{[}{]}}} option was added in numpy 1.5.0.

\item[{\sphinxstylestrong{rcond}}] \leavevmode{[}float, optional{]}
Relative condition number of the fit. Singular values smaller
than this relative to the largest singular value will be
ignored. The default value is len(x)*eps, where eps is the
relative precision of the float type, about 2e\sphinxhyphen{}16 in most
cases.

\item[{\sphinxstylestrong{full}}] \leavevmode{[}bool, optional{]}
Switch determining nature of return value. When it is False
(the default) just the coefficients are returned, when True
diagnostic information from the singular value decomposition is
also returned.

\item[{\sphinxstylestrong{w}}] \leavevmode{[}array\_like, shape (M,), optional{]}
Weights. If not None the contribution of each point
\sphinxcode{\sphinxupquote{(x{[}i{]},y{[}i{]})}} to the fit is weighted by \sphinxtitleref{w{[}i{]}}. Ideally the
weights are chosen so that the errors of the products
\sphinxcode{\sphinxupquote{w{[}i{]}*y{[}i{]}}} all have the same variance.  The default value is
None.

\DUrole{versionmodified,added}{New in version 1.5.0.}

\item[{\sphinxstylestrong{window}}] \leavevmode{[}\{{[}beg, end{]}\}, optional{]}
Window to use for the returned series. The default
value is the default class domain

\DUrole{versionmodified,added}{New in version 1.6.0.}

\end{description}

\item[{Returns}] \leavevmode\begin{description}
\item[{\sphinxstylestrong{new\_series}}] \leavevmode{[}series{]}
A series that represents the least squares fit to the data and
has the domain specified in the call.

\item[{\sphinxstylestrong{{[}resid, rank, sv, rcond{]}}}] \leavevmode{[}list{]}
These values are only returned if \sphinxtitleref{full} = True

resid – sum of squared residuals of the least squares fit
rank – the numerical rank of the scaled Vandermonde matrix
sv – singular values of the scaled Vandermonde matrix
rcond – value of \sphinxtitleref{rcond}.

For more details, see \sphinxtitleref{linalg.lstsq}.

\end{description}

\end{description}\end{quote}

\end{fulllineitems}



\subsection{numpy.polynomial.Polynomial.fromroots}
\label{\detokenize{generated/generated/numpy.polynomial.Polynomial.fromroots:numpy-polynomial-polynomial-fromroots}}\label{\detokenize{generated/generated/numpy.polynomial.Polynomial.fromroots::doc}}
method
\index{fromroots() (numpy.polynomial.Polynomial class method)@\spxentry{fromroots()}\spxextra{numpy.polynomial.Polynomial class method}}

\begin{fulllineitems}
\phantomsection\label{\detokenize{generated/generated/numpy.polynomial.Polynomial.fromroots:numpy.polynomial.Polynomial.fromroots}}\pysiglinewithargsret{\sphinxbfcode{\sphinxupquote{classmethod }}\sphinxcode{\sphinxupquote{Polynomial.}}\sphinxbfcode{\sphinxupquote{fromroots}}}{\emph{\DUrole{n}{roots}}, \emph{\DUrole{n}{domain}\DUrole{o}{=}\DUrole{default_value}{{[}{]}}}, \emph{\DUrole{n}{window}\DUrole{o}{=}\DUrole{default_value}{None}}}{}
Return series instance that has the specified roots.

Returns a series representing the product
\sphinxcode{\sphinxupquote{(x \sphinxhyphen{} r{[}0{]})*(x \sphinxhyphen{} r{[}1{]})*...*(x \sphinxhyphen{} r{[}n\sphinxhyphen{}1{]})}}, where \sphinxcode{\sphinxupquote{r}} is a
list of roots.
\begin{quote}\begin{description}
\item[{Parameters}] \leavevmode\begin{description}
\item[{\sphinxstylestrong{roots}}] \leavevmode{[}array\_like{]}
List of roots.

\item[{\sphinxstylestrong{domain}}] \leavevmode{[}\{{[}{]}, None, array\_like\}, optional{]}
Domain for the resulting series. If None the domain is the
interval from the smallest root to the largest. If {[}{]} the
domain is the class domain. The default is {[}{]}.

\item[{\sphinxstylestrong{window}}] \leavevmode{[}\{None, array\_like\}, optional{]}
Window for the returned series. If None the class window is
used. The default is None.

\end{description}

\item[{Returns}] \leavevmode\begin{description}
\item[{\sphinxstylestrong{new\_series}}] \leavevmode{[}series{]}
Series with the specified roots.

\end{description}

\end{description}\end{quote}

\end{fulllineitems}



\subsection{numpy.polynomial.Polynomial.has\_samecoef}
\label{\detokenize{generated/generated/numpy.polynomial.Polynomial.has_samecoef:numpy-polynomial-polynomial-has-samecoef}}\label{\detokenize{generated/generated/numpy.polynomial.Polynomial.has_samecoef::doc}}
method
\index{has\_samecoef() (numpy.polynomial.Polynomial method)@\spxentry{has\_samecoef()}\spxextra{numpy.polynomial.Polynomial method}}

\begin{fulllineitems}
\phantomsection\label{\detokenize{generated/generated/numpy.polynomial.Polynomial.has_samecoef:numpy.polynomial.Polynomial.has_samecoef}}\pysiglinewithargsret{\sphinxcode{\sphinxupquote{Polynomial.}}\sphinxbfcode{\sphinxupquote{has\_samecoef}}}{\emph{\DUrole{n}{other}}}{}
Check if coefficients match.

\DUrole{versionmodified,added}{New in version 1.6.0.}
\begin{quote}\begin{description}
\item[{Parameters}] \leavevmode\begin{description}
\item[{\sphinxstylestrong{other}}] \leavevmode{[}class instance{]}
The other class must have the \sphinxcode{\sphinxupquote{coef}} attribute.

\end{description}

\item[{Returns}] \leavevmode\begin{description}
\item[{\sphinxstylestrong{bool}}] \leavevmode{[}boolean{]}
True if the coefficients are the same, False otherwise.

\end{description}

\end{description}\end{quote}

\end{fulllineitems}



\subsection{numpy.polynomial.Polynomial.has\_samedomain}
\label{\detokenize{generated/generated/numpy.polynomial.Polynomial.has_samedomain:numpy-polynomial-polynomial-has-samedomain}}\label{\detokenize{generated/generated/numpy.polynomial.Polynomial.has_samedomain::doc}}
method
\index{has\_samedomain() (numpy.polynomial.Polynomial method)@\spxentry{has\_samedomain()}\spxextra{numpy.polynomial.Polynomial method}}

\begin{fulllineitems}
\phantomsection\label{\detokenize{generated/generated/numpy.polynomial.Polynomial.has_samedomain:numpy.polynomial.Polynomial.has_samedomain}}\pysiglinewithargsret{\sphinxcode{\sphinxupquote{Polynomial.}}\sphinxbfcode{\sphinxupquote{has\_samedomain}}}{\emph{\DUrole{n}{other}}}{}
Check if domains match.

\DUrole{versionmodified,added}{New in version 1.6.0.}
\begin{quote}\begin{description}
\item[{Parameters}] \leavevmode\begin{description}
\item[{\sphinxstylestrong{other}}] \leavevmode{[}class instance{]}
The other class must have the \sphinxcode{\sphinxupquote{domain}} attribute.

\end{description}

\item[{Returns}] \leavevmode\begin{description}
\item[{\sphinxstylestrong{bool}}] \leavevmode{[}boolean{]}
True if the domains are the same, False otherwise.

\end{description}

\end{description}\end{quote}

\end{fulllineitems}



\subsection{numpy.polynomial.Polynomial.has\_sametype}
\label{\detokenize{generated/generated/numpy.polynomial.Polynomial.has_sametype:numpy-polynomial-polynomial-has-sametype}}\label{\detokenize{generated/generated/numpy.polynomial.Polynomial.has_sametype::doc}}
method
\index{has\_sametype() (numpy.polynomial.Polynomial method)@\spxentry{has\_sametype()}\spxextra{numpy.polynomial.Polynomial method}}

\begin{fulllineitems}
\phantomsection\label{\detokenize{generated/generated/numpy.polynomial.Polynomial.has_sametype:numpy.polynomial.Polynomial.has_sametype}}\pysiglinewithargsret{\sphinxcode{\sphinxupquote{Polynomial.}}\sphinxbfcode{\sphinxupquote{has\_sametype}}}{\emph{\DUrole{n}{other}}}{}
Check if types match.

\DUrole{versionmodified,added}{New in version 1.7.0.}
\begin{quote}\begin{description}
\item[{Parameters}] \leavevmode\begin{description}
\item[{\sphinxstylestrong{other}}] \leavevmode{[}object{]}
Class instance.

\end{description}

\item[{Returns}] \leavevmode\begin{description}
\item[{\sphinxstylestrong{bool}}] \leavevmode{[}boolean{]}
True if other is same class as self

\end{description}

\end{description}\end{quote}

\end{fulllineitems}



\subsection{numpy.polynomial.Polynomial.has\_samewindow}
\label{\detokenize{generated/generated/numpy.polynomial.Polynomial.has_samewindow:numpy-polynomial-polynomial-has-samewindow}}\label{\detokenize{generated/generated/numpy.polynomial.Polynomial.has_samewindow::doc}}
method
\index{has\_samewindow() (numpy.polynomial.Polynomial method)@\spxentry{has\_samewindow()}\spxextra{numpy.polynomial.Polynomial method}}

\begin{fulllineitems}
\phantomsection\label{\detokenize{generated/generated/numpy.polynomial.Polynomial.has_samewindow:numpy.polynomial.Polynomial.has_samewindow}}\pysiglinewithargsret{\sphinxcode{\sphinxupquote{Polynomial.}}\sphinxbfcode{\sphinxupquote{has\_samewindow}}}{\emph{\DUrole{n}{other}}}{}
Check if windows match.

\DUrole{versionmodified,added}{New in version 1.6.0.}
\begin{quote}\begin{description}
\item[{Parameters}] \leavevmode\begin{description}
\item[{\sphinxstylestrong{other}}] \leavevmode{[}class instance{]}
The other class must have the \sphinxcode{\sphinxupquote{window}} attribute.

\end{description}

\item[{Returns}] \leavevmode\begin{description}
\item[{\sphinxstylestrong{bool}}] \leavevmode{[}boolean{]}
True if the windows are the same, False otherwise.

\end{description}

\end{description}\end{quote}

\end{fulllineitems}



\subsection{numpy.polynomial.Polynomial.identity}
\label{\detokenize{generated/generated/numpy.polynomial.Polynomial.identity:numpy-polynomial-polynomial-identity}}\label{\detokenize{generated/generated/numpy.polynomial.Polynomial.identity::doc}}
method
\index{identity() (numpy.polynomial.Polynomial class method)@\spxentry{identity()}\spxextra{numpy.polynomial.Polynomial class method}}

\begin{fulllineitems}
\phantomsection\label{\detokenize{generated/generated/numpy.polynomial.Polynomial.identity:numpy.polynomial.Polynomial.identity}}\pysiglinewithargsret{\sphinxbfcode{\sphinxupquote{classmethod }}\sphinxcode{\sphinxupquote{Polynomial.}}\sphinxbfcode{\sphinxupquote{identity}}}{\emph{\DUrole{n}{domain}\DUrole{o}{=}\DUrole{default_value}{None}}, \emph{\DUrole{n}{window}\DUrole{o}{=}\DUrole{default_value}{None}}}{}
Identity function.

If \sphinxcode{\sphinxupquote{p}} is the returned series, then \sphinxcode{\sphinxupquote{p(x) == x}} for all
values of x.
\begin{quote}\begin{description}
\item[{Parameters}] \leavevmode\begin{description}
\item[{\sphinxstylestrong{domain}}] \leavevmode{[}\{None, array\_like\}, optional{]}
If given, the array must be of the form \sphinxcode{\sphinxupquote{{[}beg, end{]}}}, where
\sphinxcode{\sphinxupquote{beg}} and \sphinxcode{\sphinxupquote{end}} are the endpoints of the domain. If None is
given then the class domain is used. The default is None.

\item[{\sphinxstylestrong{window}}] \leavevmode{[}\{None, array\_like\}, optional{]}
If given, the resulting array must be if the form
\sphinxcode{\sphinxupquote{{[}beg, end{]}}}, where \sphinxcode{\sphinxupquote{beg}} and \sphinxcode{\sphinxupquote{end}} are the endpoints of
the window. If None is given then the class window is used. The
default is None.

\end{description}

\item[{Returns}] \leavevmode\begin{description}
\item[{\sphinxstylestrong{new\_series}}] \leavevmode{[}series{]}
Series of representing the identity.

\end{description}

\end{description}\end{quote}

\end{fulllineitems}



\subsection{numpy.polynomial.Polynomial.integ}
\label{\detokenize{generated/generated/numpy.polynomial.Polynomial.integ:numpy-polynomial-polynomial-integ}}\label{\detokenize{generated/generated/numpy.polynomial.Polynomial.integ::doc}}
method
\index{integ() (numpy.polynomial.Polynomial method)@\spxentry{integ()}\spxextra{numpy.polynomial.Polynomial method}}

\begin{fulllineitems}
\phantomsection\label{\detokenize{generated/generated/numpy.polynomial.Polynomial.integ:numpy.polynomial.Polynomial.integ}}\pysiglinewithargsret{\sphinxcode{\sphinxupquote{Polynomial.}}\sphinxbfcode{\sphinxupquote{integ}}}{\emph{\DUrole{n}{m}\DUrole{o}{=}\DUrole{default_value}{1}}, \emph{\DUrole{n}{k}\DUrole{o}{=}\DUrole{default_value}{{[}{]}}}, \emph{\DUrole{n}{lbnd}\DUrole{o}{=}\DUrole{default_value}{None}}}{}
Integrate.

Return a series instance that is the definite integral of the
current series.
\begin{quote}\begin{description}
\item[{Parameters}] \leavevmode\begin{description}
\item[{\sphinxstylestrong{m}}] \leavevmode{[}non\sphinxhyphen{}negative int{]}
The number of integrations to perform.

\item[{\sphinxstylestrong{k}}] \leavevmode{[}array\_like{]}
Integration constants. The first constant is applied to the
first integration, the second to the second, and so on. The
list of values must less than or equal to \sphinxtitleref{m} in length and any
missing values are set to zero.

\item[{\sphinxstylestrong{lbnd}}] \leavevmode{[}Scalar{]}
The lower bound of the definite integral.

\end{description}

\item[{Returns}] \leavevmode\begin{description}
\item[{\sphinxstylestrong{new\_series}}] \leavevmode{[}series{]}
A new series representing the integral. The domain is the same
as the domain of the integrated series.

\end{description}

\end{description}\end{quote}

\end{fulllineitems}



\subsection{numpy.polynomial.Polynomial.linspace}
\label{\detokenize{generated/generated/numpy.polynomial.Polynomial.linspace:numpy-polynomial-polynomial-linspace}}\label{\detokenize{generated/generated/numpy.polynomial.Polynomial.linspace::doc}}
method
\index{linspace() (numpy.polynomial.Polynomial method)@\spxentry{linspace()}\spxextra{numpy.polynomial.Polynomial method}}

\begin{fulllineitems}
\phantomsection\label{\detokenize{generated/generated/numpy.polynomial.Polynomial.linspace:numpy.polynomial.Polynomial.linspace}}\pysiglinewithargsret{\sphinxcode{\sphinxupquote{Polynomial.}}\sphinxbfcode{\sphinxupquote{linspace}}}{\emph{\DUrole{n}{n}\DUrole{o}{=}\DUrole{default_value}{100}}, \emph{\DUrole{n}{domain}\DUrole{o}{=}\DUrole{default_value}{None}}}{}
Return x, y values at equally spaced points in domain.

Returns the x, y values at \sphinxtitleref{n} linearly spaced points across the
domain.  Here y is the value of the polynomial at the points x. By
default the domain is the same as that of the series instance.
This method is intended mostly as a plotting aid.

\DUrole{versionmodified,added}{New in version 1.5.0.}
\begin{quote}\begin{description}
\item[{Parameters}] \leavevmode\begin{description}
\item[{\sphinxstylestrong{n}}] \leavevmode{[}int, optional{]}
Number of point pairs to return. The default value is 100.

\item[{\sphinxstylestrong{domain}}] \leavevmode{[}\{None, array\_like\}, optional{]}
If not None, the specified domain is used instead of that of
the calling instance. It should be of the form \sphinxcode{\sphinxupquote{{[}beg,end{]}}}.
The default is None which case the class domain is used.

\end{description}

\item[{Returns}] \leavevmode\begin{description}
\item[{\sphinxstylestrong{x, y}}] \leavevmode{[}ndarray{]}
x is equal to linspace(self.domain{[}0{]}, self.domain{[}1{]}, n) and
y is the series evaluated at element of x.

\end{description}

\end{description}\end{quote}

\end{fulllineitems}



\subsection{numpy.polynomial.Polynomial.mapparms}
\label{\detokenize{generated/generated/numpy.polynomial.Polynomial.mapparms:numpy-polynomial-polynomial-mapparms}}\label{\detokenize{generated/generated/numpy.polynomial.Polynomial.mapparms::doc}}
method
\index{mapparms() (numpy.polynomial.Polynomial method)@\spxentry{mapparms()}\spxextra{numpy.polynomial.Polynomial method}}

\begin{fulllineitems}
\phantomsection\label{\detokenize{generated/generated/numpy.polynomial.Polynomial.mapparms:numpy.polynomial.Polynomial.mapparms}}\pysiglinewithargsret{\sphinxcode{\sphinxupquote{Polynomial.}}\sphinxbfcode{\sphinxupquote{mapparms}}}{}{}
Return the mapping parameters.

The returned values define a linear map \sphinxcode{\sphinxupquote{off + scl*x}} that is
applied to the input arguments before the series is evaluated. The
map depends on the \sphinxcode{\sphinxupquote{domain}} and \sphinxcode{\sphinxupquote{window}}; if the current
\sphinxcode{\sphinxupquote{domain}} is equal to the \sphinxcode{\sphinxupquote{window}} the resulting map is the
identity.  If the coefficients of the series instance are to be
used by themselves outside this class, then the linear function
must be substituted for the \sphinxcode{\sphinxupquote{x}} in the standard representation of
the base polynomials.
\begin{quote}\begin{description}
\item[{Returns}] \leavevmode\begin{description}
\item[{\sphinxstylestrong{off, scl}}] \leavevmode{[}float or complex{]}
The mapping function is defined by \sphinxcode{\sphinxupquote{off + scl*x}}.

\end{description}

\end{description}\end{quote}
\subsubsection*{Notes}

If the current domain is the interval \sphinxcode{\sphinxupquote{{[}l1, r1{]}}} and the window
is \sphinxcode{\sphinxupquote{{[}l2, r2{]}}}, then the linear mapping function \sphinxcode{\sphinxupquote{L}} is
defined by the equations:

\begin{sphinxVerbatim}[commandchars=\\\{\}]
\PYG{n}{L}\PYG{p}{(}\PYG{n}{l1}\PYG{p}{)} \PYG{o}{=} \PYG{n}{l2}
\PYG{n}{L}\PYG{p}{(}\PYG{n}{r1}\PYG{p}{)} \PYG{o}{=} \PYG{n}{r2}
\end{sphinxVerbatim}

\end{fulllineitems}



\subsection{numpy.polynomial.Polynomial.roots}
\label{\detokenize{generated/generated/numpy.polynomial.Polynomial.roots:numpy-polynomial-polynomial-roots}}\label{\detokenize{generated/generated/numpy.polynomial.Polynomial.roots::doc}}
method
\index{roots() (numpy.polynomial.Polynomial method)@\spxentry{roots()}\spxextra{numpy.polynomial.Polynomial method}}

\begin{fulllineitems}
\phantomsection\label{\detokenize{generated/generated/numpy.polynomial.Polynomial.roots:numpy.polynomial.Polynomial.roots}}\pysiglinewithargsret{\sphinxcode{\sphinxupquote{Polynomial.}}\sphinxbfcode{\sphinxupquote{roots}}}{}{}
Return the roots of the series polynomial.

Compute the roots for the series. Note that the accuracy of the
roots decrease the further outside the domain they lie.
\begin{quote}\begin{description}
\item[{Returns}] \leavevmode\begin{description}
\item[{\sphinxstylestrong{roots}}] \leavevmode{[}ndarray{]}
Array containing the roots of the series.

\end{description}

\end{description}\end{quote}

\end{fulllineitems}



\subsection{numpy.polynomial.Polynomial.trim}
\label{\detokenize{generated/generated/numpy.polynomial.Polynomial.trim:numpy-polynomial-polynomial-trim}}\label{\detokenize{generated/generated/numpy.polynomial.Polynomial.trim::doc}}
method
\index{trim() (numpy.polynomial.Polynomial method)@\spxentry{trim()}\spxextra{numpy.polynomial.Polynomial method}}

\begin{fulllineitems}
\phantomsection\label{\detokenize{generated/generated/numpy.polynomial.Polynomial.trim:numpy.polynomial.Polynomial.trim}}\pysiglinewithargsret{\sphinxcode{\sphinxupquote{Polynomial.}}\sphinxbfcode{\sphinxupquote{trim}}}{\emph{\DUrole{n}{tol}\DUrole{o}{=}\DUrole{default_value}{0}}}{}
Remove trailing coefficients

Remove trailing coefficients until a coefficient is reached whose
absolute value greater than \sphinxtitleref{tol} or the beginning of the series is
reached. If all the coefficients would be removed the series is set
to \sphinxcode{\sphinxupquote{{[}0{]}}}. A new series instance is returned with the new
coefficients.  The current instance remains unchanged.
\begin{quote}\begin{description}
\item[{Parameters}] \leavevmode\begin{description}
\item[{\sphinxstylestrong{tol}}] \leavevmode{[}non\sphinxhyphen{}negative number.{]}
All trailing coefficients less than \sphinxtitleref{tol} will be removed.

\end{description}

\item[{Returns}] \leavevmode\begin{description}
\item[{\sphinxstylestrong{new\_series}}] \leavevmode{[}series{]}
Contains the new set of coefficients.

\end{description}

\end{description}\end{quote}

\end{fulllineitems}



\subsection{numpy.polynomial.Polynomial.truncate}
\label{\detokenize{generated/generated/numpy.polynomial.Polynomial.truncate:numpy-polynomial-polynomial-truncate}}\label{\detokenize{generated/generated/numpy.polynomial.Polynomial.truncate::doc}}
method
\index{truncate() (numpy.polynomial.Polynomial method)@\spxentry{truncate()}\spxextra{numpy.polynomial.Polynomial method}}

\begin{fulllineitems}
\phantomsection\label{\detokenize{generated/generated/numpy.polynomial.Polynomial.truncate:numpy.polynomial.Polynomial.truncate}}\pysiglinewithargsret{\sphinxcode{\sphinxupquote{Polynomial.}}\sphinxbfcode{\sphinxupquote{truncate}}}{\emph{\DUrole{n}{size}}}{}
Truncate series to length \sphinxtitleref{size}.

Reduce the series to length \sphinxtitleref{size} by discarding the high
degree terms. The value of \sphinxtitleref{size} must be a positive integer. This
can be useful in least squares where the coefficients of the
high degree terms may be very small.
\begin{quote}\begin{description}
\item[{Parameters}] \leavevmode\begin{description}
\item[{\sphinxstylestrong{size}}] \leavevmode{[}positive int{]}
The series is reduced to length \sphinxtitleref{size} by discarding the high
degree terms. The value of \sphinxtitleref{size} must be a positive integer.

\end{description}

\item[{Returns}] \leavevmode\begin{description}
\item[{\sphinxstylestrong{new\_series}}] \leavevmode{[}series{]}
New instance of series with truncated coefficients.

\end{description}

\end{description}\end{quote}

\end{fulllineitems}

\subsubsection*{Properties}


\begin{savenotes}\sphinxatlongtablestart\begin{longtable}[c]{\X{1}{2}\X{1}{2}}
\hline

\endfirsthead

\multicolumn{2}{c}%
{\makebox[0pt]{\sphinxtablecontinued{\tablename\ \thetable{} \textendash{} continued from previous page}}}\\
\hline

\endhead

\hline
\multicolumn{2}{r}{\makebox[0pt][r]{\sphinxtablecontinued{continues on next page}}}\\
\endfoot

\endlastfoot

{\hyperref[\detokenize{generated/generated/numpy.polynomial.Polynomial.domain:numpy.polynomial.Polynomial.domain}]{\sphinxcrossref{\sphinxcode{\sphinxupquote{domain}}}}}
&

\\
\hline
{\hyperref[\detokenize{generated/generated/numpy.polynomial.Polynomial.maxpower:numpy.polynomial.Polynomial.maxpower}]{\sphinxcrossref{\sphinxcode{\sphinxupquote{maxpower}}}}}
&

\\
\hline
{\hyperref[\detokenize{generated/generated/numpy.polynomial.Polynomial.nickname:numpy.polynomial.Polynomial.nickname}]{\sphinxcrossref{\sphinxcode{\sphinxupquote{nickname}}}}}
&

\\
\hline
{\hyperref[\detokenize{generated/generated/numpy.polynomial.Polynomial.window:numpy.polynomial.Polynomial.window}]{\sphinxcrossref{\sphinxcode{\sphinxupquote{window}}}}}
&

\\
\hline
\end{longtable}\sphinxatlongtableend\end{savenotes}


\subsection{numpy.polynomial.Polynomial.domain}
\label{\detokenize{generated/generated/numpy.polynomial.Polynomial.domain:numpy-polynomial-polynomial-domain}}\label{\detokenize{generated/generated/numpy.polynomial.Polynomial.domain::doc}}
attribute
\index{domain (numpy.polynomial.Polynomial attribute)@\spxentry{domain}\spxextra{numpy.polynomial.Polynomial attribute}}

\begin{fulllineitems}
\phantomsection\label{\detokenize{generated/generated/numpy.polynomial.Polynomial.domain:numpy.polynomial.Polynomial.domain}}\pysigline{\sphinxcode{\sphinxupquote{Polynomial.}}\sphinxbfcode{\sphinxupquote{domain}}\sphinxbfcode{\sphinxupquote{ = array({[}\sphinxhyphen{}1,  1{]})}}}~
\end{fulllineitems}



\subsection{numpy.polynomial.Polynomial.maxpower}
\label{\detokenize{generated/generated/numpy.polynomial.Polynomial.maxpower:numpy-polynomial-polynomial-maxpower}}\label{\detokenize{generated/generated/numpy.polynomial.Polynomial.maxpower::doc}}
attribute
\index{maxpower (numpy.polynomial.Polynomial attribute)@\spxentry{maxpower}\spxextra{numpy.polynomial.Polynomial attribute}}

\begin{fulllineitems}
\phantomsection\label{\detokenize{generated/generated/numpy.polynomial.Polynomial.maxpower:numpy.polynomial.Polynomial.maxpower}}\pysigline{\sphinxcode{\sphinxupquote{Polynomial.}}\sphinxbfcode{\sphinxupquote{maxpower}}\sphinxbfcode{\sphinxupquote{ = 100}}}~
\end{fulllineitems}



\subsection{numpy.polynomial.Polynomial.nickname}
\label{\detokenize{generated/generated/numpy.polynomial.Polynomial.nickname:numpy-polynomial-polynomial-nickname}}\label{\detokenize{generated/generated/numpy.polynomial.Polynomial.nickname::doc}}
attribute
\index{nickname (numpy.polynomial.Polynomial attribute)@\spxentry{nickname}\spxextra{numpy.polynomial.Polynomial attribute}}

\begin{fulllineitems}
\phantomsection\label{\detokenize{generated/generated/numpy.polynomial.Polynomial.nickname:numpy.polynomial.Polynomial.nickname}}\pysigline{\sphinxcode{\sphinxupquote{Polynomial.}}\sphinxbfcode{\sphinxupquote{nickname}}\sphinxbfcode{\sphinxupquote{ = 'poly'}}}~
\end{fulllineitems}



\subsection{numpy.polynomial.Polynomial.window}
\label{\detokenize{generated/generated/numpy.polynomial.Polynomial.window:numpy-polynomial-polynomial-window}}\label{\detokenize{generated/generated/numpy.polynomial.Polynomial.window::doc}}
attribute
\index{window (numpy.polynomial.Polynomial attribute)@\spxentry{window}\spxextra{numpy.polynomial.Polynomial attribute}}

\begin{fulllineitems}
\phantomsection\label{\detokenize{generated/generated/numpy.polynomial.Polynomial.window:numpy.polynomial.Polynomial.window}}\pysigline{\sphinxcode{\sphinxupquote{Polynomial.}}\sphinxbfcode{\sphinxupquote{window}}\sphinxbfcode{\sphinxupquote{ = array({[}\sphinxhyphen{}1,  1{]})}}}~
\end{fulllineitems}


\end{fulllineitems}



\chapter{Microsoft 365 docs navigation guide}
\label{\detokenize{markdown:microsoft-365-docs-navigation-guide}}\label{\detokenize{markdown::doc}}
This topic provides some tips and tricks for navigating the Microsoft 365 technical documentation space.


\section{Impact to customers who don’t transition}
\label{\detokenize{markdown:impact-to-customers-who-don-t-transition}}
The following table summarizes the impact to customers who don’t transition from a Microsoft 365 Business Preview subscription to a Microsoft 365 Business subscription.




\section{Hub page}
\label{\detokenize{markdown:hub-page}}
The Microsoft 365 hub page can be found at \sphinxurl{https://aka.ms/microsoft365docs} and is the entry point for finding relevant Microsoft 365 content.

You can always navigate back to this page by selecting \sphinxstylestrong{Microsoft 365} from the header at the top of every page within the Microsoft 365 technical documentation set:


\chapter{JupyterLab Demo Start}
\label{\detokenize{markdown:jupyterlab-demo-start}}
JupyterLab: The next generation user interface for Project Jupyter

https://github.com/jupyter/jupyterlab

It started as a collaboration between:
\begin{itemize}
\item {} 
Project Jupyter

\item {} 
Bloomberg

\item {} 
(then) Continuum

\end{itemize}




\section{Tables}
\label{\detokenize{markdown:tables}}


and now involves many other{\hyperref[\detokenize{markdown:footnote_1}]{\emph{1}}}  people from many other places (not purely academic or business)

1: 주석

\begin{sphinxVerbatim}[commandchars=\\\{\}]
\PYG{k+kd}{function} \PYG{n+nx}{fancyAlert}\PYG{p}{(}\PYG{n+nx}{arg}\PYG{p}{)} \PYG{p}{\PYGZob{}}
  \PYG{k}{if}\PYG{p}{(}\PYG{n+nx}{arg}\PYG{p}{)} \PYG{p}{\PYGZob{}}
    \PYG{n+nx}{\PYGZdl{}}\PYG{p}{.}\PYG{n+nx}{facebox}\PYG{p}{(}\PYG{p}{\PYGZob{}}\PYG{n+nx}{div}\PYG{o}{:}\PYG{l+s+s1}{\PYGZsq{}\PYGZsh{}foo\PYGZsq{}}\PYG{p}{\PYGZcb{}}\PYG{p}{)}
  \PYG{p}{\PYGZcb{}}
\PYG{p}{\PYGZcb{}}
\end{sphinxVerbatim}


\section{1) Building blocks of interactive computing}
\label{\detokenize{markdown:building-blocks-of-interactive-computing}}

\subsection{Start with the launcher}
\label{\detokenize{markdown:start-with-the-launcher}}
Use it to open different activities:
\begin{itemize}
\item {} 
Notebook

\item {} 
Console

\item {} 
Editor

\item {} 
Terminal

\end{itemize}


\subsection{Notebooks}
\label{\detokenize{markdown:notebooks}}\begin{itemize}
\item {} 
Open example notebooks to show that notebooks still work

\item {} 
Collapse input/output

\item {} 
Drag and drop cells

\end{itemize}


\subsection{Demonstrate left panel plugins:}
\label{\detokenize{markdown:demonstrate-left-panel-plugins}}\begin{itemize}
\item {} 
File Browser (file operations, context menu, including drag and drop)

\item {} 
Running

\item {} 
Command Palette (fuzzy searching for ‘new’)

\end{itemize}


\subsection{Markdown example}
\label{\detokenize{markdown:markdown-example}}\begin{itemize}
\item {} 
Open \sphinxcode{\sphinxupquote{markdown\_python.md}} in the File Editor

\item {} 
View the rendered markdown, arrange side by side

\item {} 
Attach a Kernel/Console and run the code by selecting blocks and pressing
\sphinxcode{\sphinxupquote{Shift+Enter}}

\end{itemize}


\subsection{Arrange the building blocks in the main area}
\label{\detokenize{markdown:arrange-the-building-blocks-in-the-main-area}}
The dock panel allows you to arrange the activites into an
arbitrary layout.

Tabs and single document mode allow you to focus.


\section{4) File handlers}
\label{\detokenize{markdown:file-handlers}}
JupyterLab has a powerful and extensible architecture for handling a wide range of file formats:
\begin{itemize}
\item {} 
CSV
\begin{itemize}
\item {} 
\sphinxcode{\sphinxupquote{./data/iris.csv}} (small)

\item {} 
\sphinxcode{\sphinxupquote{TCGA\_Data}} (small to medium)

\item {} 
Urban\_Data\_Challenge: \sphinxcode{\sphinxupquote{data/big.csv}}

\end{itemize}

\item {} 
Images
\begin{itemize}
\item {} 
\sphinxcode{\sphinxupquote{data/hubble.png}}

\end{itemize}

\item {} 
Vega\sphinxhyphen{}Lite
\begin{itemize}
\item {} 
\sphinxcode{\sphinxupquote{data/vega.vl.json}}

\end{itemize}

\item {} 
Open DC museum GeoJSON file from \sphinxhref{http://opendata.dc.gov/datasets/2e65fc16edc3481989d2cc17e6f8c533\_54}{OpenData DC}%
\begin{footnote}[10]\sphinxAtStartFootnote
\sphinxnolinkurl{http://opendata.dc.gov/datasets/2e65fc16edc3481989d2cc17e6f8c533\_54}
%
\end{footnote}: \sphinxcode{\sphinxupquote{data/Museums\_in\_DC.geojson}}

\item {} 
Notebook demonstrating bqplot widgets: \sphinxcode{\sphinxupquote{notebooks/bqplot.ipynb}}

\end{itemize}


\section{5) Find and Replace}
\label{\detokenize{markdown:find-and-replace}}
first class support for find and replace across JupyterLab, currently supported in
notebooks and text files and is extensible for other widgets who wish to support it.


\section{6) Status Bar}
\label{\detokenize{markdown:status-bar}}
We have integrated the JupyterLab Status Bar package package into the core distribution. Extensions can add their own status to it as well


\section{7) Printing}
\label{\detokenize{markdown:printing}}
A printing system allows extensions to customize how documents and activities are printed.


\section{8) JupyterHub}
\label{\detokenize{markdown:jupyterhub}}
We now include the JupyterHub extension as a core JupyterLab extension, so you no longer need to install @jupyterlab/hub\sphinxhyphen{}extension (supporting multi\sphinxhyphen{}user + authentication workflows)


\section{9) Plugin architecture}
\label{\detokenize{markdown:plugin-architecture}}
The genius of open\sphinxhyphen{}source is being able to shape your tools to your heart’s content.

Just like Jupyter is built on top of building blocks of the protocol and message spec, \sphinxstyleemphasis{you} can build on this platform for your workflow.
\begin{itemize}
\item {} 
Everything in JupyterLab is an extension, including everything we have demoed

\item {} 
Extensions are just \sphinxcode{\sphinxupquote{npm}} packages with metadata

\item {} 
Anyone can create, package, ship plugins

\item {} 
Extension can, for example:
\begin{itemize}
\item {} 
Add things to command palette, menu

\item {} 
Add viewers for documents

\item {} 
Expose other controls (e.g., manage a spark cluster?)

\item {} 
Provide more capabilities to the system

\end{itemize}

\end{itemize}


\section{What will you build?}
\label{\detokenize{markdown:what-will-you-build}}

\chapter{Markdown}
\label{\detokenize{markdown:markdown}}
Sphinx can be configured to use markdown using the \sphinxhref{https://github.com/readthedocs/recommonmark}{recommonmark}%
\begin{footnote}[11]\sphinxAtStartFootnote
\sphinxnolinkurl{https://github.com/readthedocs/recommonmark}
%
\end{footnote}
extension. recommonmark is strict and does not natively support tables or common extensions
to markdown.


\bigskip\hrule\bigskip



\section{Body copy}
\label{\detokenize{markdown:body-copy}}
Lorem ipsum dolor sit amet, consectetur adipiscing elit. Cras arcu libero,
mollis sed massa vel, \sphinxstyleemphasis{ornare viverra ex}. Mauris a ullamcorper lacus. Nullam
urna elit, malesuada eget finibus ut, ullamcorper ac tortor. Vestibulum sodales
pulvinar nisl, pharetra aliquet est. Quisque volutpat erat ac nisi accumsan
tempor.

\sphinxstylestrong{Sed suscipit}, orci non pretium pretium, quam mi gravida metus, vel
venenatis justo est condimentum diam. Maecenas non ornare justo. Nam a ipsum
eros. {\hyperref[\detokenize{markdown:}]{\emph{Nulla aliquam}}} orci sit amet nisl posuere malesuada. Proin aliquet
nulla velit, quis ultricies orci feugiat et. \sphinxcode{\sphinxupquote{Ut tincidunt sollicitudin}}
tincidunt. Aenean ullamcorper sit amet nulla at interdum.


\section{Headings}
\label{\detokenize{markdown:headings}}

\subsection{The 3rd level}
\label{\detokenize{markdown:the-3rd-level}}

\subsubsection{The 4th level}
\label{\detokenize{markdown:the-4th-level}}

\paragraph{The 5th level}
\label{\detokenize{markdown:the-5th-level}}

\subparagraph{The 6th level}
\label{\detokenize{markdown:the-6th-level}}

\section{Headings with secondary text}
\label{\detokenize{markdown:headings-small-with-secondary-text-small}}

\subsection{The 3rd level with secondary text}
\label{\detokenize{markdown:the-3rd-level-small-with-secondary-text-small}}

\subsubsection{The 4th level with secondary text}
\label{\detokenize{markdown:the-4th-level-small-with-secondary-text-small}}

\paragraph{The 5th level with secondary text}
\label{\detokenize{markdown:the-5th-level-small-with-secondary-text-small}}

\subparagraph{The 6th level with secondary text}
\label{\detokenize{markdown:the-6th-level-small-with-secondary-text-small}}

\section{Blockquotes}
\label{\detokenize{markdown:blockquotes}}\begin{quote}

Morbi eget dapibus felis. Vivamus venenatis porttitor tortor sit amet rutrum.
Pellentesque aliquet quam enim, eu volutpat urna rutrum a. Nam vehicula nunc
mauris, a ultricies libero efficitur sed. \sphinxstyleemphasis{Class aptent} taciti sociosqu ad
litora torquent per conubia nostra, per inceptos himenaeos. Sed molestie
imperdiet consectetur.
\end{quote}


\subsection{Blockquote nesting}
\label{\detokenize{markdown:blockquote-nesting}}\begin{quote}

\sphinxstylestrong{Sed aliquet}, neque at rutrum mollis, neque nisi tincidunt nibh, vitae
faucibus lacus nunc at lacus. Nunc scelerisque, quam id cursus sodales, lorem
{\hyperref[\detokenize{markdown:}]{\emph{libero fermentum}}} urna, ut efficitur elit ligula et nunc.
\end{quote}
\begin{quote}
\begin{quote}

Mauris dictum mi lacus, sit amet pellentesque urna vehicula fringilla.
Ut sit amet placerat ante. Proin sed elementum nulla. Nunc vitae sem odio.
Suspendisse ac eros arcu. Vivamus orci erat, volutpat a tempor et, rutrum.
eu odio.
\end{quote}
\end{quote}
\begin{quote}
\begin{quote}
\begin{quote}

\sphinxcode{\sphinxupquote{Suspendisse rutrum facilisis risus}}, eu posuere neque commodo a.
Interdum et malesuada fames ac ante ipsum primis in faucibus. Sed nec leo
bibendum, sodales mauris ut, tincidunt massa.
\end{quote}
\end{quote}
\end{quote}


\subsection{Other content blocks}
\label{\detokenize{markdown:other-content-blocks}}\begin{quote}

Vestibulum vitae orci quis ante viverra ultricies ut eget turpis. Sed eu
lectus dapibus, eleifend nulla varius, lobortis turpis. In ac hendrerit nisl,
sit amet laoreet nibh.
\end{quote}

\begin{sphinxVerbatim}[commandchars=\\\{\}]
\PYG{n}{var} \PYG{n}{\PYGZus{}extends} \PYG{o}{=} \PYG{n}{function}\PYG{p}{(}\PYG{n}{target}\PYG{p}{)} \PYG{p}{\PYGZob{}}
  \PYG{k}{for} \PYG{p}{(}\PYG{n}{var} \PYG{n}{i} \PYG{o}{=} \PYG{l+m+mi}{1}\PYG{p}{;} \PYG{n}{i} \PYG{o}{\PYGZlt{}} \PYG{n}{arguments}\PYG{o}{.}\PYG{n}{length}\PYG{p}{;} \PYG{n}{i}\PYG{o}{+}\PYG{o}{+}\PYG{p}{)} \PYG{p}{\PYGZob{}}
    \PYG{n}{var} \PYG{n}{source} \PYG{o}{=} \PYG{n}{arguments}\PYG{p}{[}\PYG{n}{i}\PYG{p}{]}\PYG{p}{;}
    \PYG{k}{for} \PYG{p}{(}\PYG{n}{var} \PYG{n}{key} \PYG{o+ow}{in} \PYG{n}{source}\PYG{p}{)} \PYG{p}{\PYGZob{}}
      \PYG{n}{target}\PYG{p}{[}\PYG{n}{key}\PYG{p}{]} \PYG{o}{=} \PYG{n}{source}\PYG{p}{[}\PYG{n}{key}\PYG{p}{]}\PYG{p}{;}
    \PYG{p}{\PYGZcb{}}
  \PYG{p}{\PYGZcb{}}
  \PYG{k}{return} \PYG{n}{target}\PYG{p}{;}
\PYG{p}{\PYGZcb{}}\PYG{p}{;}
\end{sphinxVerbatim}
\begin{quote}
\begin{quote}

Praesent at \sphinxcode{\sphinxupquote{:::js return target}}, sodales nibh vel, tempor felis. Fusce
vel lacinia lacus. Suspendisse rhoncus nunc non nisi iaculis ultrices.
Donec consectetur mauris non neque imperdiet, eget volutpat libero.
\end{quote}
\end{quote}


\section{Lists}
\label{\detokenize{markdown:lists}}

\subsection{Unordered lists}
\label{\detokenize{markdown:unordered-lists}}\begin{itemize}
\item {} 
Sed sagittis eleifend rutrum. Donec vitae suscipit est. Nullam tempus tellus
non sem sollicitudin, quis rutrum leo facilisis. Nulla tempor lobortis orci,
at elementum urna sodales vitae. In in vehicula nulla, quis ornare libero.
\begin{itemize}
\item {} 
Duis mollis est eget nibh volutpat, fermentum aliquet dui mollis.

\item {} 
Nam vulputate tincidunt fringilla.

\item {} 
Nullam dignissim ultrices urna non auctor.

\end{itemize}

\item {} 
Aliquam metus eros, pretium sed nulla venenatis, faucibus auctor ex. Proin ut
eros sed sapien ullamcorper consequat. Nunc ligula ante, fringilla at aliquam
ac, aliquet sed mauris.

\item {} 
Nulla et rhoncus turpis. Mauris ultricies elementum leo. Duis efficitur
accumsan nibh eu mattis. Vivamus tempus velit eros, porttitor placerat nibh
lacinia sed. Aenean in finibus diam.

\end{itemize}


\subsection{Ordered lists}
\label{\detokenize{markdown:ordered-lists}}\begin{enumerate}
\sphinxsetlistlabels{\arabic}{enumi}{enumii}{}{.}%
\item {} 
Integer vehicula feugiat magna, a mollis tellus. Nam mollis ex ante, quis
elementum eros tempor rutrum. Aenean efficitur lobortis lacinia. Nulla
consectetur feugiat sodales.

\item {} 
Cum sociis natoque penatibus et magnis dis parturient montes, nascetur
ridiculus mus. Aliquam ornare feugiat quam et egestas. Nunc id erat et quam
pellentesque lacinia eu vel odio.
\begin{enumerate}
\sphinxsetlistlabels{\arabic}{enumii}{enumiii}{}{.}%
\item {} 
Vivamus venenatis porttitor tortor sit amet rutrum. Pellentesque aliquet
quam enim, eu volutpat urna rutrum a. Nam vehicula nunc mauris, a
ultricies libero efficitur sed.
\begin{enumerate}
\sphinxsetlistlabels{\arabic}{enumiii}{enumiv}{}{.}%
\item {} 
Mauris dictum mi lacus

\item {} 
Ut sit amet placerat ante

\item {} 
Suspendisse ac eros arcu

\end{enumerate}

\item {} 
Morbi eget dapibus felis. Vivamus venenatis porttitor tortor sit amet
rutrum. Pellentesque aliquet quam enim, eu volutpat urna rutrum a. Sed
aliquet, neque at rutrum mollis, neque nisi tincidunt nibh.

\item {} 
Pellentesque eget \sphinxcode{\sphinxupquote{:::js var \_extends}} ornare tellus, ut gravida mi.

\end{enumerate}

\begin{sphinxVerbatim}[commandchars=\\\{\}]
\PYG{n}{var} \PYG{n}{\PYGZus{}extends} \PYG{o}{=} \PYG{n}{function}\PYG{p}{(}\PYG{n}{target}\PYG{p}{)} \PYG{p}{\PYGZob{}}
  \PYG{k}{for} \PYG{p}{(}\PYG{n}{var} \PYG{n}{i} \PYG{o}{=} \PYG{l+m+mi}{1}\PYG{p}{;} \PYG{n}{i} \PYG{o}{\PYGZlt{}} \PYG{n}{arguments}\PYG{o}{.}\PYG{n}{length}\PYG{p}{;} \PYG{n}{i}\PYG{o}{+}\PYG{o}{+}\PYG{p}{)} \PYG{p}{\PYGZob{}}
    \PYG{n}{var} \PYG{n}{source} \PYG{o}{=} \PYG{n}{arguments}\PYG{p}{[}\PYG{n}{i}\PYG{p}{]}\PYG{p}{;}
    \PYG{k}{for} \PYG{p}{(}\PYG{n}{var} \PYG{n}{key} \PYG{o+ow}{in} \PYG{n}{source}\PYG{p}{)} \PYG{p}{\PYGZob{}}
      \PYG{n}{target}\PYG{p}{[}\PYG{n}{key}\PYG{p}{]} \PYG{o}{=} \PYG{n}{source}\PYG{p}{[}\PYG{n}{key}\PYG{p}{]}\PYG{p}{;}
    \PYG{p}{\PYGZcb{}}
  \PYG{p}{\PYGZcb{}}
  \PYG{k}{return} \PYG{n}{target}\PYG{p}{;}
\PYG{p}{\PYGZcb{}}\PYG{p}{;}
\end{sphinxVerbatim}

\item {} 
Vivamus id mi enim. Integer id turpis sapien. Ut condimentum lobortis
sagittis. Aliquam purus tellus, faucibus eget urna at, iaculis venenatis
nulla. Vivamus a pharetra leo.

\end{enumerate}


\subsection{Definition lists}
\label{\detokenize{markdown:definition-lists}}
\sphinxstylestrong{Not supported in commonmark, but you can use a rst definition list inside a
fenced eval\_rst block.}
\begin{description}
\item[{Lorem ipsum dolor sit amet}] \leavevmode
Sed sagittis eleifend rutrum. Donec vitae suscipit est. Nullam tempus
tellus non sem sollicitudin, quis rutrum leo facilisis. Nulla tempor
lobortis orci, at elementum urna sodales vitae. In in vehicula nulla.

Duis mollis est eget nibh volutpat, fermentum aliquet dui mollis. Nam
vulputate tincidunt fringilla. Nullam dignissim ultrices urna non
auctor.

\item[{Cras arcu libero}] \leavevmode
Aliquam metus eros, pretium sed nulla venenatis, faucibus auctor ex.
Proin ut eros sed sapien ullamcorper consequat. Nunc ligula ante,
fringilla at aliquam ac, aliquet sed mauris.

\end{description}


\section{Code blocks}
\label{\detokenize{markdown:code-blocks}}

\subsection{Inline}
\label{\detokenize{markdown:inline}}
Morbi eget \sphinxcode{\sphinxupquote{dapibus felis}}. Vivamus \sphinxstyleemphasis{\sphinxcode{\sphinxupquote{venenatis porttitor}}} tortor sit amet
rutrum. Class aptent taciti sociosqu ad litora torquent per conubia nostra,
per inceptos himenaeos. {\hyperref[\detokenize{markdown:}]{\emph{\sphinxcode{\sphinxupquote{Pellentesque aliquet quam enim}}}}}, eu volutpat urna
rutrum a.

Nam vehicula nunc \sphinxcode{\sphinxupquote{:::js return target}} mauris, a ultricies libero efficitur
sed. Sed molestie imperdiet consectetur. Vivamus a pharetra leo. Pellentesque
eget ornare tellus, ut gravida mi. Fusce vel lacinia lacus.


\subsection{Listing}
\label{\detokenize{markdown:listing}}
\begin{sphinxVerbatim}[commandchars=\\\{\}]
\PYG{n}{var} \PYG{n}{\PYGZus{}extends} \PYG{o}{=} \PYG{n}{function}\PYG{p}{(}\PYG{n}{target}\PYG{p}{)} \PYG{p}{\PYGZob{}}
  \PYG{k}{for} \PYG{p}{(}\PYG{n}{var} \PYG{n}{i} \PYG{o}{=} \PYG{l+m+mi}{1}\PYG{p}{;} \PYG{n}{i} \PYG{o}{\PYGZlt{}} \PYG{n}{arguments}\PYG{o}{.}\PYG{n}{length}\PYG{p}{;} \PYG{n}{i}\PYG{o}{+}\PYG{o}{+}\PYG{p}{)} \PYG{p}{\PYGZob{}}
    \PYG{n}{var} \PYG{n}{source} \PYG{o}{=} \PYG{n}{arguments}\PYG{p}{[}\PYG{n}{i}\PYG{p}{]}\PYG{p}{;}
    \PYG{k}{for} \PYG{p}{(}\PYG{n}{var} \PYG{n}{key} \PYG{o+ow}{in} \PYG{n}{source}\PYG{p}{)} \PYG{p}{\PYGZob{}}
      \PYG{n}{target}\PYG{p}{[}\PYG{n}{key}\PYG{p}{]} \PYG{o}{=} \PYG{n}{source}\PYG{p}{[}\PYG{n}{key}\PYG{p}{]}\PYG{p}{;}
    \PYG{p}{\PYGZcb{}}
  \PYG{p}{\PYGZcb{}}
  \PYG{k}{return} \PYG{n}{target}\PYG{p}{;}
\PYG{p}{\PYGZcb{}}\PYG{p}{;}
\end{sphinxVerbatim}


\section{Horizontal rules}
\label{\detokenize{markdown:horizontal-rules}}
Aenean in finibus diam. Duis mollis est eget nibh volutpat, fermentum aliquet
dui mollis. Nam vulputate tincidunt fringilla. Nullam dignissim ultrices urna
non auctor.


\bigskip\hrule\bigskip


Integer vehicula feugiat magna, a mollis tellus. Nam mollis ex ante, quis
elementum eros tempor rutrum. Aenean efficitur lobortis lacinia. Nulla
consectetur feugiat sodales.


\section{Data tables}
\label{\detokenize{markdown:data-tables}}
\sphinxstylestrong{Note}: Markdown table syntax requires \sphinxcode{\sphinxupquote{sphinx\_markdown\_tables}}



Sed sagittis eleifend rutrum. Donec vitae suscipit est. Nullam tempus tellus
non sem sollicitudin, quis rutrum leo facilisis. Nulla tempor lobortis orci,
at elementum urna sodales vitae. In in vehicula nulla, quis ornare libero.



Vestibulum vitae orci quis ante viverra ultricies ut eget turpis. Sed eu
lectus dapibus, eleifend nulla varius, lobortis turpis. In ac hendrerit nisl,
sit amet laoreet nibh.




\chapter{rst Cheatsheet}
\label{\detokenize{rst-cheatsheet/rst-cheatsheet:rst-cheatsheet}}\label{\detokenize{rst-cheatsheet/rst-cheatsheet::doc}}
The \sphinxhref{https://github.com/ralsina/rst-cheatsheet}{rst Cheatsheet}%
\begin{footnote}[12]\sphinxAtStartFootnote
\sphinxnolinkurl{https://github.com/ralsina/rst-cheatsheet}
%
\end{footnote} covers a wide range of
rst markup. It and its contents are
\sphinxhref{http://creativecommons.org/licenses/by/3.0/de/deed.en\_GB}{CC licensed}%
\begin{footnote}[13]\sphinxAtStartFootnote
\sphinxnolinkurl{http://creativecommons.org/licenses/by/3.0/de/deed.en\_GB}
%
\end{footnote}.


\section{Inline Markup}
\label{\detokenize{rst-cheatsheet/rst-cheatsheet:inline-markup}}
Inline markup allows words and phrases within text to have character styles (like italics and boldface) and functionality (like hyperlinks).


\begin{savenotes}\sphinxattablestart
\centering
\begin{tabular}[t]{|*{2}{\X{1}{2}|}}
\hline

\begin{sphinxVerbatimintable}[commandchars=\\\{\}]
\PYG{o}{*}\PYG{n}{emphasis}\PYG{o}{*}
\end{sphinxVerbatimintable}
&
\sphinxstyleemphasis{emphasis}
\\
\hline
\begin{sphinxVerbatimintable}[commandchars=\\\{\}]
\PYG{o}{*}\PYG{o}{*}\PYG{n}{strong} \PYG{n}{emphasis}\PYG{o}{*}\PYG{o}{*}
\end{sphinxVerbatimintable}
&
\sphinxstylestrong{strong emphasis}
\\
\hline
\begin{sphinxVerbatimintable}[commandchars=\\\{\}]
`interpreted text`
\end{sphinxVerbatimintable}
&
The rendering and meaning of interpreted text
is domain\sphinxhyphen{} or application\sphinxhyphen{}dependent.
\\
\hline
\begin{sphinxVerbatimintable}[commandchars=\\\{\}]
``inline literal``
\end{sphinxVerbatimintable}
&
\sphinxcode{\sphinxupquote{inline literal}}
\\
\hline
\begin{sphinxVerbatimintable}[commandchars=\\\{\}]
\PYG{n}{reference\PYGZus{}}
\end{sphinxVerbatimintable}
&
\sphinxhref{http://docutils.sourceforge.net/docs/user/rst/quickref.html\#hyperlink-targets}{reference}%
\begin{footnote}[14]\sphinxAtStartFootnote
\sphinxnolinkurl{http://docutils.sourceforge.net/docs/user/rst/quickref.html\#hyperlink-targets}
%
\end{footnote}
\\
\hline
\begin{sphinxVerbatimintable}[commandchars=\\\{\}]
`phrase reference`\PYGZus{}
\end{sphinxVerbatimintable}
&
\sphinxhref{http://docutils.sourceforge.net/docs/user/rst/quickref.html\#hyperlink-targets}{phrase reference}%
\begin{footnote}[15]\sphinxAtStartFootnote
\sphinxnolinkurl{http://docutils.sourceforge.net/docs/user/rst/quickref.html\#hyperlink-targets}
%
\end{footnote}
\\
\hline
\begin{sphinxVerbatimintable}[commandchars=\\\{\}]
\PYG{n}{anonymous\PYGZus{}\PYGZus{}}
\end{sphinxVerbatimintable}
&
\sphinxhref{http://docutils.sourceforge.net/docs/user/rst/quickref.html\#hyperlink-targets}{anonymous}%
\begin{footnote}[16]\sphinxAtStartFootnote
\sphinxnolinkurl{http://docutils.sourceforge.net/docs/user/rst/quickref.html\#hyperlink-targets}
%
\end{footnote}
\\
\hline
\begin{sphinxVerbatimintable}[commandchars=\\\{\}]
\PYGZus{}`inline internal target`
\end{sphinxVerbatimintable}
&
\phantomsection\label{\detokenize{rst-cheatsheet/rst-cheatsheet:inline-internal-target}}inline internal target
\\
\hline
\begin{sphinxVerbatimintable}[commandchars=\\\{\}]
\PYG{o}{|}\PYG{n}{substitution} \PYG{n}{reference}\PYG{o}{|}
\end{sphinxVerbatimintable}
&
The result is substituted in from the
substitution definition.
\\
\hline
\begin{sphinxVerbatimintable}[commandchars=\\\{\}]
\PYG{n}{footnote} \PYG{n}{reference} \PYG{p}{[}\PYG{l+m+mi}{1}\PYG{p}{]}\PYG{n}{\PYGZus{}}
\end{sphinxVerbatimintable}
&
footnote reference %
\begin{footnote}[17]\sphinxAtStartFootnote
This is the first one.
%
\end{footnote}
\\
\hline
\begin{sphinxVerbatimintable}[commandchars=\\\{\}]
\PYG{n}{citation} \PYG{n}{reference} \PYG{p}{[}\PYG{n}{CIT2002}\PYG{p}{]}\PYG{n}{\PYGZus{}}
\end{sphinxVerbatimintable}
&
citation reference \sphinxcite{rst-cheatsheet/rst-cheatsheet:cit2002}
\\
\hline
\begin{sphinxVerbatimintable}[commandchars=\\\{\}]
\PYG{n}{http}\PYG{p}{:}\PYG{o}{/}\PYG{o}{/}\PYG{n}{docutils}\PYG{o}{.}\PYG{n}{sf}\PYG{o}{.}\PYG{n}{net}\PYG{o}{/}
\end{sphinxVerbatimintable}
&
\sphinxurl{http://docutils.sf.net/}
\\
\hline
\end{tabular}
\par
\sphinxattableend\end{savenotes}


\section{Escaping with Backslashes}
\label{\detokenize{rst-cheatsheet/rst-cheatsheet:escaping-with-backslashes}}
reStructuredText uses backslashes (“") to override the special meaning given to markup characters and get
the literal characters themselves. To get a literal backslash, use an escaped backslash (“\textbackslash{}”). For example:


\begin{savenotes}\sphinxattablestart
\centering
\begin{tabular}[t]{|*{2}{\X{1}{2}|}}
\hline

\begin{sphinxVerbatimintable}[commandchars=\\\{\}]
*escape* ``with`` \PYGZdq{}\PYGZbs{}\PYGZdq{}
\end{sphinxVerbatimintable}
&
\sphinxstyleemphasis{escape} \sphinxcode{\sphinxupquote{with}} “”
\\
\hline
\begin{sphinxVerbatimintable}[commandchars=\\\{\}]
\PYGZbs{}*escape* \PYGZbs{}``with`` \PYGZdq{}\PYGZbs{}\PYGZbs{}\PYGZdq{}
\end{sphinxVerbatimintable}
&
*escape* ``with`` “"
\\
\hline
\end{tabular}
\par
\sphinxattableend\end{savenotes}


\section{Lists}
\label{\detokenize{rst-cheatsheet/rst-cheatsheet:lists}}

\begin{savenotes}\sphinxattablestart
\centering
\begin{tabular}[t]{|*{2}{\X{1}{2}|}}
\hline

\begin{sphinxVerbatimintable}[commandchars=\\\{\}]
\PYG{o}{\PYGZhy{}} \PYG{n}{This} \PYG{o+ow}{is} \PYG{n}{item} \PYG{l+m+mf}{1.} \PYG{n}{A} \PYG{n}{blank} \PYG{n}{line} \PYG{n}{before} \PYG{n}{the} \PYG{n}{first}
  \PYG{o+ow}{and} \PYG{n}{last} \PYG{n}{items} \PYG{o+ow}{is} \PYG{n}{required}\PYG{o}{.}
\PYG{o}{\PYGZhy{}} \PYG{n}{This} \PYG{o+ow}{is} \PYG{n}{item} \PYG{l+m+mi}{2}

\PYG{o}{\PYGZhy{}} \PYG{n}{Item} \PYG{l+m+mi}{3}\PYG{p}{:} \PYG{n}{blank} \PYG{n}{lines} \PYG{n}{between} \PYG{n}{items} \PYG{n}{are} \PYG{n}{optional}\PYG{o}{.}
\PYG{o}{\PYGZhy{}} \PYG{n}{Item} \PYG{l+m+mi}{4}\PYG{p}{:} \PYG{n}{Bullets} \PYG{n}{are} \PYG{l+s+s2}{\PYGZdq{}}\PYG{l+s+s2}{\PYGZhy{}}\PYG{l+s+s2}{\PYGZdq{}}\PYG{p}{,} \PYG{l+s+s2}{\PYGZdq{}}\PYG{l+s+s2}{*}\PYG{l+s+s2}{\PYGZdq{}} \PYG{o+ow}{or} \PYG{l+s+s2}{\PYGZdq{}}\PYG{l+s+s2}{+}\PYG{l+s+s2}{\PYGZdq{}}\PYG{o}{.}
  \PYG{n}{Continuing} \PYG{n}{text} \PYG{n}{must} \PYG{n}{be} \PYG{n}{aligned} \PYG{n}{after} \PYG{n}{the} \PYG{n}{bullet}
  \PYG{o+ow}{and} \PYG{n}{whitespace}\PYG{o}{.}
\PYG{o}{\PYGZhy{}} \PYG{n}{This} \PYG{n+nb}{list} \PYG{n}{item} \PYG{n}{contains} \PYG{n}{nested} \PYG{n}{items}

  \PYG{o}{\PYGZhy{}} \PYG{n}{Nested} \PYG{n}{items} \PYG{n}{must} \PYG{n}{be} \PYG{n}{indented} \PYG{n}{to} \PYG{n}{the} \PYG{n}{same}
    \PYG{n}{level}
\end{sphinxVerbatimintable}
&\begin{itemize}
\item {} 
This is item 1. A blank line before the first
and last items is required.

\item {} 
This is item 2

\item {} 
Item 3: blank lines between items are optional.

\item {} 
Item 4: Bullets are “\sphinxhyphen{}“, “*” or “+”.
Continuing text must be aligned after the bullet
and whitespace.

\item {} 
This list item contains nested items
\begin{itemize}
\item {} 
Nested items must be indented to the same
level

\end{itemize}

\end{itemize}
\\
\hline
\begin{sphinxVerbatimintable}[commandchars=\\\{\}]
\PYG{l+m+mf}{3.} \PYG{n}{This} \PYG{o+ow}{is} \PYG{n}{the} \PYG{n}{first} \PYG{n}{item}
\PYG{l+m+mf}{4.} \PYG{n}{This} \PYG{o+ow}{is} \PYG{n}{the} \PYG{n}{second} \PYG{n}{item}
\PYG{l+m+mf}{5.} \PYG{n}{Enumerators} \PYG{n}{are} \PYG{n}{arabic} \PYG{n}{numbers}\PYG{p}{,}
   \PYG{n}{single} \PYG{n}{letters}\PYG{p}{,} \PYG{o+ow}{or} \PYG{n}{roman} \PYG{n}{numerals}
\PYG{l+m+mf}{6.} \PYG{n}{List} \PYG{n}{items} \PYG{n}{should} \PYG{n}{be} \PYG{n}{sequentially}
   \PYG{n}{numbered}\PYG{p}{,} \PYG{n}{but} \PYG{n}{need} \PYG{o+ow}{not} \PYG{n}{start} \PYG{n}{at} \PYG{l+m+mi}{1}
   \PYG{p}{(}\PYG{n}{although} \PYG{o+ow}{not} \PYG{n+nb}{all} \PYG{n}{formatters} \PYG{n}{will}
   \PYG{n}{honour} \PYG{n}{the} \PYG{n}{first} \PYG{n}{index}\PYG{p}{)}\PYG{o}{.}
\PYG{c+c1}{\PYGZsh{}. This item is auto\PYGZhy{}enumerated}
\end{sphinxVerbatimintable}
&\begin{enumerate}
\sphinxsetlistlabels{\arabic}{enumi}{enumii}{}{.}%
\setcounter{enumi}{2}
\item {} 
This is the first item

\item {} 
This is the second item

\item {} 
Enumerators are arabic numbers,
single letters, or roman numerals

\item {} 
List items should be sequentially
numbered, but need not start at 1
(although not all formatters will
honour the first index).

\item {} 
This item is auto\sphinxhyphen{}enumerated

\end{enumerate}
\\
\hline
\begin{sphinxVerbatimintable}[commandchars=\\\{\}]
\PYG{n}{what}
  \PYG{n}{Definition} \PYG{n}{lists} \PYG{n}{associate} \PYG{n}{a} \PYG{n}{term} \PYG{k}{with}
  \PYG{n}{a} \PYG{n}{definition}\PYG{o}{.}

\PYG{n}{how}
  \PYG{n}{The} \PYG{n}{term} \PYG{o+ow}{is} \PYG{n}{a} \PYG{n}{one}\PYG{o}{\PYGZhy{}}\PYG{n}{line} \PYG{n}{phrase}\PYG{p}{,} \PYG{o+ow}{and} \PYG{n}{the}
  \PYG{n}{definition} \PYG{o+ow}{is} \PYG{n}{one} \PYG{o+ow}{or} \PYG{n}{more} \PYG{n}{paragraphs} \PYG{o+ow}{or}
  \PYG{n}{body} \PYG{n}{elements}\PYG{p}{,} \PYG{n}{indented} \PYG{n}{relative} \PYG{n}{to} \PYG{n}{the}
  \PYG{n}{term}\PYG{o}{.} \PYG{n}{Blank} \PYG{n}{lines} \PYG{n}{are} \PYG{o+ow}{not} \PYG{n}{allowed}
  \PYG{n}{between} \PYG{n}{term} \PYG{o+ow}{and} \PYG{n}{definition}\PYG{o}{.}
\end{sphinxVerbatimintable}
&\begin{description}
\item[{what}] \leavevmode
Definition lists associate a term with
a definition.

\item[{how}] \leavevmode
The term is a one\sphinxhyphen{}line phrase, and the
definition is one or more paragraphs or
body elements, indented relative to the
term. Blank lines are not allowed
between term and definition.

\end{description}
\\
\hline
\begin{sphinxVerbatimintable}[commandchars=\\\{\}]
\PYG{p}{:}\PYG{n}{Authors}\PYG{p}{:}
    \PYG{n}{Tony} \PYG{n}{J}\PYG{o}{.} \PYG{p}{(}\PYG{n}{Tibs}\PYG{p}{)} \PYG{n}{Ibbs}\PYG{p}{,}
    \PYG{n}{David} \PYG{n}{Goodger}

\PYG{p}{:}\PYG{n}{Version}\PYG{p}{:} \PYG{l+m+mf}{1.0} \PYG{n}{of} \PYG{l+m+mi}{2001}\PYG{o}{/}\PYG{l+m+mi}{08}\PYG{o}{/}\PYG{l+m+mi}{08}
\PYG{p}{:}\PYG{n}{Dedication}\PYG{p}{:} \PYG{n}{To} \PYG{n}{my} \PYG{n}{father}\PYG{o}{.}
\end{sphinxVerbatimintable}
&\begin{quote}\begin{description}
\item[{Authors}] \leavevmode
Tony J. (Tibs) Ibbs,
David Goodger

\item[{Version}] \leavevmode
1.0 of 2001/08/08

\item[{Dedication}] \leavevmode
To my father.

\end{description}\end{quote}
\\
\hline
\begin{sphinxVerbatimintable}[commandchars=\\\{\}]
\PYG{o}{\PYGZhy{}}\PYG{n}{a}            \PYG{n}{command}\PYG{o}{\PYGZhy{}}\PYG{n}{line} \PYG{n}{option} \PYG{l+s+s2}{\PYGZdq{}}\PYG{l+s+s2}{a}\PYG{l+s+s2}{\PYGZdq{}}
\PYG{o}{\PYGZhy{}}\PYG{n}{b} \PYG{n}{file}       \PYG{n}{options} \PYG{n}{can} \PYG{n}{have} \PYG{n}{arguments}
              \PYG{o+ow}{and} \PYG{n}{long} \PYG{n}{descriptions}
\PYG{o}{\PYGZhy{}}\PYG{o}{\PYGZhy{}}\PYG{n}{long}        \PYG{n}{options} \PYG{n}{can} \PYG{n}{be} \PYG{n}{long} \PYG{n}{also}
\PYG{o}{\PYGZhy{}}\PYG{o}{\PYGZhy{}}\PYG{n+nb}{input}\PYG{o}{=}\PYG{n}{file}  \PYG{n}{long} \PYG{n}{options} \PYG{n}{can} \PYG{n}{also} \PYG{n}{have}
              \PYG{n}{arguments}
\PYG{o}{/}\PYG{n}{V}            \PYG{n}{DOS}\PYG{o}{/}\PYG{n}{VMS}\PYG{o}{\PYGZhy{}}\PYG{n}{style} \PYG{n}{options} \PYG{n}{too}
\end{sphinxVerbatimintable}
&\begin{optionlist}{3cm}
\item [\sphinxhyphen{}a]  
command\sphinxhyphen{}line option “a”
\item [\sphinxhyphen{}b file]  
options can have arguments
and long descriptions
\item [\sphinxhyphen{}\sphinxhyphen{}long]  
options can be long also
\item [\sphinxhyphen{}\sphinxhyphen{}input=file]  
long options can also have
arguments
\item [/V]  
DOS/VMS\sphinxhyphen{}style options too
\end{optionlist}
\\
\hline
\end{tabular}
\par
\sphinxattableend\end{savenotes}




\section{Section Structure}
\label{\detokenize{rst-cheatsheet/rst-cheatsheet:section-structure}}

\begin{savenotes}\sphinxattablestart
\centering
\begin{tabular}[t]{|*{2}{\X{1}{2}|}}
\hline

\begin{sphinxVerbatimintable}[commandchars=\\\{\}]
Title
=====

Titles are underlined (or over\PYGZhy{} and underlined) with
a nonalphanumeric character at least as long as the
text.

A lone top\PYGZhy{}level section is lifted up to be the
document\PYGZsq{}s title.

Any non\PYGZhy{}alphanumeric character can be used, but
Python convention is:

* ``\PYGZsh{}`` with overline, for parts
* ``*`` with overline, for chapters
* ``=``, for sections
* ``\PYGZhy{}``, for subsections
* ``\PYGZca{}``, for subsubsections
* ``\PYGZdq{}``, for paragraphs
\end{sphinxVerbatimintable}
&
Title

Titles are underlined (or over\sphinxhyphen{} and underlined) with
a nonalphanumeric character at least as long as the
text.

A lone top\sphinxhyphen{}level section is lifted up to be the
document’s title.

Any non\sphinxhyphen{}alphanumeric character can be used, but
Python convention is:
\begin{itemize}
\item {} 
\sphinxcode{\sphinxupquote{\#}} with overline, for parts

\item {} 
\sphinxcode{\sphinxupquote{*}} with overline, for chapters

\item {} 
\sphinxcode{\sphinxupquote{=}}, for sections

\item {} 
\sphinxcode{\sphinxupquote{\sphinxhyphen{}}}, for subsections

\item {} 
\sphinxcode{\sphinxupquote{\textasciicircum{}}}, for subsubsections

\item {} 
\sphinxcode{\sphinxupquote{"}}, for paragraphs

\end{itemize}
\\
\hline
\end{tabular}
\par
\sphinxattableend\end{savenotes}


\section{Blocks}
\label{\detokenize{rst-cheatsheet/rst-cheatsheet:blocks}}

\begin{savenotes}\sphinxattablestart
\centering
\begin{tabular}[t]{|*{2}{\X{1}{2}|}}
\hline

\begin{sphinxVerbatimintable}[commandchars=\\\{\}]
\PYG{n}{This} \PYG{o+ow}{is} \PYG{n}{a} \PYG{n}{paragraph}\PYG{o}{.}

\PYG{n}{Paragraphs} \PYG{n}{line} \PYG{n}{up} \PYG{n}{at} \PYG{n}{their} \PYG{n}{left} \PYG{n}{edges}\PYG{p}{,} \PYG{o+ow}{and} \PYG{n}{are}
\PYG{n}{normally} \PYG{n}{separated} \PYG{n}{by} \PYG{n}{blank} \PYG{n}{lines}\PYG{o}{.}
\end{sphinxVerbatimintable}
&
This is a paragraph.

Paragraphs line up at their left
edges, and are normally separated
by blank lines.
\\
\hline
\begin{sphinxVerbatimintable}[commandchars=\\\{\}]
A paragraph containing only two colons indicates
the following indented or quoted text is a literal
block or quoted text is a literal block.

::

  Whitespace, newlines, blank lines, and  all kinds of
  markup (like *this* or \PYGZbs{}this) is preserved here.

You can also tack the ``::`` at the end of a
paragraph::

   It\PYGZsq{}s very convenient to use this form.

Per\PYGZhy{}line quoting can also be used for unindented
blocks::

\PYGZgt{} Useful for quotes from email and
\PYGZgt{} for Haskell literate programming.
\end{sphinxVerbatimintable}
&
A paragraph containing only two colons
indicates that the following indented
or quoted text is a literal block.

\begin{sphinxVerbatimintable}[commandchars=\\\{\}]
\PYG{n}{Whitespace}\PYG{p}{,} \PYG{n}{newlines}\PYG{p}{,} \PYG{n}{blank} \PYG{n}{lines}\PYG{p}{,} \PYG{o+ow}{and}
\PYG{n+nb}{all} \PYG{n}{kinds} \PYG{n}{of} \PYG{n}{markup} \PYG{p}{(}\PYG{n}{like} \PYG{o}{*}\PYG{n}{this}\PYG{o}{*} \PYG{o+ow}{or}
\PYGZbs{}\PYG{n}{this}\PYG{p}{)} \PYG{o+ow}{is} \PYG{n}{preserved} \PYG{n}{by} \PYG{n}{literal} \PYG{n}{blocks}\PYG{o}{.}
\end{sphinxVerbatimintable}

You can also tack the \sphinxcode{\sphinxupquote{::}} at the end of a
paragraph:

\begin{sphinxVerbatimintable}[commandchars=\\\{\}]
\PYG{n}{It}\PYG{l+s+s1}{\PYGZsq{}}\PYG{l+s+s1}{s very convenient to use this form.}
\end{sphinxVerbatimintable}

Per\sphinxhyphen{}line quoting can also be used for
unindented blocks:

\begin{sphinxVerbatimintable}[commandchars=\\\{\}]
\PYG{o}{\PYGZgt{}} \PYG{n}{Useful} \PYG{k}{for} \PYG{n}{quotes} \PYG{k+kn}{from} \PYG{n+nn}{email} \PYG{o+ow}{and}
\PYG{o}{\PYGZgt{}} \PYG{k}{for} \PYG{n}{Haskell} \PYG{n}{literate} \PYG{n}{programming}\PYG{o}{.}
\end{sphinxVerbatimintable}
\\
\hline
\begin{sphinxVerbatimintable}[commandchars=\\\{\}]
\PYG{o}{|} \PYG{n}{Line} \PYG{n}{blocks} \PYG{n}{are} \PYG{n}{useful} \PYG{k}{for} \PYG{n}{addresses}\PYG{p}{,}
\PYG{o}{|} \PYG{n}{verse}\PYG{p}{,} \PYG{o+ow}{and} \PYG{n}{adornment}\PYG{o}{\PYGZhy{}}\PYG{n}{free} \PYG{n}{lists}\PYG{o}{.}
\PYG{o}{|}
\PYG{o}{|} \PYG{n}{Each} \PYG{n}{new} \PYG{n}{line} \PYG{n}{begins} \PYG{k}{with} \PYG{n}{a}
\PYG{o}{|} \PYG{n}{vertical} \PYG{n}{bar} \PYG{p}{(}\PYG{l+s+s2}{\PYGZdq{}}\PYG{l+s+s2}{|}\PYG{l+s+s2}{\PYGZdq{}}\PYG{p}{)}\PYG{o}{.}
\PYG{o}{|}     \PYG{n}{Line} \PYG{n}{breaks} \PYG{o+ow}{and} \PYG{n}{initial} \PYG{n}{indents}
\PYG{o}{|}     \PYG{n}{are} \PYG{n}{preserved}\PYG{o}{.}
\PYG{o}{|} \PYG{n}{Continuation} \PYG{n}{lines} \PYG{n}{are} \PYG{n}{wrapped}
  \PYG{n}{portions} \PYG{n}{of} \PYG{n}{long} \PYG{n}{lines}\PYG{p}{;} \PYG{n}{they} \PYG{n}{begin}
  \PYG{k}{with} \PYG{n}{spaces} \PYG{o+ow}{in} \PYG{n}{place} \PYG{n}{of} \PYG{n}{vertical} \PYG{n}{bars}\PYG{o}{.}
\end{sphinxVerbatimintable}
&
\begin{DUlineblock}{0em}
\item[] Line blocks are useful for addresses,
\item[] verse, and adornment\sphinxhyphen{}free lists.
\item[] 
\item[] Each new line begins with a
\item[] vertical bar (“|”).
\item[]
\begin{DUlineblock}{\DUlineblockindent}
\item[] Line breaks and initial indents
\item[] are preserved.
\end{DUlineblock}
\item[] Continuation lines are wrapped
portions of long lines; they begin
with spaces in place of vertical bars.
\end{DUlineblock}
\\
\hline
\begin{sphinxVerbatimintable}[commandchars=\\\{\}]
\PYG{n}{Block} \PYG{n}{quotes} \PYG{n}{are} \PYG{n}{just}\PYG{p}{:}

    \PYG{n}{Indented} \PYG{n}{paragraphs}\PYG{p}{,}

        \PYG{o+ow}{and} \PYG{n}{they} \PYG{n}{may} \PYG{n}{nest}\PYG{o}{.}
\end{sphinxVerbatimintable}
&
Block quotes are just:
\begin{quote}

Indented paragraphs,
\begin{quote}

and they may nest.
\end{quote}
\end{quote}
\\
\hline
\begin{sphinxVerbatimintable}[commandchars=\\\{\}]
\PYG{n}{Doctest} \PYG{n}{blocks} \PYG{n}{are} \PYG{n}{interactive}
\PYG{n}{Python} \PYG{n}{sessions}\PYG{o}{.} \PYG{n}{They} \PYG{n}{begin} \PYG{k}{with}
\PYG{l+s+s2}{\PYGZdq{}}\PYG{l+s+s2}{``\PYGZgt{}\PYGZgt{}\PYGZgt{}``}\PYG{l+s+s2}{\PYGZdq{}} \PYG{o+ow}{and} \PYG{n}{end} \PYG{k}{with} \PYG{n}{a} \PYG{n}{blank} \PYG{n}{line}\PYG{o}{.}

\PYG{o}{\PYGZgt{}\PYGZgt{}}\PYG{o}{\PYGZgt{}} \PYG{n+nb}{print} \PYG{l+s+s2}{\PYGZdq{}}\PYG{l+s+s2}{This is a doctest block.}\PYG{l+s+s2}{\PYGZdq{}}
\PYG{n}{This} \PYG{o+ow}{is} \PYG{n}{a} \PYG{n}{doctest} \PYG{n}{block}\PYG{o}{.}
\end{sphinxVerbatimintable}
&
Doctest blocks are interactive
Python sessions. They begin with
“\sphinxcode{\sphinxupquote{>>>}}” and end with a blank line.

\begin{sphinxVerbatimintable}[commandchars=\\\{\}]
\PYG{g+gp}{\PYGZgt{}\PYGZgt{}\PYGZgt{} }\PYG{n+nb}{print} \PYG{l+s+s2}{\PYGZdq{}}\PYG{l+s+s2}{This is a doctest block.}\PYG{l+s+s2}{\PYGZdq{}}
\PYG{g+go}{This is a doctest block.}
\end{sphinxVerbatimintable}
\\
\hline
\begin{sphinxVerbatimintable}[commandchars=\\\{\}]
\PYG{n}{A} \PYG{n}{transition} \PYG{n}{marker} \PYG{o+ow}{is} \PYG{n}{a} \PYG{n}{horizontal} \PYG{n}{line}
\PYG{n}{of} \PYG{l+m+mi}{4} \PYG{o+ow}{or} \PYG{n}{more} \PYG{n}{repeated} \PYG{n}{punctuation}
\PYG{n}{characters}\PYG{o}{.}

\PYG{o}{\PYGZhy{}}\PYG{o}{\PYGZhy{}}\PYG{o}{\PYGZhy{}}\PYG{o}{\PYGZhy{}}\PYG{o}{\PYGZhy{}}\PYG{o}{\PYGZhy{}}\PYG{o}{\PYGZhy{}}\PYG{o}{\PYGZhy{}}\PYG{o}{\PYGZhy{}}\PYG{o}{\PYGZhy{}}\PYG{o}{\PYGZhy{}}\PYG{o}{\PYGZhy{}}

\PYG{n}{A} \PYG{n}{transition} \PYG{n}{should} \PYG{o+ow}{not} \PYG{n}{begin} \PYG{o+ow}{or} \PYG{n}{end} \PYG{n}{a}
\PYG{n}{section} \PYG{o+ow}{or} \PYG{n}{document}\PYG{p}{,} \PYG{n}{nor} \PYG{n}{should} \PYG{n}{two}
\PYG{n}{transitions} \PYG{n}{be} \PYG{n}{immediately} \PYG{n}{adjacent}\PYG{o}{.}
\end{sphinxVerbatimintable}
&
A transition marker is a horizontal line
of 4 or more repeated punctuation
characters.


\begin{savenotes}\sphinxattablestart
\centering
\begin{tabulary}{\linewidth}[t]{|T|}
\hline
\\
\hline
\end{tabulary}
\par
\sphinxattableend\end{savenotes}

A transition should not begin or end a
section or document, nor should two
transitions be immediately adjacent.
\\
\hline
\end{tabular}
\par
\sphinxattableend\end{savenotes}




\section{Tables}
\label{\detokenize{rst-cheatsheet/rst-cheatsheet:tables}}
There are two syntaxes for tables in reStructuredText. Grid tables are complete but cumbersome to create. Simple
tables are easy to create but limited (no row spans, etc.).


\begin{savenotes}\sphinxattablestart
\centering
\begin{tabular}[t]{|*{2}{\X{1}{2}|}}
\hline

\begin{sphinxVerbatimintable}[commandchars=\\\{\}]
\PYG{o}{+}\PYG{o}{\PYGZhy{}}\PYG{o}{\PYGZhy{}}\PYG{o}{\PYGZhy{}}\PYG{o}{\PYGZhy{}}\PYG{o}{\PYGZhy{}}\PYG{o}{\PYGZhy{}}\PYG{o}{\PYGZhy{}}\PYG{o}{\PYGZhy{}}\PYG{o}{\PYGZhy{}}\PYG{o}{\PYGZhy{}}\PYG{o}{\PYGZhy{}}\PYG{o}{\PYGZhy{}}\PYG{o}{+}\PYG{o}{\PYGZhy{}}\PYG{o}{\PYGZhy{}}\PYG{o}{\PYGZhy{}}\PYG{o}{\PYGZhy{}}\PYG{o}{\PYGZhy{}}\PYG{o}{\PYGZhy{}}\PYG{o}{\PYGZhy{}}\PYG{o}{\PYGZhy{}}\PYG{o}{\PYGZhy{}}\PYG{o}{\PYGZhy{}}\PYG{o}{\PYGZhy{}}\PYG{o}{\PYGZhy{}}\PYG{o}{+}\PYG{o}{\PYGZhy{}}\PYG{o}{\PYGZhy{}}\PYG{o}{\PYGZhy{}}\PYG{o}{\PYGZhy{}}\PYG{o}{\PYGZhy{}}\PYG{o}{\PYGZhy{}}\PYG{o}{\PYGZhy{}}\PYG{o}{\PYGZhy{}}\PYG{o}{\PYGZhy{}}\PYG{o}{\PYGZhy{}}\PYG{o}{\PYGZhy{}}\PYG{o}{+}
\PYG{o}{|} \PYG{n}{Header} \PYG{l+m+mi}{1}   \PYG{o}{|} \PYG{n}{Header} \PYG{l+m+mi}{2}   \PYG{o}{|} \PYG{n}{Header} \PYG{l+m+mi}{3}  \PYG{o}{|}
\PYG{o}{+}\PYG{o}{==}\PYG{o}{==}\PYG{o}{==}\PYG{o}{==}\PYG{o}{==}\PYG{o}{==}\PYG{o}{+}\PYG{o}{==}\PYG{o}{==}\PYG{o}{==}\PYG{o}{==}\PYG{o}{==}\PYG{o}{==}\PYG{o}{+}\PYG{o}{==}\PYG{o}{==}\PYG{o}{==}\PYG{o}{==}\PYG{o}{==}\PYG{o}{=}\PYG{o}{+}
\PYG{o}{|} \PYG{n}{body} \PYG{n}{row} \PYG{l+m+mi}{1} \PYG{o}{|} \PYG{n}{column} \PYG{l+m+mi}{2}   \PYG{o}{|} \PYG{n}{column} \PYG{l+m+mi}{3}  \PYG{o}{|}
\PYG{o}{+}\PYG{o}{\PYGZhy{}}\PYG{o}{\PYGZhy{}}\PYG{o}{\PYGZhy{}}\PYG{o}{\PYGZhy{}}\PYG{o}{\PYGZhy{}}\PYG{o}{\PYGZhy{}}\PYG{o}{\PYGZhy{}}\PYG{o}{\PYGZhy{}}\PYG{o}{\PYGZhy{}}\PYG{o}{\PYGZhy{}}\PYG{o}{\PYGZhy{}}\PYG{o}{\PYGZhy{}}\PYG{o}{+}\PYG{o}{\PYGZhy{}}\PYG{o}{\PYGZhy{}}\PYG{o}{\PYGZhy{}}\PYG{o}{\PYGZhy{}}\PYG{o}{\PYGZhy{}}\PYG{o}{\PYGZhy{}}\PYG{o}{\PYGZhy{}}\PYG{o}{\PYGZhy{}}\PYG{o}{\PYGZhy{}}\PYG{o}{\PYGZhy{}}\PYG{o}{\PYGZhy{}}\PYG{o}{\PYGZhy{}}\PYG{o}{+}\PYG{o}{\PYGZhy{}}\PYG{o}{\PYGZhy{}}\PYG{o}{\PYGZhy{}}\PYG{o}{\PYGZhy{}}\PYG{o}{\PYGZhy{}}\PYG{o}{\PYGZhy{}}\PYG{o}{\PYGZhy{}}\PYG{o}{\PYGZhy{}}\PYG{o}{\PYGZhy{}}\PYG{o}{\PYGZhy{}}\PYG{o}{\PYGZhy{}}\PYG{o}{+}
\PYG{o}{|} \PYG{n}{body} \PYG{n}{row} \PYG{l+m+mi}{2} \PYG{o}{|} \PYG{n}{Cells} \PYG{n}{may} \PYG{n}{span} \PYG{n}{columns}\PYG{o}{.}\PYG{o}{|}
\PYG{o}{+}\PYG{o}{\PYGZhy{}}\PYG{o}{\PYGZhy{}}\PYG{o}{\PYGZhy{}}\PYG{o}{\PYGZhy{}}\PYG{o}{\PYGZhy{}}\PYG{o}{\PYGZhy{}}\PYG{o}{\PYGZhy{}}\PYG{o}{\PYGZhy{}}\PYG{o}{\PYGZhy{}}\PYG{o}{\PYGZhy{}}\PYG{o}{\PYGZhy{}}\PYG{o}{\PYGZhy{}}\PYG{o}{+}\PYG{o}{\PYGZhy{}}\PYG{o}{\PYGZhy{}}\PYG{o}{\PYGZhy{}}\PYG{o}{\PYGZhy{}}\PYG{o}{\PYGZhy{}}\PYG{o}{\PYGZhy{}}\PYG{o}{\PYGZhy{}}\PYG{o}{\PYGZhy{}}\PYG{o}{\PYGZhy{}}\PYG{o}{\PYGZhy{}}\PYG{o}{\PYGZhy{}}\PYG{o}{\PYGZhy{}}\PYG{o}{+}\PYG{o}{\PYGZhy{}}\PYG{o}{\PYGZhy{}}\PYG{o}{\PYGZhy{}}\PYG{o}{\PYGZhy{}}\PYG{o}{\PYGZhy{}}\PYG{o}{\PYGZhy{}}\PYG{o}{\PYGZhy{}}\PYG{o}{\PYGZhy{}}\PYG{o}{\PYGZhy{}}\PYG{o}{\PYGZhy{}}\PYG{o}{\PYGZhy{}}\PYG{o}{+}
\PYG{o}{|} \PYG{n}{body} \PYG{n}{row} \PYG{l+m+mi}{3} \PYG{o}{|} \PYG{n}{Cells} \PYG{n}{may}  \PYG{o}{|} \PYG{o}{\PYGZhy{}} \PYG{n}{Cells}   \PYG{o}{|}
\PYG{o}{+}\PYG{o}{\PYGZhy{}}\PYG{o}{\PYGZhy{}}\PYG{o}{\PYGZhy{}}\PYG{o}{\PYGZhy{}}\PYG{o}{\PYGZhy{}}\PYG{o}{\PYGZhy{}}\PYG{o}{\PYGZhy{}}\PYG{o}{\PYGZhy{}}\PYG{o}{\PYGZhy{}}\PYG{o}{\PYGZhy{}}\PYG{o}{\PYGZhy{}}\PYG{o}{\PYGZhy{}}\PYG{o}{+} \PYG{n}{span} \PYG{n}{rows}\PYG{o}{.} \PYG{o}{|} \PYG{o}{\PYGZhy{}} \PYG{n}{contain} \PYG{o}{|}
\PYG{o}{|} \PYG{n}{body} \PYG{n}{row} \PYG{l+m+mi}{4} \PYG{o}{|}            \PYG{o}{|} \PYG{o}{\PYGZhy{}} \PYG{n}{blocks}\PYG{o}{.} \PYG{o}{|}
\PYG{o}{+}\PYG{o}{\PYGZhy{}}\PYG{o}{\PYGZhy{}}\PYG{o}{\PYGZhy{}}\PYG{o}{\PYGZhy{}}\PYG{o}{\PYGZhy{}}\PYG{o}{\PYGZhy{}}\PYG{o}{\PYGZhy{}}\PYG{o}{\PYGZhy{}}\PYG{o}{\PYGZhy{}}\PYG{o}{\PYGZhy{}}\PYG{o}{\PYGZhy{}}\PYG{o}{\PYGZhy{}}\PYG{o}{+}\PYG{o}{\PYGZhy{}}\PYG{o}{\PYGZhy{}}\PYG{o}{\PYGZhy{}}\PYG{o}{\PYGZhy{}}\PYG{o}{\PYGZhy{}}\PYG{o}{\PYGZhy{}}\PYG{o}{\PYGZhy{}}\PYG{o}{\PYGZhy{}}\PYG{o}{\PYGZhy{}}\PYG{o}{\PYGZhy{}}\PYG{o}{\PYGZhy{}}\PYG{o}{\PYGZhy{}}\PYG{o}{+}\PYG{o}{\PYGZhy{}}\PYG{o}{\PYGZhy{}}\PYG{o}{\PYGZhy{}}\PYG{o}{\PYGZhy{}}\PYG{o}{\PYGZhy{}}\PYG{o}{\PYGZhy{}}\PYG{o}{\PYGZhy{}}\PYG{o}{\PYGZhy{}}\PYG{o}{\PYGZhy{}}\PYG{o}{\PYGZhy{}}\PYG{o}{\PYGZhy{}}\PYG{o}{+}
\end{sphinxVerbatimintable}
&

\begin{savenotes}\sphinxattablestart
\centering
\begin{tabular}[t]{|*{3}{\X{1}{3}|}}
\hline
\sphinxstyletheadfamily 
Header 1
&\sphinxstyletheadfamily 
Header 2
&\sphinxstyletheadfamily 
Header 3
\\
\hline
body row 1
&
column 2
&
column 3
\\
\hline
body row 2
&\sphinxstartmulticolumn{2}%
\begin{varwidth}[t]{\sphinxcolwidth{2}{3}}
Cells may span columns.
\par
\vskip-\baselineskip\vbox{\hbox{\strut}}\end{varwidth}%
\sphinxstopmulticolumn
\\
\hline
body row 3
&\sphinxmultirow{2}{10}{%
\begin{varwidth}[t]{\sphinxcolwidth{1}{3}}
Cells may
span rows.
\par
\vskip-\baselineskip\vbox{\hbox{\strut}}\end{varwidth}%
}%
&\sphinxmultirow{2}{11}{%
\begin{varwidth}[t]{\sphinxcolwidth{1}{3}}
\begin{itemize}
\item {} 
Cells

\item {} 
contain

\item {} 
blocks.

\end{itemize}
\par
\vskip-\baselineskip\vbox{\hbox{\strut}}\end{varwidth}%
}%
\\
\cline{1-1}
body row 4
&\sphinxtablestrut{10}&\sphinxtablestrut{11}\\
\hline
\end{tabular}
\par
\sphinxattableend\end{savenotes}
\\
\hline
\begin{sphinxVerbatimintable}[commandchars=\\\{\}]
\PYG{o}{==}\PYG{o}{==}\PYG{o}{=}  \PYG{o}{==}\PYG{o}{==}\PYG{o}{=}  \PYG{o}{==}\PYG{o}{==}\PYG{o}{==}
   \PYG{n}{Inputs}     \PYG{n}{Output}
\PYG{o}{\PYGZhy{}}\PYG{o}{\PYGZhy{}}\PYG{o}{\PYGZhy{}}\PYG{o}{\PYGZhy{}}\PYG{o}{\PYGZhy{}}\PYG{o}{\PYGZhy{}}\PYG{o}{\PYGZhy{}}\PYG{o}{\PYGZhy{}}\PYG{o}{\PYGZhy{}}\PYG{o}{\PYGZhy{}}\PYG{o}{\PYGZhy{}}\PYG{o}{\PYGZhy{}}  \PYG{o}{\PYGZhy{}}\PYG{o}{\PYGZhy{}}\PYG{o}{\PYGZhy{}}\PYG{o}{\PYGZhy{}}\PYG{o}{\PYGZhy{}}\PYG{o}{\PYGZhy{}}
  \PYG{n}{A}      \PYG{n}{B}    \PYG{n}{A} \PYG{o+ow}{or} \PYG{n}{B}
\PYG{o}{==}\PYG{o}{==}\PYG{o}{=}  \PYG{o}{==}\PYG{o}{==}\PYG{o}{=}  \PYG{o}{==}\PYG{o}{==}\PYG{o}{==}
\PYG{k+kc}{False}  \PYG{k+kc}{False}  \PYG{k+kc}{False}
\PYG{k+kc}{True}   \PYG{k+kc}{False}  \PYG{k+kc}{True}
\PYG{k+kc}{False}  \PYG{k+kc}{True}   \PYG{k+kc}{True}
\PYG{k+kc}{True}   \PYG{k+kc}{True}   \PYG{k+kc}{True}
\PYG{o}{==}\PYG{o}{==}\PYG{o}{=}  \PYG{o}{==}\PYG{o}{==}\PYG{o}{=}  \PYG{o}{==}\PYG{o}{==}\PYG{o}{==}
\end{sphinxVerbatimintable}
&

\begin{savenotes}\sphinxattablestart
\centering
\begin{tabulary}{\linewidth}[t]{|T|T|T|}
\hline
\sphinxstartmulticolumn{2}%
\begin{varwidth}[t]{\sphinxcolwidth{2}{3}}
\sphinxstyletheadfamily Inputs
\par
\vskip-\baselineskip\vbox{\hbox{\strut}}\end{varwidth}%
\sphinxstopmulticolumn
&\sphinxstyletheadfamily 
Output
\\
\hline\sphinxstyletheadfamily 
A
&\sphinxstyletheadfamily 
B
&\sphinxstyletheadfamily 
A or B
\\
\hline
False
&
False
&
False
\\
\hline
True
&
False
&
True
\\
\hline
False
&
True
&
True
\\
\hline
True
&
True
&
True
\\
\hline
\end{tabulary}
\par
\sphinxattableend\end{savenotes}
\\
\hline
\end{tabular}
\par
\sphinxattableend\end{savenotes}


\section{Explicit Markup}
\label{\detokenize{rst-cheatsheet/rst-cheatsheet:explicit-markup}}
Explicit markup blocks are used for constructs which float (footnotes), have no direct paper\sphinxhyphen{}document representation
(hyperlink targets, comments), or require specialized processing (directives).
They all begin with two periods and whitespace, the “explicit markup start”.


\begin{savenotes}\sphinxattablestart
\centering
\begin{tabular}[t]{|*{2}{\X{1}{2}|}}
\hline

\begin{sphinxVerbatimintable}[commandchars=\\\{\}]
\PYG{n}{Footnote} \PYG{n}{references}\PYG{p}{,} \PYG{n}{like} \PYG{p}{[}\PYG{l+m+mi}{5}\PYG{p}{]}\PYG{n}{\PYGZus{}}\PYG{o}{.}
\PYG{n}{Note} \PYG{n}{that} \PYG{n}{footnotes} \PYG{n}{may} \PYG{n}{get}
\PYG{n}{rearranged}\PYG{p}{,} \PYG{n}{e}\PYG{o}{.}\PYG{n}{g}\PYG{o}{.}\PYG{p}{,} \PYG{n}{to} \PYG{n}{the} \PYG{n}{bottom} \PYG{n}{of}
\PYG{n}{the} \PYG{l+s+s2}{\PYGZdq{}}\PYG{l+s+s2}{page}\PYG{l+s+s2}{\PYGZdq{}}\PYG{o}{.}

\PYG{o}{.}\PYG{o}{.} \PYG{p}{[}\PYG{l+m+mi}{5}\PYG{p}{]} \PYG{n}{A} \PYG{n}{numerical} \PYG{n}{footnote}\PYG{o}{.} \PYG{n}{Note}
   \PYG{n}{there}\PYG{l+s+s1}{\PYGZsq{}}\PYG{l+s+s1}{s no colon after the ``]``.}
\end{sphinxVerbatimintable}
&
Footnote references, like %
\begin{footnote}[5]\sphinxAtStartFootnote
A numerical footnote. Note
there’s no colon after the \sphinxcode{\sphinxupquote{{]}}}.
%
\end{footnote}.
Note that footnotes may get
rearranged, e.g., to the bottom of
the “page”.
\\
\hline
\begin{sphinxVerbatimintable}[commandchars=\\\{\}]
\PYG{n}{Autonumbered} \PYG{n}{footnotes} \PYG{n}{are}
\PYG{n}{possible}\PYG{p}{,} \PYG{n}{like} \PYG{n}{using} \PYG{p}{[}\PYG{c+c1}{\PYGZsh{}]\PYGZus{} and [\PYGZsh{}]\PYGZus{}.}

\PYG{o}{.}\PYG{o}{.} \PYG{p}{[}\PYG{c+c1}{\PYGZsh{}] This is the first one.}
\PYG{o}{.}\PYG{o}{.} \PYG{p}{[}\PYG{c+c1}{\PYGZsh{}] This is the second one.}

\PYG{n}{They} \PYG{n}{may} \PYG{n}{be} \PYG{n}{assigned} \PYG{l+s+s1}{\PYGZsq{}}\PYG{l+s+s1}{autonumber}
\PYG{n}{labels}\PYG{l+s+s1}{\PYGZsq{}}\PYG{l+s+s1}{ \PYGZhy{} for instance,}
\PYG{p}{[}\PYG{c+c1}{\PYGZsh{}fourth]\PYGZus{} and [\PYGZsh{}third]\PYGZus{}.}

\PYG{o}{.}\PYG{o}{.} \PYG{p}{[}\PYG{c+c1}{\PYGZsh{}third] a.k.a. third\PYGZus{}}

\PYG{o}{.}\PYG{o}{.} \PYG{p}{[}\PYG{c+c1}{\PYGZsh{}fourth] a.k.a. fourth\PYGZus{}}
\end{sphinxVerbatimintable}
&
Autonumbered footnotes are
possible, like using \sphinxfootnotemark[17] and %
\begin{footnote}[18]\sphinxAtStartFootnote
This is the second one.
%
\end{footnote}.

They may be assigned ‘autonumber
labels’ \sphinxhyphen{} for instance,
%
\begin{footnote}[20]\sphinxAtStartFootnote
a.k.a. {\hyperref[\detokenize{rst-cheatsheet/rst-cheatsheet:fourth}]{\sphinxcrossref{fourth}}}
%
\end{footnote} and %
\begin{footnote}[19]\sphinxAtStartFootnote
a.k.a. {\hyperref[\detokenize{rst-cheatsheet/rst-cheatsheet:third}]{\sphinxcrossref{third}}}
%
\end{footnote}.
\\
\hline
\begin{sphinxVerbatimintable}[commandchars=\\\{\}]
\PYG{n}{Auto}\PYG{o}{\PYGZhy{}}\PYG{n}{symbol} \PYG{n}{footnotes} \PYG{n}{are} \PYG{n}{also}
\PYG{n}{possible}\PYG{p}{,} \PYG{n}{like} \PYG{n}{this}\PYG{p}{:} \PYG{p}{[}\PYG{o}{*}\PYG{p}{]}\PYG{n}{\PYGZus{}} \PYG{o+ow}{and} \PYG{p}{[}\PYG{o}{*}\PYG{p}{]}\PYG{n}{\PYGZus{}}\PYG{o}{.}

\PYG{o}{.}\PYG{o}{.} \PYG{p}{[}\PYG{o}{*}\PYG{p}{]} \PYG{n}{This} \PYG{o+ow}{is} \PYG{n}{the} \PYG{n}{first} \PYG{n}{one}\PYG{o}{.}
\PYG{o}{.}\PYG{o}{.} \PYG{p}{[}\PYG{o}{*}\PYG{p}{]} \PYG{n}{This} \PYG{o+ow}{is} \PYG{n}{the} \PYG{n}{second} \PYG{n}{one}\PYG{o}{.}
\end{sphinxVerbatimintable}
&
Auto\sphinxhyphen{}symbol footnotes are also
possible, like this: %
\begin{footnote}[21]\sphinxAtStartFootnote
This is the first one.
%
\end{footnote} and %
\begin{footnote}[22]\sphinxAtStartFootnote
This is the second one.
%
\end{footnote}.
\\
\hline
\begin{sphinxVerbatimintable}[commandchars=\\\{\}]
\PYG{n}{Citation} \PYG{n}{references}\PYG{p}{,} \PYG{n}{like} \PYG{p}{[}\PYG{n}{CIT2002}\PYG{p}{]}\PYG{n}{\PYGZus{}}\PYG{o}{.}
\PYG{n}{Note} \PYG{n}{that} \PYG{n}{citations} \PYG{n}{may} \PYG{n}{get}
\PYG{n}{rearranged}\PYG{p}{,} \PYG{n}{e}\PYG{o}{.}\PYG{n}{g}\PYG{o}{.}\PYG{p}{,} \PYG{n}{to} \PYG{n}{the} \PYG{n}{bottom} \PYG{n}{of}
\PYG{n}{the} \PYG{l+s+s2}{\PYGZdq{}}\PYG{l+s+s2}{page}\PYG{l+s+s2}{\PYGZdq{}}\PYG{o}{.}

\PYG{o}{.}\PYG{o}{.} \PYG{p}{[}\PYG{n}{CIT2002}\PYG{p}{]} \PYG{n}{A} \PYG{n}{citation}
   \PYG{p}{(}\PYG{k}{as} \PYG{n}{often} \PYG{n}{used} \PYG{o+ow}{in} \PYG{n}{journals}\PYG{p}{)}\PYG{o}{.}

\PYG{n}{Citation} \PYG{n}{labels} \PYG{n}{contain} \PYG{n}{alphanumerics}\PYG{p}{,}
\PYG{n}{underlines}\PYG{p}{,} \PYG{n}{hyphens} \PYG{o+ow}{and} \PYG{n}{fullstops}\PYG{o}{.}
\PYG{n}{Case} \PYG{o+ow}{is} \PYG{o+ow}{not} \PYG{n}{significant}\PYG{o}{.}

\PYG{n}{Given} \PYG{n}{a} \PYG{n}{citation} \PYG{n}{like} \PYG{p}{[}\PYG{n}{this}\PYG{p}{]}\PYG{n}{\PYGZus{}}\PYG{p}{,} \PYG{n}{one}
\PYG{n}{can} \PYG{n}{also} \PYG{n}{refer} \PYG{n}{to} \PYG{n}{it} \PYG{n}{like} \PYG{n}{this\PYGZus{}}\PYG{o}{.}

\PYG{o}{.}\PYG{o}{.} \PYG{p}{[}\PYG{n}{this}\PYG{p}{]} \PYG{n}{here}\PYG{o}{.}
\end{sphinxVerbatimintable}
&
Citation references, like \sphinxcite{rst-cheatsheet/rst-cheatsheet:cit2002}.
Note that citations may get
rearranged, e.g., to the bottom of
the “page”.

Citation labels contain alphanumerics,
underlines, hyphens and fullstops.
Case is not significant.

Given a citation like \sphinxcite{rst-cheatsheet/rst-cheatsheet:this}, one
can also refer to it like {\hyperref[\detokenize{rst-cheatsheet/rst-cheatsheet:this}]{\sphinxcrossref{this}}}.
\\
\hline
\begin{sphinxVerbatimintable}[commandchars=\\\{\}]
\PYG{n}{External} \PYG{n}{hyperlinks}\PYG{p}{,} \PYG{n}{like} \PYG{n}{Python\PYGZus{}}\PYG{o}{.}

\PYG{o}{.}\PYG{o}{.} \PYG{n}{\PYGZus{}Python}\PYG{p}{:} \PYG{n}{http}\PYG{p}{:}\PYG{o}{/}\PYG{o}{/}\PYG{n}{www}\PYG{o}{.}\PYG{n}{python}\PYG{o}{.}\PYG{n}{org}\PYG{o}{/}
\end{sphinxVerbatimintable}
&
External hyperlinks, like \sphinxhref{http://www.python.org/}{Python}%
\begin{footnote}[23]\sphinxAtStartFootnote
\sphinxnolinkurl{http://www.python.org/}
%
\end{footnote}.
\\
\hline
\begin{sphinxVerbatimintable}[commandchars=\\\{\}]
External hyperlinks, like `Python
\PYGZlt{}http://www.python.org/\PYGZgt{}`\PYGZus{}.
\end{sphinxVerbatimintable}
&
External hyperlinks, like \sphinxhref{http://www.python.org/}{Python}%
\begin{footnote}[24]\sphinxAtStartFootnote
\sphinxnolinkurl{http://www.python.org/}
%
\end{footnote}.
\\
\hline
\begin{sphinxVerbatimintable}[commandchars=\\\{\}]
\PYG{n}{Internal} \PYG{n}{crossreferences}\PYG{p}{,} \PYG{n}{like} \PYG{n}{example\PYGZus{}}\PYG{o}{.}

\PYG{o}{.}\PYG{o}{.} \PYG{n}{\PYGZus{}example}\PYG{p}{:}

\PYG{n}{This} \PYG{o+ow}{is} \PYG{n}{an} \PYG{n}{example} \PYG{n}{crossreference} \PYG{n}{target}\PYG{o}{.}
\end{sphinxVerbatimintable}
&
Internal crossreferences, like {\hyperref[\detokenize{rst-cheatsheet/rst-cheatsheet:example}]{\sphinxcrossref{example}}}.

\phantomsection\label{\detokenize{rst-cheatsheet/rst-cheatsheet:example}}
This is an example crossreference target.
\\
\hline
\begin{sphinxVerbatimintable}[commandchars=\\\{\}]
Python\PYGZus{} is `my favourite
programming language`\PYGZus{}\PYGZus{}.

.. \PYGZus{}Python: http://www.python.org/

\PYGZus{}\PYGZus{} Python\PYGZus{}
\end{sphinxVerbatimintable}
&
\sphinxhref{http://www.python.org/}{Python}%
\begin{footnote}[25]\sphinxAtStartFootnote
\sphinxnolinkurl{http://www.python.org/}
%
\end{footnote} is \sphinxhref{http://www.python.org/}{my favourite
programming language}%
\begin{footnote}[26]\sphinxAtStartFootnote
\sphinxnolinkurl{http://www.python.org/}
%
\end{footnote}.
\\
\hline
\begin{sphinxVerbatimintable}[commandchars=\\\{\}]
Titles are targets, too
=======================

Implict references, like `Titles are targets, too`\PYGZus{}.
\end{sphinxVerbatimintable}
&\phantomsection\label{\detokenize{rst-cheatsheet/rst-cheatsheet:titles-are-targets-too}}
Titles are targets, too

Implict references, like
{\hyperref[\detokenize{rst-cheatsheet/rst-cheatsheet:titles-are-targets-too}]{\sphinxcrossref{Titles are targets, too}}}.
\\
\hline\sphinxstartmulticolumn{2}%
\begin{varwidth}[t]{\sphinxcolwidth{2}{2}}
Directives are a general\sphinxhyphen{}purpose extension mechanism, a way of adding support for new constructs without adding
new syntax. For a description of all standard directives, see reStructuredText Directives (\sphinxurl{http://is.gd/2Ecqh}).
\par
\vskip-\baselineskip\vbox{\hbox{\strut}}\end{varwidth}%
\sphinxstopmulticolumn
\\
\hline
\begin{sphinxVerbatimintable}[commandchars=\\\{\}]
\PYG{n}{For} \PYG{n}{instance}\PYG{p}{:}

\PYG{o}{.}\PYG{o}{.} \PYG{n}{image}\PYG{p}{:}\PYG{p}{:} \PYG{n}{magnetic}\PYG{o}{\PYGZhy{}}\PYG{n}{balls}\PYG{o}{.}\PYG{n}{jpg}
   \PYG{p}{:}\PYG{n}{width}\PYG{p}{:} \PYG{l+m+mi}{40}\PYG{n}{pt}
\end{sphinxVerbatimintable}
&
For instance:

\noindent\sphinxincludegraphics[width=40bp]{{magnetic-balls}.jpg}
\\
\hline\sphinxstartmulticolumn{2}%
\begin{varwidth}[t]{\sphinxcolwidth{2}{2}}
Substitutions are like inline directives, allowing graphics and arbitrary constructs within text.
\par
\vskip-\baselineskip\vbox{\hbox{\strut}}\end{varwidth}%
\sphinxstopmulticolumn
\\
\hline
\begin{sphinxVerbatimintable}[commandchars=\\\{\}]
\PYG{n}{The} \PYG{o}{|}\PYG{n}{biohazard}\PYG{o}{|} \PYG{n}{symbol} \PYG{n}{must} \PYG{n}{be} \PYG{n}{used} \PYG{n}{on} \PYG{n}{containers} \PYG{n}{used} \PYG{n}{to}
\PYG{n}{dispose} \PYG{n}{of} \PYG{n}{medical} \PYG{n}{waste}\PYG{o}{.}

\PYG{o}{.}\PYG{o}{.} \PYG{o}{|}\PYG{n}{biohazard}\PYG{o}{|} \PYG{n}{image}\PYG{p}{:}\PYG{p}{:} \PYG{n}{biohazard}\PYG{o}{.}\PYG{n}{png}
   \PYG{p}{:}\PYG{n}{align}\PYG{p}{:} \PYG{n}{middle}
   \PYG{p}{:}\PYG{n}{width}\PYG{p}{:} \PYG{l+m+mi}{12}
\end{sphinxVerbatimintable}
&
The \raisebox{-0.5\height}{\sphinxincludegraphics[width=12\sphinxpxdimen]{{biohazard}.png}} symbol must be used on containers used to
dispose of medical waste.
\\
\hline\sphinxstartmulticolumn{2}%
\begin{varwidth}[t]{\sphinxcolwidth{2}{2}}
Any text which begins with an explicit markup start but doesn’t use the syntax of any of the constructs above, is a comment.
\par
\vskip-\baselineskip\vbox{\hbox{\strut}}\end{varwidth}%
\sphinxstopmulticolumn
\\
\hline
\begin{sphinxVerbatimintable}[commandchars=\\\{\}]
\PYG{o}{.}\PYG{o}{.} \PYG{n}{This} \PYG{n}{text} \PYG{n}{will} \PYG{o+ow}{not} \PYG{n}{be} \PYG{n}{shown}
   \PYG{p}{(}\PYG{n}{but}\PYG{p}{,} \PYG{k}{for} \PYG{n}{instance}\PYG{p}{,} \PYG{o+ow}{in} \PYG{n}{HTML} \PYG{n}{might} \PYG{n}{be}
   \PYG{n}{rendered} \PYG{k}{as} \PYG{n}{an} \PYG{n}{HTML} \PYG{n}{comment}\PYG{p}{)}
\end{sphinxVerbatimintable}
&\\
\hline
\begin{sphinxVerbatimintable}[commandchars=\\\{\}]
\PYG{n}{An} \PYG{l+s+s2}{\PYGZdq{}}\PYG{l+s+s2}{empty comment}\PYG{l+s+s2}{\PYGZdq{}} \PYG{n}{does} \PYG{o+ow}{not}
\PYG{n}{consume} \PYG{n}{following} \PYG{n}{blocks}\PYG{o}{.}
\PYG{p}{(}\PYG{n}{An} \PYG{n}{empty} \PYG{n}{comment} \PYG{o+ow}{is} \PYG{l+s+s2}{\PYGZdq{}}\PYG{l+s+s2}{..}\PYG{l+s+s2}{\PYGZdq{}} \PYG{k}{with}
\PYG{n}{blank} \PYG{n}{lines} \PYG{n}{before} \PYG{o+ow}{and} \PYG{n}{after}\PYG{o}{.}\PYG{p}{)}

\PYG{o}{.}\PYG{o}{.}

        \PYG{n}{So} \PYG{n}{this} \PYG{n}{block} \PYG{o+ow}{is} \PYG{o+ow}{not} \PYG{l+s+s2}{\PYGZdq{}}\PYG{l+s+s2}{lost}\PYG{l+s+s2}{\PYGZdq{}}\PYG{p}{,}
        \PYG{n}{despite} \PYG{n}{its} \PYG{n}{indentation}\PYG{o}{.}
\end{sphinxVerbatimintable}
&
An “empty comment” does not
consume following blocks.
(An empty comment is “..” with
blank lines before and after.)
\begin{quote}

So this block is not “lost”,
despite its indentation.
\end{quote}
\\
\hline
\end{tabular}
\par
\sphinxattableend\end{savenotes}


\section{Credits}
\label{\detokenize{rst-cheatsheet/rst-cheatsheet:credits}}

\begin{savenotes}\sphinxattablestart
\centering
\begin{tabulary}{\linewidth}[t]{|T|T|}
\hline

CP Font from LiquiType:
&
\sphinxurl{http://www.liquitype.com/workshop/type\_design/cp-mono}
\\
\hline
Magnetic Balls V2 image by fdecomite:
&
\sphinxurl{http://www.flickr.com/photos/fdecomite/2926556794/}
\\
\hline
Sponsored by Net Managers
&
\sphinxurl{http://www.netmanagers.com.ar}
\\
\hline
Typeset using rst2pdf
&
\sphinxurl{http://rst2pdf.googlecode.com}
\\
\hline
\end{tabulary}
\par
\sphinxattableend\end{savenotes}


\begin{savenotes}\sphinxattablestart
\centering
\begin{tabulary}{\linewidth}[t]{|T|T|T|T|}
\hline

© \DUrole{small}{2009 Roberto Alsina <ralsina@netmanagers.com.ar>  /  Creative Commons Attribution\sphinxhyphen{}Noncommercial\sphinxhyphen{}Share Alike 2.5 Argentina License}
&
\raisebox{-0.5\height}{\sphinxincludegraphics[width=8bp]{{attrib}.png}} \DUrole{small}{Based on quickref.txt from docutils}
&
\raisebox{-0.5\height}{\sphinxincludegraphics[width=8bp]{{noncomm}.png}} \DUrole{small}{Non\sphinxhyphen{}Commercial}
&
\raisebox{-0.5\height}{\sphinxincludegraphics[width=8bp]{{sharealike}.png}} \DUrole{small}{Share Alike}
\\
\hline
\end{tabulary}
\par
\sphinxattableend\end{savenotes}


\chapter{Change Log}
\label{\detokenize{change-log:change-log}}\label{\detokenize{change-log::doc}}
\begin{sphinxadmonition}{warning}{Warning:}
Changes are not being tracked until a beta\sphinxhyphen{}quality release is made.
\end{sphinxadmonition}

The change log will appear here.


\chapter{License}
\label{\detokenize{license:license}}\label{\detokenize{license::doc}}
Copyright (c) 2019 Kevin Sheppard <\sphinxhref{mailto:kevin.k.sheppard@gmail.com}{kevin.k.sheppard@gmail.com}>

Derived from:
\begin{itemize}
\item {} 
Material for Mkdocs: Copyright (c) 2016\sphinxhyphen{}2019 Martin Donath <\sphinxhref{mailto:martin.donath@squidfunk.com}{martin.donath@squidfunk.com}>

\item {} 
Guzzle Sphinx Theme: Copyright (c) 2013 Michael Dowling <\sphinxhref{mailto:mtdowling@gmail.com}{mtdowling@gmail.com}>

\end{itemize}

Permission is hereby granted, free of charge, to any person obtaining a copy
of this software and associated documentation files (the “Software”), to deal
in the Software without restriction, including without limitation the rights
to use, copy, modify, merge, publish, distribute, sublicense, and/or sell
copies of the Software, and to permit persons to whom the Software is furnished
to do so, subject to the following conditions:

The above copyright notice and this permission notice shall be included in all
copies or substantial portions of the Software.

THE SOFTWARE IS PROVIDED “AS IS”, WITHOUT WARRANTY OF ANY KIND, EXPRESS OR
IMPLIED, INCLUDING BUT NOT LIMITED TO THE WARRANTIES OF MERCHANTABILITY, FITNESS
FOR A PARTICULAR PURPOSE AND NONINFRINGEMENT. IN NO EVENT SHALL THE AUTHORS OR
COPYRIGHT HOLDERS BE LIABLE FOR ANY CLAIM, DAMAGES OR OTHER LIABILITY, WHETHER
IN AN ACTION OF CONTRACT, TORT OR OTHERWISE, ARISING FROM, OUT OF OR IN CONNECTION
WITH THE SOFTWARE OR THE USE OR OTHER DEALINGS IN THE SOFTWARE.


\chapter{Index}
\label{\detokenize{index:index}}
\DUrole{xref,std,std-ref}{genindex}

\begin{sphinxthebibliography}{CIT2002}
\bibitem[CIT2002]{rst-cheatsheet/rst-cheatsheet:cit2002}
A citation
(as often used in journals).
\bibitem[this]{rst-cheatsheet/rst-cheatsheet:this}
here.
\end{sphinxthebibliography}



\renewcommand{\indexname}{Index}
\footnotesize\raggedright\printindex
\end{document}
%% Generated by Sphinx.
\def\sphinxdocclass{report}
\documentclass[letterpaper,10pt,english]{sphinxmanual}
\ifdefined\pdfpxdimen
   \let\sphinxpxdimen\pdfpxdimen\else\newdimen\sphinxpxdimen
\fi \sphinxpxdimen=.75bp\relax
%% turn off hyperref patch of \index as sphinx.xdy xindy module takes care of
%% suitable \hyperpage mark-up, working around hyperref-xindy incompatibility
\PassOptionsToPackage{hyperindex=false}{hyperref}

\PassOptionsToPackage{warn}{textcomp}

\catcode`^^^^00a0\active\protected\def^^^^00a0{\leavevmode\nobreak\ }
\usepackage{cmap}
\usepackage{fontspec}
\defaultfontfeatures[\rmfamily,\sffamily,\ttfamily]{}
\usepackage{amsmath,amssymb,amstext}
\usepackage{polyglossia}
\setmainlanguage{english}



\usepackage{kotex}
\setmainfont[Mapping=tex-text]{NanumBarunGothic}
\setsansfont[Mapping=tex-text]{Noto Sans CJK KR}
\setmonofont{Monaco}
\setmainhangulfont[Mapping=tex-text]{NanumBarunGothic}
\setsanshangulfont[Mapping=tex-text]{Noto Sans CJK KR}
\setmonohangulfont{Monaco}


\usepackage[Bjornstrup]{fncychap}
\usepackage{sphinx}

\fvset{fontsize=\small}
\usepackage{geometry}


% Include hyperref last.
\usepackage{hyperref}
% Fix anchor placement for figures with captions.
\usepackage{hypcap}% it must be loaded after hyperref.
% Set up styles of URL: it should be placed after hyperref.
\urlstyle{same}

\addto\captionsenglish{\renewcommand{\contentsname}{Other Examples and Uses}}

\usepackage{sphinxmessages}
\setcounter{tocdepth}{1}


\usepackage[titles]{tocloft}
\cftsetpnumwidth {1.25cm}\cftsetrmarg{1.5cm}
\setlength{\cftchapnumwidth}{0.75cm}
\setlength{\cftsecindent}{\cftchapnumwidth}
\setlength{\cftsecnumwidth}{1.25cm}


\title{Material for Sphinx}
\date{Dec 24, 2020}
\release{0.0.32+6.g3c3d153.dirty}
\author{Changbum Hong}
\newcommand{\sphinxlogo}{\vbox{}}
\renewcommand{\releasename}{Release}
\makeindex
\begin{document}

\pagestyle{empty}
\sphinxmaketitle
\pagestyle{plain}
\sphinxtableofcontents
\pagestyle{normal}
\phantomsection\label{\detokenize{index::doc}}


\noindent\sphinxincludegraphics{{screenshot}.png}

This theme provides a responsive Material Design theme for Sphinx
documentation. It derives heavily from
\sphinxhref{https://squidfunk.github.io/mkdocs-material/}{Material for MkDocs}%
\begin{footnote}[1]\sphinxAtStartFootnote
\sphinxnolinkurl{https://squidfunk.github.io/mkdocs-material/}
%
\end{footnote},
and also uses code from
\sphinxhref{https://github.com/guzzle/guzzle\_sphinx\_theme}{Guzzle Sphinx Theme}%
\begin{footnote}[2]\sphinxAtStartFootnote
\sphinxnolinkurl{https://github.com/guzzle/guzzle\_sphinx\_theme}
%
\end{footnote}.


\chapter{로드맵}
\label{\detokenize{index:id1}}
본 문서는 매터리얼 디자인을 위한 대한민국입니다.
\sphinxhref{https://github.com/bashtage/sphinx-material}{Material for Sphinx}%
\begin{footnote}[3]\sphinxAtStartFootnote
\sphinxnolinkurl{https://github.com/bashtage/sphinx-material}
%
\end{footnote} is a work in progress.  While
I believe that it is ready for use, there are a number of important limitation.  The most
important it to improve the CSS generation to use
\sphinxhref{https://en.wikipedia.org/wiki/Sass\_(stylesheet\_language)}{SASS}%
\begin{footnote}[4]\sphinxAtStartFootnote
\sphinxnolinkurl{https://en.wikipedia.org/wiki/Sass\_(stylesheet\_language)}
%
\end{footnote}. It uses some python to
modify Sphinx output, which is not ideal.

The other issues are:
\begin{itemize}
\item {} 
improving the documentation;

\item {} 
providing examples;

\item {} 
sidebar customization;

\item {} 
improving the search box; and

\item {} 
ensuring that all Sphinx blocks work as intended.

\end{itemize}

You can see how it works on \sphinxhref{https://www.statsmodels.org/}{statsmodels}%
\begin{footnote}[6]\sphinxAtStartFootnote
\sphinxnolinkurl{https://www.statsmodels.org/}
%
\end{footnote}.


\chapter{시작하기}
\label{\detokenize{index:id2}}
Install from git

\begin{sphinxVerbatim}[commandchars=\\\{\}]
pip install git+https://github.com/bashtage/sphinx\PYGZhy{}material.git
\end{sphinxVerbatim}

Update your \sphinxcode{\sphinxupquote{conf.py}} with the required changes:

\begin{sphinxVerbatim}[commandchars=\\\{\}]
\PYG{n}{html\PYGZus{}theme} \PYG{o}{=} \PYG{l+s+s1}{\PYGZsq{}}\PYG{l+s+s1}{sphinx\PYGZus{}material}\PYG{l+s+s1}{\PYGZsq{}}
\end{sphinxVerbatim}

There are a lot more ways to customize this theme. See \DUrole{xref,std,std-ref}{Customization}
or \sphinxcode{\sphinxupquote{theme.conf}} for more details.

\begin{sphinxVerbatim}[commandchars=\\\{\}]
\PYG{n}{html\PYGZus{}theme} \PYG{o}{=} \PYG{l+s+s1}{\PYGZsq{}}\PYG{l+s+s1}{sphinx\PYGZus{}material}\PYG{l+s+s1}{\PYGZsq{}}

\PYG{c+c1}{\PYGZsh{} Material theme options (see theme.conf for more information)}
\PYG{n}{html\PYGZus{}theme\PYGZus{}options} \PYG{o}{=} \PYG{p}{\PYGZob{}}

    \PYG{c+c1}{\PYGZsh{} Set the name of the project to appear in the navigation.}
    \PYG{l+s+s1}{\PYGZsq{}}\PYG{l+s+s1}{nav\PYGZus{}title}\PYG{l+s+s1}{\PYGZsq{}}\PYG{p}{:} \PYG{l+s+s1}{\PYGZsq{}}\PYG{l+s+s1}{Project Name}\PYG{l+s+s1}{\PYGZsq{}}\PYG{p}{,}

    \PYG{c+c1}{\PYGZsh{} Set you GA account ID to enable tracking}
    \PYG{l+s+s1}{\PYGZsq{}}\PYG{l+s+s1}{google\PYGZus{}analytics\PYGZus{}account}\PYG{l+s+s1}{\PYGZsq{}}\PYG{p}{:} \PYG{l+s+s1}{\PYGZsq{}}\PYG{l+s+s1}{UA\PYGZhy{}XXXXX}\PYG{l+s+s1}{\PYGZsq{}}\PYG{p}{,}

    \PYG{c+c1}{\PYGZsh{} Specify a base\PYGZus{}url used to generate sitemap.xml. If not}
    \PYG{c+c1}{\PYGZsh{} specified, then no sitemap will be built.}
    \PYG{l+s+s1}{\PYGZsq{}}\PYG{l+s+s1}{base\PYGZus{}url}\PYG{l+s+s1}{\PYGZsq{}}\PYG{p}{:} \PYG{l+s+s1}{\PYGZsq{}}\PYG{l+s+s1}{https://project.github.io/project}\PYG{l+s+s1}{\PYGZsq{}}\PYG{p}{,}

    \PYG{c+c1}{\PYGZsh{} Set the color and the accent color}
    \PYG{l+s+s1}{\PYGZsq{}}\PYG{l+s+s1}{color\PYGZus{}primary}\PYG{l+s+s1}{\PYGZsq{}}\PYG{p}{:} \PYG{l+s+s1}{\PYGZsq{}}\PYG{l+s+s1}{blue}\PYG{l+s+s1}{\PYGZsq{}}\PYG{p}{,}
    \PYG{l+s+s1}{\PYGZsq{}}\PYG{l+s+s1}{color\PYGZus{}accent}\PYG{l+s+s1}{\PYGZsq{}}\PYG{p}{:} \PYG{l+s+s1}{\PYGZsq{}}\PYG{l+s+s1}{light\PYGZhy{}blue}\PYG{l+s+s1}{\PYGZsq{}}\PYG{p}{,}

    \PYG{c+c1}{\PYGZsh{} Set the repo location to get a badge with stats}
    \PYG{l+s+s1}{\PYGZsq{}}\PYG{l+s+s1}{repo\PYGZus{}url}\PYG{l+s+s1}{\PYGZsq{}}\PYG{p}{:} \PYG{l+s+s1}{\PYGZsq{}}\PYG{l+s+s1}{https://github.com/project/project/}\PYG{l+s+s1}{\PYGZsq{}}\PYG{p}{,}
    \PYG{l+s+s1}{\PYGZsq{}}\PYG{l+s+s1}{repo\PYGZus{}name}\PYG{l+s+s1}{\PYGZsq{}}\PYG{p}{:} \PYG{l+s+s1}{\PYGZsq{}}\PYG{l+s+s1}{Project}\PYG{l+s+s1}{\PYGZsq{}}\PYG{p}{,}

    \PYG{c+c1}{\PYGZsh{} Visible levels of the global TOC; \PYGZhy{}1 means unlimited}
    \PYG{l+s+s1}{\PYGZsq{}}\PYG{l+s+s1}{globaltoc\PYGZus{}depth}\PYG{l+s+s1}{\PYGZsq{}}\PYG{p}{:} \PYG{l+m+mi}{3}\PYG{p}{,}
    \PYG{c+c1}{\PYGZsh{} If False, expand all TOC entries}
    \PYG{l+s+s1}{\PYGZsq{}}\PYG{l+s+s1}{globaltoc\PYGZus{}collapse}\PYG{l+s+s1}{\PYGZsq{}}\PYG{p}{:} \PYG{n+nb+bp}{False}\PYG{p}{,}
    \PYG{c+c1}{\PYGZsh{} If True, show hidden TOC entries}
    \PYG{l+s+s1}{\PYGZsq{}}\PYG{l+s+s1}{globaltoc\PYGZus{}includehidden}\PYG{l+s+s1}{\PYGZsq{}}\PYG{p}{:} \PYG{n+nb+bp}{False}\PYG{p}{,}
\PYG{p}{\PYGZcb{}}
\end{sphinxVerbatim}


\section{NumPy Docstrings}
\label{\detokenize{numpydoc:numpy-docstrings}}\label{\detokenize{numpydoc::doc}}
This page shows how \sphinxcode{\sphinxupquote{autosummary}} works with \sphinxcode{\sphinxupquote{numpydoc}} and a
NumPy\sphinxhyphen{}style docstring.


\section{Markdown}
\label{\detokenize{markdown:markdown}}\label{\detokenize{markdown::doc}}
Sphinx can be configured to use markdown using the \sphinxhref{https://github.com/readthedocs/recommonmark}{recommonmark}%
\begin{footnote}[7]\sphinxAtStartFootnote
\sphinxnolinkurl{https://github.com/readthedocs/recommonmark}
%
\end{footnote}
extension. recommonmark is strict and does not natively support tables or common extensions
to markdown.


\bigskip\hrule\bigskip



\subsection{Body copy}
\label{\detokenize{markdown:body-copy}}
Lorem ipsum dolor sit amet, consectetur adipiscing elit. Cras arcu libero,
mollis sed massa vel, \sphinxstyleemphasis{ornare viverra ex}. Mauris a ullamcorper lacus. Nullam
urna elit, malesuada eget finibus ut, ullamcorper ac tortor. Vestibulum sodales
pulvinar nisl, pharetra aliquet est. Quisque volutpat erat ac nisi accumsan
tempor.

\sphinxstylestrong{Sed suscipit}, orci non pretium pretium, quam mi gravida metus, vel
venenatis justo est condimentum diam. Maecenas non ornare justo. Nam a ipsum
eros. {\hyperref[\detokenize{markdown:}]{\emph{Nulla aliquam}}} orci sit amet nisl posuere malesuada. Proin aliquet
nulla velit, quis ultricies orci feugiat et. \sphinxcode{\sphinxupquote{Ut tincidunt sollicitudin}}
tincidunt. Aenean ullamcorper sit amet nulla at interdum.


\subsection{Headings}
\label{\detokenize{markdown:headings}}

\subsubsection{The 3rd level}
\label{\detokenize{markdown:the-3rd-level}}

\paragraph{The 4th level}
\label{\detokenize{markdown:the-4th-level}}

\subparagraph{The 5th level}
\label{\detokenize{markdown:the-5th-level}}

\subparagraph{The 6th level}
\label{\detokenize{markdown:the-6th-level}}

\subsection{Headings with secondary text}
\label{\detokenize{markdown:headings-small-with-secondary-text-small}}

\subsubsection{The 3rd level with secondary text}
\label{\detokenize{markdown:the-3rd-level-small-with-secondary-text-small}}

\paragraph{The 4th level with secondary text}
\label{\detokenize{markdown:the-4th-level-small-with-secondary-text-small}}

\subparagraph{The 5th level with secondary text}
\label{\detokenize{markdown:the-5th-level-small-with-secondary-text-small}}

\subparagraph{The 6th level with secondary text}
\label{\detokenize{markdown:the-6th-level-small-with-secondary-text-small}}

\subsection{Blockquotes}
\label{\detokenize{markdown:blockquotes}}\begin{quote}

Morbi eget dapibus felis. Vivamus venenatis porttitor tortor sit amet rutrum.
Pellentesque aliquet quam enim, eu volutpat urna rutrum a. Nam vehicula nunc
mauris, a ultricies libero efficitur sed. \sphinxstyleemphasis{Class aptent} taciti sociosqu ad
litora torquent per conubia nostra, per inceptos himenaeos. Sed molestie
imperdiet consectetur.
\end{quote}


\subsubsection{Blockquote nesting}
\label{\detokenize{markdown:blockquote-nesting}}\begin{quote}

\sphinxstylestrong{Sed aliquet}, neque at rutrum mollis, neque nisi tincidunt nibh, vitae
faucibus lacus nunc at lacus. Nunc scelerisque, quam id cursus sodales, lorem
{\hyperref[\detokenize{markdown:}]{\emph{libero fermentum}}} urna, ut efficitur elit ligula et nunc.
\end{quote}
\begin{quote}
\begin{quote}

Mauris dictum mi lacus, sit amet pellentesque urna vehicula fringilla.
Ut sit amet placerat ante. Proin sed elementum nulla. Nunc vitae sem odio.
Suspendisse ac eros arcu. Vivamus orci erat, volutpat a tempor et, rutrum.
eu odio.
\end{quote}
\end{quote}
\begin{quote}
\begin{quote}
\begin{quote}

\sphinxcode{\sphinxupquote{Suspendisse rutrum facilisis risus}}, eu posuere neque commodo a.
Interdum et malesuada fames ac ante ipsum primis in faucibus. Sed nec leo
bibendum, sodales mauris ut, tincidunt massa.
\end{quote}
\end{quote}
\end{quote}


\subsubsection{Other content blocks}
\label{\detokenize{markdown:other-content-blocks}}\begin{quote}

Vestibulum vitae orci quis ante viverra ultricies ut eget turpis. Sed eu
lectus dapibus, eleifend nulla varius, lobortis turpis. In ac hendrerit nisl,
sit amet laoreet nibh.
\end{quote}

\begin{sphinxVerbatim}[commandchars=\\\{\}]
\PYG{n}{var} \PYG{n}{\PYGZus{}extends} \PYG{o}{=} \PYG{n}{function}\PYG{p}{(}\PYG{n}{target}\PYG{p}{)} \PYG{p}{\PYGZob{}}
  \PYG{k}{for} \PYG{p}{(}\PYG{n}{var} \PYG{n}{i} \PYG{o}{=} \PYG{l+m+mi}{1}\PYG{p}{;} \PYG{n}{i} \PYG{o}{\PYGZlt{}} \PYG{n}{arguments}\PYG{o}{.}\PYG{n}{length}\PYG{p}{;} \PYG{n}{i}\PYG{o}{+}\PYG{o}{+}\PYG{p}{)} \PYG{p}{\PYGZob{}}
    \PYG{n}{var} \PYG{n}{source} \PYG{o}{=} \PYG{n}{arguments}\PYG{p}{[}\PYG{n}{i}\PYG{p}{]}\PYG{p}{;}
    \PYG{k}{for} \PYG{p}{(}\PYG{n}{var} \PYG{n}{key} \PYG{o+ow}{in} \PYG{n}{source}\PYG{p}{)} \PYG{p}{\PYGZob{}}
      \PYG{n}{target}\PYG{p}{[}\PYG{n}{key}\PYG{p}{]} \PYG{o}{=} \PYG{n}{source}\PYG{p}{[}\PYG{n}{key}\PYG{p}{]}\PYG{p}{;}
    \PYG{p}{\PYGZcb{}}
  \PYG{p}{\PYGZcb{}}
  \PYG{k}{return} \PYG{n}{target}\PYG{p}{;}
\PYG{p}{\PYGZcb{}}\PYG{p}{;}
\end{sphinxVerbatim}
\begin{quote}
\begin{quote}

Praesent at \sphinxcode{\sphinxupquote{:::js return target}}, sodales nibh vel, tempor felis. Fusce
vel lacinia lacus. Suspendisse rhoncus nunc non nisi iaculis ultrices.
Donec consectetur mauris non neque imperdiet, eget volutpat libero.
\end{quote}
\end{quote}


\subsection{Lists}
\label{\detokenize{markdown:lists}}

\subsubsection{Unordered lists}
\label{\detokenize{markdown:unordered-lists}}\begin{itemize}
\item {} 
Sed sagittis eleifend rutrum. Donec vitae suscipit est. Nullam tempus tellus
non sem sollicitudin, quis rutrum leo facilisis. Nulla tempor lobortis orci,
at elementum urna sodales vitae. In in vehicula nulla, quis ornare libero.
\begin{itemize}
\item {} 
Duis mollis est eget nibh volutpat, fermentum aliquet dui mollis.

\item {} 
Nam vulputate tincidunt fringilla.

\item {} 
Nullam dignissim ultrices urna non auctor.

\end{itemize}

\item {} 
Aliquam metus eros, pretium sed nulla venenatis, faucibus auctor ex. Proin ut
eros sed sapien ullamcorper consequat. Nunc ligula ante, fringilla at aliquam
ac, aliquet sed mauris.

\item {} 
Nulla et rhoncus turpis. Mauris ultricies elementum leo. Duis efficitur
accumsan nibh eu mattis. Vivamus tempus velit eros, porttitor placerat nibh
lacinia sed. Aenean in finibus diam.

\end{itemize}


\subsubsection{Ordered lists}
\label{\detokenize{markdown:ordered-lists}}\begin{enumerate}
\sphinxsetlistlabels{\arabic}{enumi}{enumii}{}{.}%
\item {} 
Integer vehicula feugiat magna, a mollis tellus. Nam mollis ex ante, quis
elementum eros tempor rutrum. Aenean efficitur lobortis lacinia. Nulla
consectetur feugiat sodales.

\item {} 
Cum sociis natoque penatibus et magnis dis parturient montes, nascetur
ridiculus mus. Aliquam ornare feugiat quam et egestas. Nunc id erat et quam
pellentesque lacinia eu vel odio.
\begin{enumerate}
\sphinxsetlistlabels{\arabic}{enumii}{enumiii}{}{.}%
\item {} 
Vivamus venenatis porttitor tortor sit amet rutrum. Pellentesque aliquet
quam enim, eu volutpat urna rutrum a. Nam vehicula nunc mauris, a
ultricies libero efficitur sed.
\begin{enumerate}
\sphinxsetlistlabels{\arabic}{enumiii}{enumiv}{}{.}%
\item {} 
Mauris dictum mi lacus

\item {} 
Ut sit amet placerat ante

\item {} 
Suspendisse ac eros arcu

\end{enumerate}

\item {} 
Morbi eget dapibus felis. Vivamus venenatis porttitor tortor sit amet
rutrum. Pellentesque aliquet quam enim, eu volutpat urna rutrum a. Sed
aliquet, neque at rutrum mollis, neque nisi tincidunt nibh.

\item {} 
Pellentesque eget \sphinxcode{\sphinxupquote{:::js var \_extends}} ornare tellus, ut gravida mi.

\end{enumerate}

\begin{sphinxVerbatim}[commandchars=\\\{\}]
\PYG{n}{var} \PYG{n}{\PYGZus{}extends} \PYG{o}{=} \PYG{n}{function}\PYG{p}{(}\PYG{n}{target}\PYG{p}{)} \PYG{p}{\PYGZob{}}
  \PYG{k}{for} \PYG{p}{(}\PYG{n}{var} \PYG{n}{i} \PYG{o}{=} \PYG{l+m+mi}{1}\PYG{p}{;} \PYG{n}{i} \PYG{o}{\PYGZlt{}} \PYG{n}{arguments}\PYG{o}{.}\PYG{n}{length}\PYG{p}{;} \PYG{n}{i}\PYG{o}{+}\PYG{o}{+}\PYG{p}{)} \PYG{p}{\PYGZob{}}
    \PYG{n}{var} \PYG{n}{source} \PYG{o}{=} \PYG{n}{arguments}\PYG{p}{[}\PYG{n}{i}\PYG{p}{]}\PYG{p}{;}
    \PYG{k}{for} \PYG{p}{(}\PYG{n}{var} \PYG{n}{key} \PYG{o+ow}{in} \PYG{n}{source}\PYG{p}{)} \PYG{p}{\PYGZob{}}
      \PYG{n}{target}\PYG{p}{[}\PYG{n}{key}\PYG{p}{]} \PYG{o}{=} \PYG{n}{source}\PYG{p}{[}\PYG{n}{key}\PYG{p}{]}\PYG{p}{;}
    \PYG{p}{\PYGZcb{}}
  \PYG{p}{\PYGZcb{}}
  \PYG{k}{return} \PYG{n}{target}\PYG{p}{;}
\PYG{p}{\PYGZcb{}}\PYG{p}{;}
\end{sphinxVerbatim}

\item {} 
Vivamus id mi enim. Integer id turpis sapien. Ut condimentum lobortis
sagittis. Aliquam purus tellus, faucibus eget urna at, iaculis venenatis
nulla. Vivamus a pharetra leo.

\end{enumerate}


\subsubsection{Definition lists}
\label{\detokenize{markdown:definition-lists}}
\sphinxstylestrong{Not supported in commonmark, but you can use a rst definition list inside a
fenced eval\_rst block.}
\begin{description}
\item[{Lorem ipsum dolor sit amet}] \leavevmode
Sed sagittis eleifend rutrum. Donec vitae suscipit est. Nullam tempus
tellus non sem sollicitudin, quis rutrum leo facilisis. Nulla tempor
lobortis orci, at elementum urna sodales vitae. In in vehicula nulla.

Duis mollis est eget nibh volutpat, fermentum aliquet dui mollis. Nam
vulputate tincidunt fringilla. Nullam dignissim ultrices urna non
auctor.

\item[{Cras arcu libero}] \leavevmode
Aliquam metus eros, pretium sed nulla venenatis, faucibus auctor ex.
Proin ut eros sed sapien ullamcorper consequat. Nunc ligula ante,
fringilla at aliquam ac, aliquet sed mauris.

\end{description}


\subsection{Code blocks}
\label{\detokenize{markdown:code-blocks}}

\subsubsection{Inline}
\label{\detokenize{markdown:inline}}
Morbi eget \sphinxcode{\sphinxupquote{dapibus felis}}. Vivamus \sphinxstyleemphasis{\sphinxcode{\sphinxupquote{venenatis porttitor}}} tortor sit amet
rutrum. Class aptent taciti sociosqu ad litora torquent per conubia nostra,
per inceptos himenaeos. {\hyperref[\detokenize{markdown:}]{\emph{\sphinxcode{\sphinxupquote{Pellentesque aliquet quam enim}}}}}, eu volutpat urna
rutrum a.

Nam vehicula nunc \sphinxcode{\sphinxupquote{:::js return target}} mauris, a ultricies libero efficitur
sed. Sed molestie imperdiet consectetur. Vivamus a pharetra leo. Pellentesque
eget ornare tellus, ut gravida mi. Fusce vel lacinia lacus.


\subsubsection{Listing}
\label{\detokenize{markdown:listing}}
\begin{sphinxVerbatim}[commandchars=\\\{\}]
\PYG{n}{var} \PYG{n}{\PYGZus{}extends} \PYG{o}{=} \PYG{n}{function}\PYG{p}{(}\PYG{n}{target}\PYG{p}{)} \PYG{p}{\PYGZob{}}
  \PYG{k}{for} \PYG{p}{(}\PYG{n}{var} \PYG{n}{i} \PYG{o}{=} \PYG{l+m+mi}{1}\PYG{p}{;} \PYG{n}{i} \PYG{o}{\PYGZlt{}} \PYG{n}{arguments}\PYG{o}{.}\PYG{n}{length}\PYG{p}{;} \PYG{n}{i}\PYG{o}{+}\PYG{o}{+}\PYG{p}{)} \PYG{p}{\PYGZob{}}
    \PYG{n}{var} \PYG{n}{source} \PYG{o}{=} \PYG{n}{arguments}\PYG{p}{[}\PYG{n}{i}\PYG{p}{]}\PYG{p}{;}
    \PYG{k}{for} \PYG{p}{(}\PYG{n}{var} \PYG{n}{key} \PYG{o+ow}{in} \PYG{n}{source}\PYG{p}{)} \PYG{p}{\PYGZob{}}
      \PYG{n}{target}\PYG{p}{[}\PYG{n}{key}\PYG{p}{]} \PYG{o}{=} \PYG{n}{source}\PYG{p}{[}\PYG{n}{key}\PYG{p}{]}\PYG{p}{;}
    \PYG{p}{\PYGZcb{}}
  \PYG{p}{\PYGZcb{}}
  \PYG{k}{return} \PYG{n}{target}\PYG{p}{;}
\PYG{p}{\PYGZcb{}}\PYG{p}{;}
\end{sphinxVerbatim}


\subsection{Horizontal rules}
\label{\detokenize{markdown:horizontal-rules}}
Aenean in finibus diam. Duis mollis est eget nibh volutpat, fermentum aliquet
dui mollis. Nam vulputate tincidunt fringilla. Nullam dignissim ultrices urna
non auctor.


\bigskip\hrule\bigskip


Integer vehicula feugiat magna, a mollis tellus. Nam mollis ex ante, quis
elementum eros tempor rutrum. Aenean efficitur lobortis lacinia. Nulla
consectetur feugiat sodales.


\subsection{Data tables}
\label{\detokenize{markdown:data-tables}}
\sphinxstylestrong{Note}: Markdown table syntax requires \sphinxcode{\sphinxupquote{sphinx\_markdown\_tables}}



Sed sagittis eleifend rutrum. Donec vitae suscipit est. Nullam tempus tellus
non sem sollicitudin, quis rutrum leo facilisis. Nulla tempor lobortis orci,
at elementum urna sodales vitae. In in vehicula nulla, quis ornare libero.



Vestibulum vitae orci quis ante viverra ultricies ut eget turpis. Sed eu
lectus dapibus, eleifend nulla varius, lobortis turpis. In ac hendrerit nisl,
sit amet laoreet nibh.




\section{rst Cheatsheet}
\label{\detokenize{rst-cheatsheet/rst-cheatsheet:rst-cheatsheet}}\label{\detokenize{rst-cheatsheet/rst-cheatsheet::doc}}
The \sphinxhref{https://github.com/ralsina/rst-cheatsheet}{rst Cheatsheet}%
\begin{footnote}[8]\sphinxAtStartFootnote
\sphinxnolinkurl{https://github.com/ralsina/rst-cheatsheet}
%
\end{footnote} covers a wide range of
rst markup. It and its contents are
\sphinxhref{http://creativecommons.org/licenses/by/3.0/de/deed.en\_GB}{CC licensed}%
\begin{footnote}[9]\sphinxAtStartFootnote
\sphinxnolinkurl{http://creativecommons.org/licenses/by/3.0/de/deed.en\_GB}
%
\end{footnote}.


\subsection{Inline Markup}
\label{\detokenize{rst-cheatsheet/rst-cheatsheet:inline-markup}}
Inline markup allows words and phrases within text to have character styles (like italics and boldface) and functionality (like hyperlinks).


\begin{savenotes}\sphinxattablestart
\centering
\begin{tabular}[t]{|*{2}{\X{1}{2}|}}
\hline

\begin{sphinxVerbatimintable}[commandchars=\\\{\}]
\PYG{o}{*}\PYG{n}{emphasis}\PYG{o}{*}
\end{sphinxVerbatimintable}
&
\sphinxstyleemphasis{emphasis}
\\
\hline
\begin{sphinxVerbatimintable}[commandchars=\\\{\}]
\PYG{o}{*}\PYG{o}{*}\PYG{n}{strong} \PYG{n}{emphasis}\PYG{o}{*}\PYG{o}{*}
\end{sphinxVerbatimintable}
&
\sphinxstylestrong{strong emphasis}
\\
\hline
\begin{sphinxVerbatimintable}[commandchars=\\\{\}]
`interpreted text`
\end{sphinxVerbatimintable}
&
The rendering and meaning of interpreted text
is domain\sphinxhyphen{} or application\sphinxhyphen{}dependent.
\\
\hline
\begin{sphinxVerbatimintable}[commandchars=\\\{\}]
``inline literal``
\end{sphinxVerbatimintable}
&
\sphinxcode{\sphinxupquote{inline literal}}
\\
\hline
\begin{sphinxVerbatimintable}[commandchars=\\\{\}]
\PYG{n}{reference\PYGZus{}}
\end{sphinxVerbatimintable}
&
\sphinxhref{http://docutils.sourceforge.net/docs/user/rst/quickref.html\#hyperlink-targets}{reference}%
\begin{footnote}[10]\sphinxAtStartFootnote
\sphinxnolinkurl{http://docutils.sourceforge.net/docs/user/rst/quickref.html\#hyperlink-targets}
%
\end{footnote}
\\
\hline
\begin{sphinxVerbatimintable}[commandchars=\\\{\}]
`phrase reference`\PYGZus{}
\end{sphinxVerbatimintable}
&
\sphinxhref{http://docutils.sourceforge.net/docs/user/rst/quickref.html\#hyperlink-targets}{phrase reference}%
\begin{footnote}[11]\sphinxAtStartFootnote
\sphinxnolinkurl{http://docutils.sourceforge.net/docs/user/rst/quickref.html\#hyperlink-targets}
%
\end{footnote}
\\
\hline
\begin{sphinxVerbatimintable}[commandchars=\\\{\}]
\PYG{n}{anonymous\PYGZus{}\PYGZus{}}
\end{sphinxVerbatimintable}
&
\sphinxhref{http://docutils.sourceforge.net/docs/user/rst/quickref.html\#hyperlink-targets}{anonymous}%
\begin{footnote}[12]\sphinxAtStartFootnote
\sphinxnolinkurl{http://docutils.sourceforge.net/docs/user/rst/quickref.html\#hyperlink-targets}
%
\end{footnote}
\\
\hline
\begin{sphinxVerbatimintable}[commandchars=\\\{\}]
\PYGZus{}`inline internal target`
\end{sphinxVerbatimintable}
&
\phantomsection\label{\detokenize{rst-cheatsheet/rst-cheatsheet:inline-internal-target}}inline internal target
\\
\hline
\begin{sphinxVerbatimintable}[commandchars=\\\{\}]
\PYG{o}{|}\PYG{n}{substitution} \PYG{n}{reference}\PYG{o}{|}
\end{sphinxVerbatimintable}
&
The result is substituted in from the
substitution definition.
\\
\hline
\begin{sphinxVerbatimintable}[commandchars=\\\{\}]
\PYG{n}{footnote} \PYG{n}{reference} \PYG{p}{[}\PYG{l+m+mi}{1}\PYG{p}{]}\PYG{n}{\PYGZus{}}
\end{sphinxVerbatimintable}
&
footnote reference %
\begin{footnote}[13]\sphinxAtStartFootnote
This is the first one.
%
\end{footnote}
\\
\hline
\begin{sphinxVerbatimintable}[commandchars=\\\{\}]
\PYG{n}{citation} \PYG{n}{reference} \PYG{p}{[}\PYG{n}{CIT2002}\PYG{p}{]}\PYG{n}{\PYGZus{}}
\end{sphinxVerbatimintable}
&
citation reference \sphinxcite{rst-cheatsheet/rst-cheatsheet:cit2002}
\\
\hline
\begin{sphinxVerbatimintable}[commandchars=\\\{\}]
\PYG{n}{http}\PYG{p}{:}\PYG{o}{/}\PYG{o}{/}\PYG{n}{docutils}\PYG{o}{.}\PYG{n}{sf}\PYG{o}{.}\PYG{n}{net}\PYG{o}{/}
\end{sphinxVerbatimintable}
&
\sphinxurl{http://docutils.sf.net/}
\\
\hline
\end{tabular}
\par
\sphinxattableend\end{savenotes}


\subsection{Escaping with Backslashes}
\label{\detokenize{rst-cheatsheet/rst-cheatsheet:escaping-with-backslashes}}
reStructuredText uses backslashes (“") to override the special meaning given to markup characters and get
the literal characters themselves. To get a literal backslash, use an escaped backslash (“\textbackslash{}”). For example:


\begin{savenotes}\sphinxattablestart
\centering
\begin{tabular}[t]{|*{2}{\X{1}{2}|}}
\hline

\begin{sphinxVerbatimintable}[commandchars=\\\{\}]
*escape* ``with`` \PYGZdq{}\PYGZbs{}\PYGZdq{}
\end{sphinxVerbatimintable}
&
\sphinxstyleemphasis{escape} \sphinxcode{\sphinxupquote{with}} “”
\\
\hline
\begin{sphinxVerbatimintable}[commandchars=\\\{\}]
\PYGZbs{}*escape* \PYGZbs{}``with`` \PYGZdq{}\PYGZbs{}\PYGZbs{}\PYGZdq{}
\end{sphinxVerbatimintable}
&
*escape* ``with`` “"
\\
\hline
\end{tabular}
\par
\sphinxattableend\end{savenotes}


\subsection{Lists}
\label{\detokenize{rst-cheatsheet/rst-cheatsheet:lists}}

\begin{savenotes}\sphinxattablestart
\centering
\begin{tabular}[t]{|*{2}{\X{1}{2}|}}
\hline

\begin{sphinxVerbatimintable}[commandchars=\\\{\}]
\PYG{o}{\PYGZhy{}} \PYG{n}{This} \PYG{o+ow}{is} \PYG{n}{item} \PYG{l+m+mf}{1.} \PYG{n}{A} \PYG{n}{blank} \PYG{n}{line} \PYG{n}{before} \PYG{n}{the} \PYG{n}{first}
  \PYG{o+ow}{and} \PYG{n}{last} \PYG{n}{items} \PYG{o+ow}{is} \PYG{n}{required}\PYG{o}{.}
\PYG{o}{\PYGZhy{}} \PYG{n}{This} \PYG{o+ow}{is} \PYG{n}{item} \PYG{l+m+mi}{2}

\PYG{o}{\PYGZhy{}} \PYG{n}{Item} \PYG{l+m+mi}{3}\PYG{p}{:} \PYG{n}{blank} \PYG{n}{lines} \PYG{n}{between} \PYG{n}{items} \PYG{n}{are} \PYG{n}{optional}\PYG{o}{.}
\PYG{o}{\PYGZhy{}} \PYG{n}{Item} \PYG{l+m+mi}{4}\PYG{p}{:} \PYG{n}{Bullets} \PYG{n}{are} \PYG{l+s+s2}{\PYGZdq{}}\PYG{l+s+s2}{\PYGZhy{}}\PYG{l+s+s2}{\PYGZdq{}}\PYG{p}{,} \PYG{l+s+s2}{\PYGZdq{}}\PYG{l+s+s2}{*}\PYG{l+s+s2}{\PYGZdq{}} \PYG{o+ow}{or} \PYG{l+s+s2}{\PYGZdq{}}\PYG{l+s+s2}{+}\PYG{l+s+s2}{\PYGZdq{}}\PYG{o}{.}
  \PYG{n}{Continuing} \PYG{n}{text} \PYG{n}{must} \PYG{n}{be} \PYG{n}{aligned} \PYG{n}{after} \PYG{n}{the} \PYG{n}{bullet}
  \PYG{o+ow}{and} \PYG{n}{whitespace}\PYG{o}{.}
\PYG{o}{\PYGZhy{}} \PYG{n}{This} \PYG{n+nb}{list} \PYG{n}{item} \PYG{n}{contains} \PYG{n}{nested} \PYG{n}{items}

  \PYG{o}{\PYGZhy{}} \PYG{n}{Nested} \PYG{n}{items} \PYG{n}{must} \PYG{n}{be} \PYG{n}{indented} \PYG{n}{to} \PYG{n}{the} \PYG{n}{same}
    \PYG{n}{level}
\end{sphinxVerbatimintable}
&\begin{itemize}
\item {} 
This is item 1. A blank line before the first
and last items is required.

\item {} 
This is item 2

\item {} 
Item 3: blank lines between items are optional.

\item {} 
Item 4: Bullets are “\sphinxhyphen{}“, “*” or “+”.
Continuing text must be aligned after the bullet
and whitespace.

\item {} 
This list item contains nested items
\begin{itemize}
\item {} 
Nested items must be indented to the same
level

\end{itemize}

\end{itemize}
\\
\hline
\begin{sphinxVerbatimintable}[commandchars=\\\{\}]
\PYG{l+m+mf}{3.} \PYG{n}{This} \PYG{o+ow}{is} \PYG{n}{the} \PYG{n}{first} \PYG{n}{item}
\PYG{l+m+mf}{4.} \PYG{n}{This} \PYG{o+ow}{is} \PYG{n}{the} \PYG{n}{second} \PYG{n}{item}
\PYG{l+m+mf}{5.} \PYG{n}{Enumerators} \PYG{n}{are} \PYG{n}{arabic} \PYG{n}{numbers}\PYG{p}{,}
   \PYG{n}{single} \PYG{n}{letters}\PYG{p}{,} \PYG{o+ow}{or} \PYG{n}{roman} \PYG{n}{numerals}
\PYG{l+m+mf}{6.} \PYG{n}{List} \PYG{n}{items} \PYG{n}{should} \PYG{n}{be} \PYG{n}{sequentially}
   \PYG{n}{numbered}\PYG{p}{,} \PYG{n}{but} \PYG{n}{need} \PYG{o+ow}{not} \PYG{n}{start} \PYG{n}{at} \PYG{l+m+mi}{1}
   \PYG{p}{(}\PYG{n}{although} \PYG{o+ow}{not} \PYG{n+nb}{all} \PYG{n}{formatters} \PYG{n}{will}
   \PYG{n}{honour} \PYG{n}{the} \PYG{n}{first} \PYG{n}{index}\PYG{p}{)}\PYG{o}{.}
\PYG{c+c1}{\PYGZsh{}. This item is auto\PYGZhy{}enumerated}
\end{sphinxVerbatimintable}
&\begin{enumerate}
\sphinxsetlistlabels{\arabic}{enumi}{enumii}{}{.}%
\setcounter{enumi}{2}
\item {} 
This is the first item

\item {} 
This is the second item

\item {} 
Enumerators are arabic numbers,
single letters, or roman numerals

\item {} 
List items should be sequentially
numbered, but need not start at 1
(although not all formatters will
honour the first index).

\item {} 
This item is auto\sphinxhyphen{}enumerated

\end{enumerate}
\\
\hline
\begin{sphinxVerbatimintable}[commandchars=\\\{\}]
\PYG{n}{what}
  \PYG{n}{Definition} \PYG{n}{lists} \PYG{n}{associate} \PYG{n}{a} \PYG{n}{term} \PYG{k}{with}
  \PYG{n}{a} \PYG{n}{definition}\PYG{o}{.}

\PYG{n}{how}
  \PYG{n}{The} \PYG{n}{term} \PYG{o+ow}{is} \PYG{n}{a} \PYG{n}{one}\PYG{o}{\PYGZhy{}}\PYG{n}{line} \PYG{n}{phrase}\PYG{p}{,} \PYG{o+ow}{and} \PYG{n}{the}
  \PYG{n}{definition} \PYG{o+ow}{is} \PYG{n}{one} \PYG{o+ow}{or} \PYG{n}{more} \PYG{n}{paragraphs} \PYG{o+ow}{or}
  \PYG{n}{body} \PYG{n}{elements}\PYG{p}{,} \PYG{n}{indented} \PYG{n}{relative} \PYG{n}{to} \PYG{n}{the}
  \PYG{n}{term}\PYG{o}{.} \PYG{n}{Blank} \PYG{n}{lines} \PYG{n}{are} \PYG{o+ow}{not} \PYG{n}{allowed}
  \PYG{n}{between} \PYG{n}{term} \PYG{o+ow}{and} \PYG{n}{definition}\PYG{o}{.}
\end{sphinxVerbatimintable}
&\begin{description}
\item[{what}] \leavevmode
Definition lists associate a term with
a definition.

\item[{how}] \leavevmode
The term is a one\sphinxhyphen{}line phrase, and the
definition is one or more paragraphs or
body elements, indented relative to the
term. Blank lines are not allowed
between term and definition.

\end{description}
\\
\hline
\begin{sphinxVerbatimintable}[commandchars=\\\{\}]
\PYG{p}{:}\PYG{n}{Authors}\PYG{p}{:}
    \PYG{n}{Tony} \PYG{n}{J}\PYG{o}{.} \PYG{p}{(}\PYG{n}{Tibs}\PYG{p}{)} \PYG{n}{Ibbs}\PYG{p}{,}
    \PYG{n}{David} \PYG{n}{Goodger}

\PYG{p}{:}\PYG{n}{Version}\PYG{p}{:} \PYG{l+m+mf}{1.0} \PYG{n}{of} \PYG{l+m+mi}{2001}\PYG{o}{/}\PYG{l+m+mi}{08}\PYG{o}{/}\PYG{l+m+mi}{08}
\PYG{p}{:}\PYG{n}{Dedication}\PYG{p}{:} \PYG{n}{To} \PYG{n}{my} \PYG{n}{father}\PYG{o}{.}
\end{sphinxVerbatimintable}
&\begin{quote}\begin{description}
\item[{Authors}] \leavevmode
Tony J. (Tibs) Ibbs,
David Goodger

\item[{Version}] \leavevmode
1.0 of 2001/08/08

\item[{Dedication}] \leavevmode
To my father.

\end{description}\end{quote}
\\
\hline
\begin{sphinxVerbatimintable}[commandchars=\\\{\}]
\PYG{o}{\PYGZhy{}}\PYG{n}{a}            \PYG{n}{command}\PYG{o}{\PYGZhy{}}\PYG{n}{line} \PYG{n}{option} \PYG{l+s+s2}{\PYGZdq{}}\PYG{l+s+s2}{a}\PYG{l+s+s2}{\PYGZdq{}}
\PYG{o}{\PYGZhy{}}\PYG{n}{b} \PYG{n}{file}       \PYG{n}{options} \PYG{n}{can} \PYG{n}{have} \PYG{n}{arguments}
              \PYG{o+ow}{and} \PYG{n}{long} \PYG{n}{descriptions}
\PYG{o}{\PYGZhy{}}\PYG{o}{\PYGZhy{}}\PYG{n}{long}        \PYG{n}{options} \PYG{n}{can} \PYG{n}{be} \PYG{n}{long} \PYG{n}{also}
\PYG{o}{\PYGZhy{}}\PYG{o}{\PYGZhy{}}\PYG{n+nb}{input}\PYG{o}{=}\PYG{n}{file}  \PYG{n}{long} \PYG{n}{options} \PYG{n}{can} \PYG{n}{also} \PYG{n}{have}
              \PYG{n}{arguments}
\PYG{o}{/}\PYG{n}{V}            \PYG{n}{DOS}\PYG{o}{/}\PYG{n}{VMS}\PYG{o}{\PYGZhy{}}\PYG{n}{style} \PYG{n}{options} \PYG{n}{too}
\end{sphinxVerbatimintable}
&\begin{optionlist}{3cm}
\item [\sphinxhyphen{}a]  
command\sphinxhyphen{}line option “a”
\item [\sphinxhyphen{}b file]  
options can have arguments
and long descriptions
\item [\sphinxhyphen{}\sphinxhyphen{}long]  
options can be long also
\item [\sphinxhyphen{}\sphinxhyphen{}input=file]  
long options can also have
arguments
\item [/V]  
DOS/VMS\sphinxhyphen{}style options too
\end{optionlist}
\\
\hline
\end{tabular}
\par
\sphinxattableend\end{savenotes}




\subsection{Section Structure}
\label{\detokenize{rst-cheatsheet/rst-cheatsheet:section-structure}}

\begin{savenotes}\sphinxattablestart
\centering
\begin{tabular}[t]{|*{2}{\X{1}{2}|}}
\hline

\begin{sphinxVerbatimintable}[commandchars=\\\{\}]
Title
=====

Titles are underlined (or over\PYGZhy{} and underlined) with
a nonalphanumeric character at least as long as the
text.

A lone top\PYGZhy{}level section is lifted up to be the
document\PYGZsq{}s title.

Any non\PYGZhy{}alphanumeric character can be used, but
Python convention is:

* ``\PYGZsh{}`` with overline, for parts
* ``*`` with overline, for chapters
* ``=``, for sections
* ``\PYGZhy{}``, for subsections
* ``\PYGZca{}``, for subsubsections
* ``\PYGZdq{}``, for paragraphs
\end{sphinxVerbatimintable}
&
Title

Titles are underlined (or over\sphinxhyphen{} and underlined) with
a nonalphanumeric character at least as long as the
text.

A lone top\sphinxhyphen{}level section is lifted up to be the
document’s title.

Any non\sphinxhyphen{}alphanumeric character can be used, but
Python convention is:
\begin{itemize}
\item {} 
\sphinxcode{\sphinxupquote{\#}} with overline, for parts

\item {} 
\sphinxcode{\sphinxupquote{*}} with overline, for chapters

\item {} 
\sphinxcode{\sphinxupquote{=}}, for sections

\item {} 
\sphinxcode{\sphinxupquote{\sphinxhyphen{}}}, for subsections

\item {} 
\sphinxcode{\sphinxupquote{\textasciicircum{}}}, for subsubsections

\item {} 
\sphinxcode{\sphinxupquote{"}}, for paragraphs

\end{itemize}
\\
\hline
\end{tabular}
\par
\sphinxattableend\end{savenotes}


\subsection{Blocks}
\label{\detokenize{rst-cheatsheet/rst-cheatsheet:blocks}}

\begin{savenotes}\sphinxattablestart
\centering
\begin{tabular}[t]{|*{2}{\X{1}{2}|}}
\hline

\begin{sphinxVerbatimintable}[commandchars=\\\{\}]
\PYG{n}{This} \PYG{o+ow}{is} \PYG{n}{a} \PYG{n}{paragraph}\PYG{o}{.}

\PYG{n}{Paragraphs} \PYG{n}{line} \PYG{n}{up} \PYG{n}{at} \PYG{n}{their} \PYG{n}{left} \PYG{n}{edges}\PYG{p}{,} \PYG{o+ow}{and} \PYG{n}{are}
\PYG{n}{normally} \PYG{n}{separated} \PYG{n}{by} \PYG{n}{blank} \PYG{n}{lines}\PYG{o}{.}
\end{sphinxVerbatimintable}
&
This is a paragraph.

Paragraphs line up at their left
edges, and are normally separated
by blank lines.
\\
\hline
\begin{sphinxVerbatimintable}[commandchars=\\\{\}]
A paragraph containing only two colons indicates
the following indented or quoted text is a literal
block or quoted text is a literal block.

::

  Whitespace, newlines, blank lines, and  all kinds of
  markup (like *this* or \PYGZbs{}this) is preserved here.

You can also tack the ``::`` at the end of a
paragraph::

   It\PYGZsq{}s very convenient to use this form.

Per\PYGZhy{}line quoting can also be used for unindented
blocks::

\PYGZgt{} Useful for quotes from email and
\PYGZgt{} for Haskell literate programming.
\end{sphinxVerbatimintable}
&
A paragraph containing only two colons
indicates that the following indented
or quoted text is a literal block.

\begin{sphinxVerbatimintable}[commandchars=\\\{\}]
\PYG{n}{Whitespace}\PYG{p}{,} \PYG{n}{newlines}\PYG{p}{,} \PYG{n}{blank} \PYG{n}{lines}\PYG{p}{,} \PYG{o+ow}{and}
\PYG{n+nb}{all} \PYG{n}{kinds} \PYG{n}{of} \PYG{n}{markup} \PYG{p}{(}\PYG{n}{like} \PYG{o}{*}\PYG{n}{this}\PYG{o}{*} \PYG{o+ow}{or}
\PYGZbs{}\PYG{n}{this}\PYG{p}{)} \PYG{o+ow}{is} \PYG{n}{preserved} \PYG{n}{by} \PYG{n}{literal} \PYG{n}{blocks}\PYG{o}{.}
\end{sphinxVerbatimintable}

You can also tack the \sphinxcode{\sphinxupquote{::}} at the end of a
paragraph:

\begin{sphinxVerbatimintable}[commandchars=\\\{\}]
\PYG{n}{It}\PYG{l+s+s1}{\PYGZsq{}}\PYG{l+s+s1}{s very convenient to use this form.}
\end{sphinxVerbatimintable}

Per\sphinxhyphen{}line quoting can also be used for
unindented blocks:

\begin{sphinxVerbatimintable}[commandchars=\\\{\}]
\PYG{o}{\PYGZgt{}} \PYG{n}{Useful} \PYG{k}{for} \PYG{n}{quotes} \PYG{k+kn}{from} \PYG{n+nn}{email} \PYG{o+ow}{and}
\PYG{o}{\PYGZgt{}} \PYG{k}{for} \PYG{n}{Haskell} \PYG{n}{literate} \PYG{n}{programming}\PYG{o}{.}
\end{sphinxVerbatimintable}
\\
\hline
\begin{sphinxVerbatimintable}[commandchars=\\\{\}]
\PYG{o}{|} \PYG{n}{Line} \PYG{n}{blocks} \PYG{n}{are} \PYG{n}{useful} \PYG{k}{for} \PYG{n}{addresses}\PYG{p}{,}
\PYG{o}{|} \PYG{n}{verse}\PYG{p}{,} \PYG{o+ow}{and} \PYG{n}{adornment}\PYG{o}{\PYGZhy{}}\PYG{n}{free} \PYG{n}{lists}\PYG{o}{.}
\PYG{o}{|}
\PYG{o}{|} \PYG{n}{Each} \PYG{n}{new} \PYG{n}{line} \PYG{n}{begins} \PYG{k}{with} \PYG{n}{a}
\PYG{o}{|} \PYG{n}{vertical} \PYG{n}{bar} \PYG{p}{(}\PYG{l+s+s2}{\PYGZdq{}}\PYG{l+s+s2}{|}\PYG{l+s+s2}{\PYGZdq{}}\PYG{p}{)}\PYG{o}{.}
\PYG{o}{|}     \PYG{n}{Line} \PYG{n}{breaks} \PYG{o+ow}{and} \PYG{n}{initial} \PYG{n}{indents}
\PYG{o}{|}     \PYG{n}{are} \PYG{n}{preserved}\PYG{o}{.}
\PYG{o}{|} \PYG{n}{Continuation} \PYG{n}{lines} \PYG{n}{are} \PYG{n}{wrapped}
  \PYG{n}{portions} \PYG{n}{of} \PYG{n}{long} \PYG{n}{lines}\PYG{p}{;} \PYG{n}{they} \PYG{n}{begin}
  \PYG{k}{with} \PYG{n}{spaces} \PYG{o+ow}{in} \PYG{n}{place} \PYG{n}{of} \PYG{n}{vertical} \PYG{n}{bars}\PYG{o}{.}
\end{sphinxVerbatimintable}
&
\begin{DUlineblock}{0em}
\item[] Line blocks are useful for addresses,
\item[] verse, and adornment\sphinxhyphen{}free lists.
\item[] 
\item[] Each new line begins with a
\item[] vertical bar (“|”).
\item[]
\begin{DUlineblock}{\DUlineblockindent}
\item[] Line breaks and initial indents
\item[] are preserved.
\end{DUlineblock}
\item[] Continuation lines are wrapped
portions of long lines; they begin
with spaces in place of vertical bars.
\end{DUlineblock}
\\
\hline
\begin{sphinxVerbatimintable}[commandchars=\\\{\}]
\PYG{n}{Block} \PYG{n}{quotes} \PYG{n}{are} \PYG{n}{just}\PYG{p}{:}

    \PYG{n}{Indented} \PYG{n}{paragraphs}\PYG{p}{,}

        \PYG{o+ow}{and} \PYG{n}{they} \PYG{n}{may} \PYG{n}{nest}\PYG{o}{.}
\end{sphinxVerbatimintable}
&
Block quotes are just:
\begin{quote}

Indented paragraphs,
\begin{quote}

and they may nest.
\end{quote}
\end{quote}
\\
\hline
\begin{sphinxVerbatimintable}[commandchars=\\\{\}]
\PYG{n}{Doctest} \PYG{n}{blocks} \PYG{n}{are} \PYG{n}{interactive}
\PYG{n}{Python} \PYG{n}{sessions}\PYG{o}{.} \PYG{n}{They} \PYG{n}{begin} \PYG{k}{with}
\PYG{l+s+s2}{\PYGZdq{}}\PYG{l+s+s2}{``\PYGZgt{}\PYGZgt{}\PYGZgt{}``}\PYG{l+s+s2}{\PYGZdq{}} \PYG{o+ow}{and} \PYG{n}{end} \PYG{k}{with} \PYG{n}{a} \PYG{n}{blank} \PYG{n}{line}\PYG{o}{.}

\PYG{o}{\PYGZgt{}\PYGZgt{}}\PYG{o}{\PYGZgt{}} \PYG{n+nb}{print} \PYG{l+s+s2}{\PYGZdq{}}\PYG{l+s+s2}{This is a doctest block.}\PYG{l+s+s2}{\PYGZdq{}}
\PYG{n}{This} \PYG{o+ow}{is} \PYG{n}{a} \PYG{n}{doctest} \PYG{n}{block}\PYG{o}{.}
\end{sphinxVerbatimintable}
&
Doctest blocks are interactive
Python sessions. They begin with
“\sphinxcode{\sphinxupquote{>>>}}” and end with a blank line.

\begin{sphinxVerbatimintable}[commandchars=\\\{\}]
\PYG{g+gp}{\PYGZgt{}\PYGZgt{}\PYGZgt{} }\PYG{n+nb}{print} \PYG{l+s+s2}{\PYGZdq{}}\PYG{l+s+s2}{This is a doctest block.}\PYG{l+s+s2}{\PYGZdq{}}
\PYG{g+go}{This is a doctest block.}
\end{sphinxVerbatimintable}
\\
\hline
\begin{sphinxVerbatimintable}[commandchars=\\\{\}]
\PYG{n}{A} \PYG{n}{transition} \PYG{n}{marker} \PYG{o+ow}{is} \PYG{n}{a} \PYG{n}{horizontal} \PYG{n}{line}
\PYG{n}{of} \PYG{l+m+mi}{4} \PYG{o+ow}{or} \PYG{n}{more} \PYG{n}{repeated} \PYG{n}{punctuation}
\PYG{n}{characters}\PYG{o}{.}

\PYG{o}{\PYGZhy{}}\PYG{o}{\PYGZhy{}}\PYG{o}{\PYGZhy{}}\PYG{o}{\PYGZhy{}}\PYG{o}{\PYGZhy{}}\PYG{o}{\PYGZhy{}}\PYG{o}{\PYGZhy{}}\PYG{o}{\PYGZhy{}}\PYG{o}{\PYGZhy{}}\PYG{o}{\PYGZhy{}}\PYG{o}{\PYGZhy{}}\PYG{o}{\PYGZhy{}}

\PYG{n}{A} \PYG{n}{transition} \PYG{n}{should} \PYG{o+ow}{not} \PYG{n}{begin} \PYG{o+ow}{or} \PYG{n}{end} \PYG{n}{a}
\PYG{n}{section} \PYG{o+ow}{or} \PYG{n}{document}\PYG{p}{,} \PYG{n}{nor} \PYG{n}{should} \PYG{n}{two}
\PYG{n}{transitions} \PYG{n}{be} \PYG{n}{immediately} \PYG{n}{adjacent}\PYG{o}{.}
\end{sphinxVerbatimintable}
&
A transition marker is a horizontal line
of 4 or more repeated punctuation
characters.


\begin{savenotes}\sphinxattablestart
\centering
\begin{tabulary}{\linewidth}[t]{|T|}
\hline
\\
\hline
\end{tabulary}
\par
\sphinxattableend\end{savenotes}

A transition should not begin or end a
section or document, nor should two
transitions be immediately adjacent.
\\
\hline
\end{tabular}
\par
\sphinxattableend\end{savenotes}




\subsection{Tables}
\label{\detokenize{rst-cheatsheet/rst-cheatsheet:tables}}
There are two syntaxes for tables in reStructuredText. Grid tables are complete but cumbersome to create. Simple
tables are easy to create but limited (no row spans, etc.).


\begin{savenotes}\sphinxattablestart
\centering
\begin{tabular}[t]{|*{2}{\X{1}{2}|}}
\hline

\begin{sphinxVerbatimintable}[commandchars=\\\{\}]
\PYG{o}{+}\PYG{o}{\PYGZhy{}}\PYG{o}{\PYGZhy{}}\PYG{o}{\PYGZhy{}}\PYG{o}{\PYGZhy{}}\PYG{o}{\PYGZhy{}}\PYG{o}{\PYGZhy{}}\PYG{o}{\PYGZhy{}}\PYG{o}{\PYGZhy{}}\PYG{o}{\PYGZhy{}}\PYG{o}{\PYGZhy{}}\PYG{o}{\PYGZhy{}}\PYG{o}{\PYGZhy{}}\PYG{o}{+}\PYG{o}{\PYGZhy{}}\PYG{o}{\PYGZhy{}}\PYG{o}{\PYGZhy{}}\PYG{o}{\PYGZhy{}}\PYG{o}{\PYGZhy{}}\PYG{o}{\PYGZhy{}}\PYG{o}{\PYGZhy{}}\PYG{o}{\PYGZhy{}}\PYG{o}{\PYGZhy{}}\PYG{o}{\PYGZhy{}}\PYG{o}{\PYGZhy{}}\PYG{o}{\PYGZhy{}}\PYG{o}{+}\PYG{o}{\PYGZhy{}}\PYG{o}{\PYGZhy{}}\PYG{o}{\PYGZhy{}}\PYG{o}{\PYGZhy{}}\PYG{o}{\PYGZhy{}}\PYG{o}{\PYGZhy{}}\PYG{o}{\PYGZhy{}}\PYG{o}{\PYGZhy{}}\PYG{o}{\PYGZhy{}}\PYG{o}{\PYGZhy{}}\PYG{o}{\PYGZhy{}}\PYG{o}{+}
\PYG{o}{|} \PYG{n}{Header} \PYG{l+m+mi}{1}   \PYG{o}{|} \PYG{n}{Header} \PYG{l+m+mi}{2}   \PYG{o}{|} \PYG{n}{Header} \PYG{l+m+mi}{3}  \PYG{o}{|}
\PYG{o}{+}\PYG{o}{==}\PYG{o}{==}\PYG{o}{==}\PYG{o}{==}\PYG{o}{==}\PYG{o}{==}\PYG{o}{+}\PYG{o}{==}\PYG{o}{==}\PYG{o}{==}\PYG{o}{==}\PYG{o}{==}\PYG{o}{==}\PYG{o}{+}\PYG{o}{==}\PYG{o}{==}\PYG{o}{==}\PYG{o}{==}\PYG{o}{==}\PYG{o}{=}\PYG{o}{+}
\PYG{o}{|} \PYG{n}{body} \PYG{n}{row} \PYG{l+m+mi}{1} \PYG{o}{|} \PYG{n}{column} \PYG{l+m+mi}{2}   \PYG{o}{|} \PYG{n}{column} \PYG{l+m+mi}{3}  \PYG{o}{|}
\PYG{o}{+}\PYG{o}{\PYGZhy{}}\PYG{o}{\PYGZhy{}}\PYG{o}{\PYGZhy{}}\PYG{o}{\PYGZhy{}}\PYG{o}{\PYGZhy{}}\PYG{o}{\PYGZhy{}}\PYG{o}{\PYGZhy{}}\PYG{o}{\PYGZhy{}}\PYG{o}{\PYGZhy{}}\PYG{o}{\PYGZhy{}}\PYG{o}{\PYGZhy{}}\PYG{o}{\PYGZhy{}}\PYG{o}{+}\PYG{o}{\PYGZhy{}}\PYG{o}{\PYGZhy{}}\PYG{o}{\PYGZhy{}}\PYG{o}{\PYGZhy{}}\PYG{o}{\PYGZhy{}}\PYG{o}{\PYGZhy{}}\PYG{o}{\PYGZhy{}}\PYG{o}{\PYGZhy{}}\PYG{o}{\PYGZhy{}}\PYG{o}{\PYGZhy{}}\PYG{o}{\PYGZhy{}}\PYG{o}{\PYGZhy{}}\PYG{o}{+}\PYG{o}{\PYGZhy{}}\PYG{o}{\PYGZhy{}}\PYG{o}{\PYGZhy{}}\PYG{o}{\PYGZhy{}}\PYG{o}{\PYGZhy{}}\PYG{o}{\PYGZhy{}}\PYG{o}{\PYGZhy{}}\PYG{o}{\PYGZhy{}}\PYG{o}{\PYGZhy{}}\PYG{o}{\PYGZhy{}}\PYG{o}{\PYGZhy{}}\PYG{o}{+}
\PYG{o}{|} \PYG{n}{body} \PYG{n}{row} \PYG{l+m+mi}{2} \PYG{o}{|} \PYG{n}{Cells} \PYG{n}{may} \PYG{n}{span} \PYG{n}{columns}\PYG{o}{.}\PYG{o}{|}
\PYG{o}{+}\PYG{o}{\PYGZhy{}}\PYG{o}{\PYGZhy{}}\PYG{o}{\PYGZhy{}}\PYG{o}{\PYGZhy{}}\PYG{o}{\PYGZhy{}}\PYG{o}{\PYGZhy{}}\PYG{o}{\PYGZhy{}}\PYG{o}{\PYGZhy{}}\PYG{o}{\PYGZhy{}}\PYG{o}{\PYGZhy{}}\PYG{o}{\PYGZhy{}}\PYG{o}{\PYGZhy{}}\PYG{o}{+}\PYG{o}{\PYGZhy{}}\PYG{o}{\PYGZhy{}}\PYG{o}{\PYGZhy{}}\PYG{o}{\PYGZhy{}}\PYG{o}{\PYGZhy{}}\PYG{o}{\PYGZhy{}}\PYG{o}{\PYGZhy{}}\PYG{o}{\PYGZhy{}}\PYG{o}{\PYGZhy{}}\PYG{o}{\PYGZhy{}}\PYG{o}{\PYGZhy{}}\PYG{o}{\PYGZhy{}}\PYG{o}{+}\PYG{o}{\PYGZhy{}}\PYG{o}{\PYGZhy{}}\PYG{o}{\PYGZhy{}}\PYG{o}{\PYGZhy{}}\PYG{o}{\PYGZhy{}}\PYG{o}{\PYGZhy{}}\PYG{o}{\PYGZhy{}}\PYG{o}{\PYGZhy{}}\PYG{o}{\PYGZhy{}}\PYG{o}{\PYGZhy{}}\PYG{o}{\PYGZhy{}}\PYG{o}{+}
\PYG{o}{|} \PYG{n}{body} \PYG{n}{row} \PYG{l+m+mi}{3} \PYG{o}{|} \PYG{n}{Cells} \PYG{n}{may}  \PYG{o}{|} \PYG{o}{\PYGZhy{}} \PYG{n}{Cells}   \PYG{o}{|}
\PYG{o}{+}\PYG{o}{\PYGZhy{}}\PYG{o}{\PYGZhy{}}\PYG{o}{\PYGZhy{}}\PYG{o}{\PYGZhy{}}\PYG{o}{\PYGZhy{}}\PYG{o}{\PYGZhy{}}\PYG{o}{\PYGZhy{}}\PYG{o}{\PYGZhy{}}\PYG{o}{\PYGZhy{}}\PYG{o}{\PYGZhy{}}\PYG{o}{\PYGZhy{}}\PYG{o}{\PYGZhy{}}\PYG{o}{+} \PYG{n}{span} \PYG{n}{rows}\PYG{o}{.} \PYG{o}{|} \PYG{o}{\PYGZhy{}} \PYG{n}{contain} \PYG{o}{|}
\PYG{o}{|} \PYG{n}{body} \PYG{n}{row} \PYG{l+m+mi}{4} \PYG{o}{|}            \PYG{o}{|} \PYG{o}{\PYGZhy{}} \PYG{n}{blocks}\PYG{o}{.} \PYG{o}{|}
\PYG{o}{+}\PYG{o}{\PYGZhy{}}\PYG{o}{\PYGZhy{}}\PYG{o}{\PYGZhy{}}\PYG{o}{\PYGZhy{}}\PYG{o}{\PYGZhy{}}\PYG{o}{\PYGZhy{}}\PYG{o}{\PYGZhy{}}\PYG{o}{\PYGZhy{}}\PYG{o}{\PYGZhy{}}\PYG{o}{\PYGZhy{}}\PYG{o}{\PYGZhy{}}\PYG{o}{\PYGZhy{}}\PYG{o}{+}\PYG{o}{\PYGZhy{}}\PYG{o}{\PYGZhy{}}\PYG{o}{\PYGZhy{}}\PYG{o}{\PYGZhy{}}\PYG{o}{\PYGZhy{}}\PYG{o}{\PYGZhy{}}\PYG{o}{\PYGZhy{}}\PYG{o}{\PYGZhy{}}\PYG{o}{\PYGZhy{}}\PYG{o}{\PYGZhy{}}\PYG{o}{\PYGZhy{}}\PYG{o}{\PYGZhy{}}\PYG{o}{+}\PYG{o}{\PYGZhy{}}\PYG{o}{\PYGZhy{}}\PYG{o}{\PYGZhy{}}\PYG{o}{\PYGZhy{}}\PYG{o}{\PYGZhy{}}\PYG{o}{\PYGZhy{}}\PYG{o}{\PYGZhy{}}\PYG{o}{\PYGZhy{}}\PYG{o}{\PYGZhy{}}\PYG{o}{\PYGZhy{}}\PYG{o}{\PYGZhy{}}\PYG{o}{+}
\end{sphinxVerbatimintable}
&

\begin{savenotes}\sphinxattablestart
\centering
\begin{tabular}[t]{|*{3}{\X{1}{3}|}}
\hline
\sphinxstyletheadfamily 
Header 1
&\sphinxstyletheadfamily 
Header 2
&\sphinxstyletheadfamily 
Header 3
\\
\hline
body row 1
&
column 2
&
column 3
\\
\hline
body row 2
&\sphinxstartmulticolumn{2}%
\begin{varwidth}[t]{\sphinxcolwidth{2}{3}}
Cells may span columns.
\par
\vskip-\baselineskip\vbox{\hbox{\strut}}\end{varwidth}%
\sphinxstopmulticolumn
\\
\hline
body row 3
&\sphinxmultirow{2}{10}{%
\begin{varwidth}[t]{\sphinxcolwidth{1}{3}}
Cells may
span rows.
\par
\vskip-\baselineskip\vbox{\hbox{\strut}}\end{varwidth}%
}%
&\sphinxmultirow{2}{11}{%
\begin{varwidth}[t]{\sphinxcolwidth{1}{3}}
\begin{itemize}
\item {} 
Cells

\item {} 
contain

\item {} 
blocks.

\end{itemize}
\par
\vskip-\baselineskip\vbox{\hbox{\strut}}\end{varwidth}%
}%
\\
\cline{1-1}
body row 4
&\sphinxtablestrut{10}&\sphinxtablestrut{11}\\
\hline
\end{tabular}
\par
\sphinxattableend\end{savenotes}
\\
\hline
\begin{sphinxVerbatimintable}[commandchars=\\\{\}]
\PYG{o}{==}\PYG{o}{==}\PYG{o}{=}  \PYG{o}{==}\PYG{o}{==}\PYG{o}{=}  \PYG{o}{==}\PYG{o}{==}\PYG{o}{==}
   \PYG{n}{Inputs}     \PYG{n}{Output}
\PYG{o}{\PYGZhy{}}\PYG{o}{\PYGZhy{}}\PYG{o}{\PYGZhy{}}\PYG{o}{\PYGZhy{}}\PYG{o}{\PYGZhy{}}\PYG{o}{\PYGZhy{}}\PYG{o}{\PYGZhy{}}\PYG{o}{\PYGZhy{}}\PYG{o}{\PYGZhy{}}\PYG{o}{\PYGZhy{}}\PYG{o}{\PYGZhy{}}\PYG{o}{\PYGZhy{}}  \PYG{o}{\PYGZhy{}}\PYG{o}{\PYGZhy{}}\PYG{o}{\PYGZhy{}}\PYG{o}{\PYGZhy{}}\PYG{o}{\PYGZhy{}}\PYG{o}{\PYGZhy{}}
  \PYG{n}{A}      \PYG{n}{B}    \PYG{n}{A} \PYG{o+ow}{or} \PYG{n}{B}
\PYG{o}{==}\PYG{o}{==}\PYG{o}{=}  \PYG{o}{==}\PYG{o}{==}\PYG{o}{=}  \PYG{o}{==}\PYG{o}{==}\PYG{o}{==}
\PYG{k+kc}{False}  \PYG{k+kc}{False}  \PYG{k+kc}{False}
\PYG{k+kc}{True}   \PYG{k+kc}{False}  \PYG{k+kc}{True}
\PYG{k+kc}{False}  \PYG{k+kc}{True}   \PYG{k+kc}{True}
\PYG{k+kc}{True}   \PYG{k+kc}{True}   \PYG{k+kc}{True}
\PYG{o}{==}\PYG{o}{==}\PYG{o}{=}  \PYG{o}{==}\PYG{o}{==}\PYG{o}{=}  \PYG{o}{==}\PYG{o}{==}\PYG{o}{==}
\end{sphinxVerbatimintable}
&

\begin{savenotes}\sphinxattablestart
\centering
\begin{tabulary}{\linewidth}[t]{|T|T|T|}
\hline
\sphinxstartmulticolumn{2}%
\begin{varwidth}[t]{\sphinxcolwidth{2}{3}}
\sphinxstyletheadfamily Inputs
\par
\vskip-\baselineskip\vbox{\hbox{\strut}}\end{varwidth}%
\sphinxstopmulticolumn
&\sphinxstyletheadfamily 
Output
\\
\hline\sphinxstyletheadfamily 
A
&\sphinxstyletheadfamily 
B
&\sphinxstyletheadfamily 
A or B
\\
\hline
False
&
False
&
False
\\
\hline
True
&
False
&
True
\\
\hline
False
&
True
&
True
\\
\hline
True
&
True
&
True
\\
\hline
\end{tabulary}
\par
\sphinxattableend\end{savenotes}
\\
\hline
\end{tabular}
\par
\sphinxattableend\end{savenotes}


\subsection{Explicit Markup}
\label{\detokenize{rst-cheatsheet/rst-cheatsheet:explicit-markup}}
Explicit markup blocks are used for constructs which float (footnotes), have no direct paper\sphinxhyphen{}document representation
(hyperlink targets, comments), or require specialized processing (directives).
They all begin with two periods and whitespace, the “explicit markup start”.


\begin{savenotes}\sphinxattablestart
\centering
\begin{tabular}[t]{|*{2}{\X{1}{2}|}}
\hline

\begin{sphinxVerbatimintable}[commandchars=\\\{\}]
\PYG{n}{Footnote} \PYG{n}{references}\PYG{p}{,} \PYG{n}{like} \PYG{p}{[}\PYG{l+m+mi}{5}\PYG{p}{]}\PYG{n}{\PYGZus{}}\PYG{o}{.}
\PYG{n}{Note} \PYG{n}{that} \PYG{n}{footnotes} \PYG{n}{may} \PYG{n}{get}
\PYG{n}{rearranged}\PYG{p}{,} \PYG{n}{e}\PYG{o}{.}\PYG{n}{g}\PYG{o}{.}\PYG{p}{,} \PYG{n}{to} \PYG{n}{the} \PYG{n}{bottom} \PYG{n}{of}
\PYG{n}{the} \PYG{l+s+s2}{\PYGZdq{}}\PYG{l+s+s2}{page}\PYG{l+s+s2}{\PYGZdq{}}\PYG{o}{.}

\PYG{o}{.}\PYG{o}{.} \PYG{p}{[}\PYG{l+m+mi}{5}\PYG{p}{]} \PYG{n}{A} \PYG{n}{numerical} \PYG{n}{footnote}\PYG{o}{.} \PYG{n}{Note}
   \PYG{n}{there}\PYG{l+s+s1}{\PYGZsq{}}\PYG{l+s+s1}{s no colon after the ``]``.}
\end{sphinxVerbatimintable}
&
Footnote references, like %
\begin{footnote}[5]\sphinxAtStartFootnote
A numerical footnote. Note
there’s no colon after the \sphinxcode{\sphinxupquote{{]}}}.
%
\end{footnote}.
Note that footnotes may get
rearranged, e.g., to the bottom of
the “page”.
\\
\hline
\begin{sphinxVerbatimintable}[commandchars=\\\{\}]
\PYG{n}{Autonumbered} \PYG{n}{footnotes} \PYG{n}{are}
\PYG{n}{possible}\PYG{p}{,} \PYG{n}{like} \PYG{n}{using} \PYG{p}{[}\PYG{c+c1}{\PYGZsh{}]\PYGZus{} and [\PYGZsh{}]\PYGZus{}.}

\PYG{o}{.}\PYG{o}{.} \PYG{p}{[}\PYG{c+c1}{\PYGZsh{}] This is the first one.}
\PYG{o}{.}\PYG{o}{.} \PYG{p}{[}\PYG{c+c1}{\PYGZsh{}] This is the second one.}

\PYG{n}{They} \PYG{n}{may} \PYG{n}{be} \PYG{n}{assigned} \PYG{l+s+s1}{\PYGZsq{}}\PYG{l+s+s1}{autonumber}
\PYG{n}{labels}\PYG{l+s+s1}{\PYGZsq{}}\PYG{l+s+s1}{ \PYGZhy{} for instance,}
\PYG{p}{[}\PYG{c+c1}{\PYGZsh{}fourth]\PYGZus{} and [\PYGZsh{}third]\PYGZus{}.}

\PYG{o}{.}\PYG{o}{.} \PYG{p}{[}\PYG{c+c1}{\PYGZsh{}third] a.k.a. third\PYGZus{}}

\PYG{o}{.}\PYG{o}{.} \PYG{p}{[}\PYG{c+c1}{\PYGZsh{}fourth] a.k.a. fourth\PYGZus{}}
\end{sphinxVerbatimintable}
&
Autonumbered footnotes are
possible, like using \sphinxfootnotemark[13] and %
\begin{footnote}[14]\sphinxAtStartFootnote
This is the second one.
%
\end{footnote}.

They may be assigned ‘autonumber
labels’ \sphinxhyphen{} for instance,
%
\begin{footnote}[16]\sphinxAtStartFootnote
a.k.a. {\hyperref[\detokenize{rst-cheatsheet/rst-cheatsheet:fourth}]{\sphinxcrossref{fourth}}}
%
\end{footnote} and %
\begin{footnote}[15]\sphinxAtStartFootnote
a.k.a. {\hyperref[\detokenize{rst-cheatsheet/rst-cheatsheet:third}]{\sphinxcrossref{third}}}
%
\end{footnote}.
\\
\hline
\begin{sphinxVerbatimintable}[commandchars=\\\{\}]
\PYG{n}{Auto}\PYG{o}{\PYGZhy{}}\PYG{n}{symbol} \PYG{n}{footnotes} \PYG{n}{are} \PYG{n}{also}
\PYG{n}{possible}\PYG{p}{,} \PYG{n}{like} \PYG{n}{this}\PYG{p}{:} \PYG{p}{[}\PYG{o}{*}\PYG{p}{]}\PYG{n}{\PYGZus{}} \PYG{o+ow}{and} \PYG{p}{[}\PYG{o}{*}\PYG{p}{]}\PYG{n}{\PYGZus{}}\PYG{o}{.}

\PYG{o}{.}\PYG{o}{.} \PYG{p}{[}\PYG{o}{*}\PYG{p}{]} \PYG{n}{This} \PYG{o+ow}{is} \PYG{n}{the} \PYG{n}{first} \PYG{n}{one}\PYG{o}{.}
\PYG{o}{.}\PYG{o}{.} \PYG{p}{[}\PYG{o}{*}\PYG{p}{]} \PYG{n}{This} \PYG{o+ow}{is} \PYG{n}{the} \PYG{n}{second} \PYG{n}{one}\PYG{o}{.}
\end{sphinxVerbatimintable}
&
Auto\sphinxhyphen{}symbol footnotes are also
possible, like this: %
\begin{footnote}[17]\sphinxAtStartFootnote
This is the first one.
%
\end{footnote} and %
\begin{footnote}[18]\sphinxAtStartFootnote
This is the second one.
%
\end{footnote}.
\\
\hline
\begin{sphinxVerbatimintable}[commandchars=\\\{\}]
\PYG{n}{Citation} \PYG{n}{references}\PYG{p}{,} \PYG{n}{like} \PYG{p}{[}\PYG{n}{CIT2002}\PYG{p}{]}\PYG{n}{\PYGZus{}}\PYG{o}{.}
\PYG{n}{Note} \PYG{n}{that} \PYG{n}{citations} \PYG{n}{may} \PYG{n}{get}
\PYG{n}{rearranged}\PYG{p}{,} \PYG{n}{e}\PYG{o}{.}\PYG{n}{g}\PYG{o}{.}\PYG{p}{,} \PYG{n}{to} \PYG{n}{the} \PYG{n}{bottom} \PYG{n}{of}
\PYG{n}{the} \PYG{l+s+s2}{\PYGZdq{}}\PYG{l+s+s2}{page}\PYG{l+s+s2}{\PYGZdq{}}\PYG{o}{.}

\PYG{o}{.}\PYG{o}{.} \PYG{p}{[}\PYG{n}{CIT2002}\PYG{p}{]} \PYG{n}{A} \PYG{n}{citation}
   \PYG{p}{(}\PYG{k}{as} \PYG{n}{often} \PYG{n}{used} \PYG{o+ow}{in} \PYG{n}{journals}\PYG{p}{)}\PYG{o}{.}

\PYG{n}{Citation} \PYG{n}{labels} \PYG{n}{contain} \PYG{n}{alphanumerics}\PYG{p}{,}
\PYG{n}{underlines}\PYG{p}{,} \PYG{n}{hyphens} \PYG{o+ow}{and} \PYG{n}{fullstops}\PYG{o}{.}
\PYG{n}{Case} \PYG{o+ow}{is} \PYG{o+ow}{not} \PYG{n}{significant}\PYG{o}{.}

\PYG{n}{Given} \PYG{n}{a} \PYG{n}{citation} \PYG{n}{like} \PYG{p}{[}\PYG{n}{this}\PYG{p}{]}\PYG{n}{\PYGZus{}}\PYG{p}{,} \PYG{n}{one}
\PYG{n}{can} \PYG{n}{also} \PYG{n}{refer} \PYG{n}{to} \PYG{n}{it} \PYG{n}{like} \PYG{n}{this\PYGZus{}}\PYG{o}{.}

\PYG{o}{.}\PYG{o}{.} \PYG{p}{[}\PYG{n}{this}\PYG{p}{]} \PYG{n}{here}\PYG{o}{.}
\end{sphinxVerbatimintable}
&
Citation references, like \sphinxcite{rst-cheatsheet/rst-cheatsheet:cit2002}.
Note that citations may get
rearranged, e.g., to the bottom of
the “page”.

Citation labels contain alphanumerics,
underlines, hyphens and fullstops.
Case is not significant.

Given a citation like \sphinxcite{rst-cheatsheet/rst-cheatsheet:this}, one
can also refer to it like {\hyperref[\detokenize{rst-cheatsheet/rst-cheatsheet:this}]{\sphinxcrossref{this}}}.
\\
\hline
\begin{sphinxVerbatimintable}[commandchars=\\\{\}]
\PYG{n}{External} \PYG{n}{hyperlinks}\PYG{p}{,} \PYG{n}{like} \PYG{n}{Python\PYGZus{}}\PYG{o}{.}

\PYG{o}{.}\PYG{o}{.} \PYG{n}{\PYGZus{}Python}\PYG{p}{:} \PYG{n}{http}\PYG{p}{:}\PYG{o}{/}\PYG{o}{/}\PYG{n}{www}\PYG{o}{.}\PYG{n}{python}\PYG{o}{.}\PYG{n}{org}\PYG{o}{/}
\end{sphinxVerbatimintable}
&
External hyperlinks, like \sphinxhref{http://www.python.org/}{Python}%
\begin{footnote}[19]\sphinxAtStartFootnote
\sphinxnolinkurl{http://www.python.org/}
%
\end{footnote}.
\\
\hline
\begin{sphinxVerbatimintable}[commandchars=\\\{\}]
External hyperlinks, like `Python
\PYGZlt{}http://www.python.org/\PYGZgt{}`\PYGZus{}.
\end{sphinxVerbatimintable}
&
External hyperlinks, like \sphinxhref{http://www.python.org/}{Python}%
\begin{footnote}[20]\sphinxAtStartFootnote
\sphinxnolinkurl{http://www.python.org/}
%
\end{footnote}.
\\
\hline
\begin{sphinxVerbatimintable}[commandchars=\\\{\}]
\PYG{n}{Internal} \PYG{n}{crossreferences}\PYG{p}{,} \PYG{n}{like} \PYG{n}{example\PYGZus{}}\PYG{o}{.}

\PYG{o}{.}\PYG{o}{.} \PYG{n}{\PYGZus{}example}\PYG{p}{:}

\PYG{n}{This} \PYG{o+ow}{is} \PYG{n}{an} \PYG{n}{example} \PYG{n}{crossreference} \PYG{n}{target}\PYG{o}{.}
\end{sphinxVerbatimintable}
&
Internal crossreferences, like {\hyperref[\detokenize{rst-cheatsheet/rst-cheatsheet:example}]{\sphinxcrossref{example}}}.

\phantomsection\label{\detokenize{rst-cheatsheet/rst-cheatsheet:example}}
This is an example crossreference target.
\\
\hline
\begin{sphinxVerbatimintable}[commandchars=\\\{\}]
Python\PYGZus{} is `my favourite
programming language`\PYGZus{}\PYGZus{}.

.. \PYGZus{}Python: http://www.python.org/

\PYGZus{}\PYGZus{} Python\PYGZus{}
\end{sphinxVerbatimintable}
&
\sphinxhref{http://www.python.org/}{Python}%
\begin{footnote}[21]\sphinxAtStartFootnote
\sphinxnolinkurl{http://www.python.org/}
%
\end{footnote} is \sphinxhref{http://www.python.org/}{my favourite
programming language}%
\begin{footnote}[22]\sphinxAtStartFootnote
\sphinxnolinkurl{http://www.python.org/}
%
\end{footnote}.
\\
\hline
\begin{sphinxVerbatimintable}[commandchars=\\\{\}]
Titles are targets, too
=======================

Implict references, like `Titles are targets, too`\PYGZus{}.
\end{sphinxVerbatimintable}
&\phantomsection\label{\detokenize{rst-cheatsheet/rst-cheatsheet:titles-are-targets-too}}
Titles are targets, too

Implict references, like
{\hyperref[\detokenize{rst-cheatsheet/rst-cheatsheet:titles-are-targets-too}]{\sphinxcrossref{Titles are targets, too}}}.
\\
\hline\sphinxstartmulticolumn{2}%
\begin{varwidth}[t]{\sphinxcolwidth{2}{2}}
Directives are a general\sphinxhyphen{}purpose extension mechanism, a way of adding support for new constructs without adding
new syntax. For a description of all standard directives, see reStructuredText Directives (\sphinxurl{http://is.gd/2Ecqh}).
\par
\vskip-\baselineskip\vbox{\hbox{\strut}}\end{varwidth}%
\sphinxstopmulticolumn
\\
\hline
\begin{sphinxVerbatimintable}[commandchars=\\\{\}]
\PYG{n}{For} \PYG{n}{instance}\PYG{p}{:}

\PYG{o}{.}\PYG{o}{.} \PYG{n}{image}\PYG{p}{:}\PYG{p}{:} \PYG{n}{magnetic}\PYG{o}{\PYGZhy{}}\PYG{n}{balls}\PYG{o}{.}\PYG{n}{jpg}
   \PYG{p}{:}\PYG{n}{width}\PYG{p}{:} \PYG{l+m+mi}{40}\PYG{n}{pt}
\end{sphinxVerbatimintable}
&
For instance:

\noindent\sphinxincludegraphics[width=40bp]{{magnetic-balls}.jpg}
\\
\hline\sphinxstartmulticolumn{2}%
\begin{varwidth}[t]{\sphinxcolwidth{2}{2}}
Substitutions are like inline directives, allowing graphics and arbitrary constructs within text.
\par
\vskip-\baselineskip\vbox{\hbox{\strut}}\end{varwidth}%
\sphinxstopmulticolumn
\\
\hline
\begin{sphinxVerbatimintable}[commandchars=\\\{\}]
\PYG{n}{The} \PYG{o}{|}\PYG{n}{biohazard}\PYG{o}{|} \PYG{n}{symbol} \PYG{n}{must} \PYG{n}{be} \PYG{n}{used} \PYG{n}{on} \PYG{n}{containers} \PYG{n}{used} \PYG{n}{to}
\PYG{n}{dispose} \PYG{n}{of} \PYG{n}{medical} \PYG{n}{waste}\PYG{o}{.}

\PYG{o}{.}\PYG{o}{.} \PYG{o}{|}\PYG{n}{biohazard}\PYG{o}{|} \PYG{n}{image}\PYG{p}{:}\PYG{p}{:} \PYG{n}{biohazard}\PYG{o}{.}\PYG{n}{png}
   \PYG{p}{:}\PYG{n}{align}\PYG{p}{:} \PYG{n}{middle}
   \PYG{p}{:}\PYG{n}{width}\PYG{p}{:} \PYG{l+m+mi}{12}
\end{sphinxVerbatimintable}
&
The \raisebox{-0.5\height}{\sphinxincludegraphics[width=12\sphinxpxdimen]{{biohazard}.png}} symbol must be used on containers used to
dispose of medical waste.
\\
\hline\sphinxstartmulticolumn{2}%
\begin{varwidth}[t]{\sphinxcolwidth{2}{2}}
Any text which begins with an explicit markup start but doesn’t use the syntax of any of the constructs above, is a comment.
\par
\vskip-\baselineskip\vbox{\hbox{\strut}}\end{varwidth}%
\sphinxstopmulticolumn
\\
\hline
\begin{sphinxVerbatimintable}[commandchars=\\\{\}]
\PYG{o}{.}\PYG{o}{.} \PYG{n}{This} \PYG{n}{text} \PYG{n}{will} \PYG{o+ow}{not} \PYG{n}{be} \PYG{n}{shown}
   \PYG{p}{(}\PYG{n}{but}\PYG{p}{,} \PYG{k}{for} \PYG{n}{instance}\PYG{p}{,} \PYG{o+ow}{in} \PYG{n}{HTML} \PYG{n}{might} \PYG{n}{be}
   \PYG{n}{rendered} \PYG{k}{as} \PYG{n}{an} \PYG{n}{HTML} \PYG{n}{comment}\PYG{p}{)}
\end{sphinxVerbatimintable}
&\\
\hline
\begin{sphinxVerbatimintable}[commandchars=\\\{\}]
\PYG{n}{An} \PYG{l+s+s2}{\PYGZdq{}}\PYG{l+s+s2}{empty comment}\PYG{l+s+s2}{\PYGZdq{}} \PYG{n}{does} \PYG{o+ow}{not}
\PYG{n}{consume} \PYG{n}{following} \PYG{n}{blocks}\PYG{o}{.}
\PYG{p}{(}\PYG{n}{An} \PYG{n}{empty} \PYG{n}{comment} \PYG{o+ow}{is} \PYG{l+s+s2}{\PYGZdq{}}\PYG{l+s+s2}{..}\PYG{l+s+s2}{\PYGZdq{}} \PYG{k}{with}
\PYG{n}{blank} \PYG{n}{lines} \PYG{n}{before} \PYG{o+ow}{and} \PYG{n}{after}\PYG{o}{.}\PYG{p}{)}

\PYG{o}{.}\PYG{o}{.}

        \PYG{n}{So} \PYG{n}{this} \PYG{n}{block} \PYG{o+ow}{is} \PYG{o+ow}{not} \PYG{l+s+s2}{\PYGZdq{}}\PYG{l+s+s2}{lost}\PYG{l+s+s2}{\PYGZdq{}}\PYG{p}{,}
        \PYG{n}{despite} \PYG{n}{its} \PYG{n}{indentation}\PYG{o}{.}
\end{sphinxVerbatimintable}
&
An “empty comment” does not
consume following blocks.
(An empty comment is “..” with
blank lines before and after.)
\begin{quote}

So this block is not “lost”,
despite its indentation.
\end{quote}
\\
\hline
\end{tabular}
\par
\sphinxattableend\end{savenotes}


\subsection{Credits}
\label{\detokenize{rst-cheatsheet/rst-cheatsheet:credits}}

\begin{savenotes}\sphinxattablestart
\centering
\begin{tabulary}{\linewidth}[t]{|T|T|}
\hline

CP Font from LiquiType:
&
\sphinxurl{http://www.liquitype.com/workshop/type\_design/cp-mono}
\\
\hline
Magnetic Balls V2 image by fdecomite:
&
\sphinxurl{http://www.flickr.com/photos/fdecomite/2926556794/}
\\
\hline
Sponsored by Net Managers
&
\sphinxurl{http://www.netmanagers.com.ar}
\\
\hline
Typeset using rst2pdf
&
\sphinxurl{http://rst2pdf.googlecode.com}
\\
\hline
\end{tabulary}
\par
\sphinxattableend\end{savenotes}


\begin{savenotes}\sphinxattablestart
\centering
\begin{tabulary}{\linewidth}[t]{|T|T|T|T|}
\hline

© \DUrole{small}{2009 Roberto Alsina <ralsina@netmanagers.com.ar>  /  Creative Commons Attribution\sphinxhyphen{}Noncommercial\sphinxhyphen{}Share Alike 2.5 Argentina License}
&
\raisebox{-0.5\height}{\sphinxincludegraphics[width=8bp]{{attrib}.png}} \DUrole{small}{Based on quickref.txt from docutils}
&
\raisebox{-0.5\height}{\sphinxincludegraphics[width=8bp]{{noncomm}.png}} \DUrole{small}{Non\sphinxhyphen{}Commercial}
&
\raisebox{-0.5\height}{\sphinxincludegraphics[width=8bp]{{sharealike}.png}} \DUrole{small}{Share Alike}
\\
\hline
\end{tabulary}
\par
\sphinxattableend\end{savenotes}


\section{Change Log}
\label{\detokenize{change-log:change-log}}\label{\detokenize{change-log::doc}}
\begin{sphinxadmonition}{warning}{Warning:}
Changes are not being tracked until a beta\sphinxhyphen{}quality release is made.
\end{sphinxadmonition}

The change log will appear here.


\section{License}
\label{\detokenize{license:license}}\label{\detokenize{license::doc}}
Copyright (c) 2019 Kevin Sheppard <\sphinxhref{mailto:kevin.k.sheppard@gmail.com}{kevin.k.sheppard@gmail.com}>

Derived from:
\begin{itemize}
\item {} 
Material for Mkdocs: Copyright (c) 2016\sphinxhyphen{}2019 Martin Donath <\sphinxhref{mailto:martin.donath@squidfunk.com}{martin.donath@squidfunk.com}>

\item {} 
Guzzle Sphinx Theme: Copyright (c) 2013 Michael Dowling <\sphinxhref{mailto:mtdowling@gmail.com}{mtdowling@gmail.com}>

\end{itemize}

Permission is hereby granted, free of charge, to any person obtaining a copy
of this software and associated documentation files (the “Software”), to deal
in the Software without restriction, including without limitation the rights
to use, copy, modify, merge, publish, distribute, sublicense, and/or sell
copies of the Software, and to permit persons to whom the Software is furnished
to do so, subject to the following conditions:

The above copyright notice and this permission notice shall be included in all
copies or substantial portions of the Software.

THE SOFTWARE IS PROVIDED “AS IS”, WITHOUT WARRANTY OF ANY KIND, EXPRESS OR
IMPLIED, INCLUDING BUT NOT LIMITED TO THE WARRANTIES OF MERCHANTABILITY, FITNESS
FOR A PARTICULAR PURPOSE AND NONINFRINGEMENT. IN NO EVENT SHALL THE AUTHORS OR
COPYRIGHT HOLDERS BE LIABLE FOR ANY CLAIM, DAMAGES OR OTHER LIABILITY, WHETHER
IN AN ACTION OF CONTRACT, TORT OR OTHERWISE, ARISING FROM, OUT OF OR IN CONNECTION
WITH THE SOFTWARE OR THE USE OR OTHER DEALINGS IN THE SOFTWARE.


\section{Index}
\label{\detokenize{index:index}}
\DUrole{xref,std,std-ref}{genindex}

\begin{sphinxthebibliography}{CIT2002}
\bibitem[CIT2002]{rst-cheatsheet/rst-cheatsheet:cit2002}
A citation
(as often used in journals).
\bibitem[this]{rst-cheatsheet/rst-cheatsheet:this}
here.
\end{sphinxthebibliography}



\renewcommand{\indexname}{Index}
\footnotesize\raggedright\printindex
\end{document}